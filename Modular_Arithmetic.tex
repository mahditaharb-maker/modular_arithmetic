\documentclass[11pt]{amsart}
\usepackage{amsmath, amsthm, amssymb}
\newcommand{\order}{\preceq}
\usepackage{setspace}
\usepackage{abstract}
\usepackage{enumitem}
\usepackage{pifont} 
\usepackage{xcolor}
\usepackage[margin=0.8in]{geometry}
\usepackage{newtxtext, newtxmath}  
\usepackage{microtype}             % Better typography
\usepackage{etoolbox}
\usepackage{amsfonts}
% Remove spacing from ALL list environments
\setlist{noitemsep, topsep=0pt, parsep=0pt, partopsep=0pt}
% Or customize each separately
\setlist[enumerate]{noitemsep, topsep=2pt, parsep=0pt}  % enumerate settings
\setlist[itemize]{noitemsep, topsep=0pt, parsep=0pt}     % itemize settings  
\setlist[description]{noitemsep, topsep=0pt, parsep=0pt} % description settings
% Define \Bbbk before newtxmath loads
\let\Bbbk\relax  % Release the definition
\usepackage{newtxtext, newtxmath}
% Math packages
\usepackage{mathtools}
\usepackage{bm}                    % Bold math symbols
\usepackage{physics}               % Physics notation 
\usepackage{graphicx}
\usepackage{hyperref}
\usepackage{cleveref}
\usepackage{algorithm, algpseudocode}
\usepackage{booktabs, array}       % Better tables
\usepackage{float}
\usepackage{longtable}
\usepackage{ragged2e}
\usepackage{algorithm}
\usepackage{algpseudocode}
\usepackage{subcaption}            % For figure management
\usepackage{titlesec}
\usepackage{titlecaps}  % For intelligent title casing
\usepackage{lipsum}
\usepackage{mathrsfs}
\usepackage{array}
\usepackage{tikz}
\usetikzlibrary{shapes,arrows,positioning,automata,fit,calc}
\usepackage{listings}
\usepackage{stmaryrd}
\hypersetup{
	colorlinks=true,
	linkcolor=blue,
	citecolor=red,
	urlcolor=magenta,
	pdftitle={Modular Arithmetic, Many-Valued Logic, and Algebraic Structures},
	pdfauthor={Brahimi, Mahdi. Tahar.}
}
% ============================================
% CENTERED ABSTRACT FOR AMSART
% ============================================
% Save the original abstract environment
\let\oldabstract\abstract
\let\endoldabstract\endabstract
% Redefine abstract for centered page
\renewenvironment{abstract}{%
	\clearpage
	\thispagestyle{empty} % No page number on abstract page
	\null\vfil % Vertical 
		\textbf{\Large Abstract}\par
		\vspace{1.5em}
		\begin{minipage}{0.8\textwidth} % Adjust width as needed
			\normalfont\normalsize
		}{%
		\end{minipage}
	\vfil\null % Complete vertical centering
	\clearpage
	\thispagestyle{plain} % Restore page style
}
% ============================================
% REST OF YOUR PREAMBLE (unchanged)
% ============================================
% Custom commands for exercise environment
\newcounter{exercise}
\newcounter{subexercise}[exercise]
\newenvironment{exerlist}
{\begin{enumerate}[label=\textbf{Exercise \arabic*:}, leftmargin=*]}
	{\end{enumerate}}
\newenvironment{subexerlist}
{\begin{enumerate}[label=(\alph*), leftmargin=*]}
	{\end{enumerate}}
% Section formatting
\titleformat{\section}
{\normalfont\normalsize\bfseries\scshape}
{\thesection}
{1em}
{}
\titleformat{\subsection}
{\normalfont\scriptsize\bfseries\itshape}
{\thesubsection}
{1em}
{}
\titleformat{\subsubsection}
{\normalfont\tiny\bfseries}
{\thesubsubsection}
{1em}
{}
\usepackage[style=alphabetic]{biblatex}
\addbibresource{references.bib}
\renewcommand*{\bibfont}{\normalfont\tiny}
\usetikzlibrary{shapes.geometric, arrows.meta, positioning, fit, backgrounds}
% Define column types for prooftable
\newcolumntype{S}{>{\raggedright\arraybackslash}p{0.75\textwidth}}
\newcolumntype{J}{>{\raggedright\arraybackslash}p{0.25\textwidth}}
% prooftable environment
\newenvironment{prooftable}
{%
	\setlength{\LTpre}{0pt}%
	\setlength{\LTpost}{0pt}%
	\noindent
	\renewcommand{\arraystretch}{1.3}%
	\begin{longtable}{r S J}%
		
		\textbf{Step} & \textbf{Statement} & \textbf{Justification} \\
		
		\endfirsthead
		
		\textbf{Step} & \textbf{Statement} & \textbf{Justification} \\
		
		\endhead
		
		\multicolumn{3}{r}{\textit{Continued on next page}} \\
		\endfoot
		
		\endlastfoot
	}
	{%
	\end{longtable}%
}
% Configure cleveref
\crefname{definition}{Definition}{Definitions}
\crefname{theorem}{Theorem}{Theorems}
\crefname{lemma}{Lemma}{Lemmas}
\crefname{proposition}{Proposition}{Propositions}
\crefname{corollary}{Corollary}{Corollaries}
\crefname{example}{Example}{Examples}
% Table of contents depth
\setcounter{tocdepth}{2}
\pagestyle{plain}
\pagenumbering{gobble}
\makeatletter
% Theorem style
\newtheoremstyle{break}
{\topsep}{\topsep}
{\normalfont}
{}
{\bfseries}
{.}
{\newline}
{\thmname{#1}\thmnumber{ #2}\thmnote{ (#3)}}
% Proof environment
\renewenvironment{proof}[1][\proofname]{%
	\par
	\noindent\textbf{#1.}\par
	\vspace{0.4em}%
	\noindent\ignorespaces
}{%
	\par\nobreak
	\hfill$\square$\par
}
\theoremstyle{break}
% Theorem environments
\newtheorem{theorem}{Theorem}[section]
\newtheorem{definition}{Definition}[section]
\newtheorem{proposition}{Proposition}[section]
\newtheorem{lemma}{Lemma}[section]
\newtheorem{corollary}{Corollary}[section]
\newtheorem{conjecture}{Conjecture}[section]
\newtheorem{axiom}{Axiom}[section]
\newtheorem{claim}{Claim}[section]
\newtheorem{condition}{Condition}[section]
\newtheorem{criterion}{Criterion}[section]
\newtheorem{problem}{Problem}[section]
\theoremstyle{break}
\newtheorem{example}{Example}
\newtheorem{exercises}{Exercise}
\newtheorem{solution}{\underline{Solution}}
\newtheorem{remark}{Remark}[section]
\newtheorem{notation}{Notation}[section]
\newtheorem{summary}{Summary}[section]
\newtheorem{application}{Application}
% Custom commands
\newcommand{\card}[1]{\sharp #1}
\newcommand{\powerset}{\mathcal{P}}
\newcommand{\set}[1]{\{#1\}}
\newcommand{\M}{\mathcal{M}}
\newcommand{\Mod}{\mathrm{Mod}}
\newcommand{\tneg}{\neg}
\newcommand{\tand}{\wedge}
\newcommand{\tor}{\vee}    
\newcommand{\timp}{\to}
\newcommand{\tequiv}{\leftrightarrow}
\newcommand{\withdate}[2]{#1\,(#2)}
\newcommand{\Zp}{\mathbb{Z}_p}
\newcommand{\Zn}{\mathbb{Z}_n}
\newcommand{\minorder}{\min_{\preceq}}
\newcommand{\maxorder}{\max_{\preceq}}
\newcommand{\Z}{\mathbb{Z}}
\newcommand{\Om}{\Omega}
\newcommand{\GF}{\mathrm{GF}}
\newcommand{\code}{\mathcal{C}}
\newcommand{\Hamming}{\mathcal{H}}
\newcommand{\sem}[1]{\llbracket #1 \rrbracket}
% ============================================
% TITLE AND AUTHOR
% ============================================
\title{Modular Arithmetic, Many-Valued Logic,\\ and Algebraic Structures}
\author{Brahimi, Mahdi. Tahar.}
% Title page
\makeatletter
\renewcommand{\maketitle}{%
	\thispagestyle{empty}%
	\vspace*{2cm}%
	\begin{center}%
		{\LARGE\bfseries Modular Arithmetic, Many-Valued Logic,\\ and Algebraic Structures \par}%
		\vspace{1.5cm}%
		{\large\bfseries Brahimi, Mahdi. Tahar.\par}%
		\vspace{0.5cm}%
		{\normalsize 
			Department of Mathematics\\
			University of Mohamed Boudiaf\\
			Msila, Algeria \par}%
		\vspace{0.3cm}%
		{\normalsize \texttt{mahditahar.brahimi@univ-msila.dz} \par}%
		\vspace{1cm}%
		{\large \@date \par}%
		\vspace{1cm}%
		% Subject class
		\begin{minipage}{0.8\textwidth}
			\raggedright
			\parindent=0pt
			\hangindent=3.5cm
		\tiny	\textbf{Mathematics Subject Classification:} 11A07, 11T71, 14H52 (Primary); 03B50, 16S34, 44A35 (Secondary)\par
		\end{minipage}	
		\vspace{0.5cm}
		% Keywords
		\begin{minipage}{0.8\textwidth}
			\raggedright
			\parindent=0pt
			\hangindent=1.5cm
		\tiny	\textbf{Keywords:} Hensel Lifting Theorem \textbf{HLT}, Łukasiewicz Many-Valued logic \textbf{MV}, Convolution algebras,\\ Elliptic curves, Chinese Remainder Theorem \textbf{CRT}, Modular Descent Argument $\lightning$,\, Automated Theorem Prover \textbf{ATP}.\par
		\end{minipage}
	\end{center}%
	\vspace{2cm}%
	\thispagestyle{empty}%
}
\makeatother
% Example environments
\newenvironment{exampleitem}[1][]
{\par\vspace{12pt}\noindent%
	\refstepcounter{example}%
	\textbf{Example \theexample}%
	\if\relax\detokenize{#1}\relax\else\ \textnormal{[#1]}\fi%
	\textbf{:}\nopagebreak\vspace{8pt}\par\noindent\ignorespaces}
{\par\vspace{12pt}}

\newlist{exampleitems}{itemize}{1}
\setlist[exampleitems]{
	label=\textbullet,
	leftmargin=*,
	itemsep=5pt,
	parsep=0pt,
	topsep=5pt,
	partopsep=0pt
}
\newcommand{\setexamplespacing}[1]{\setlength{\examplespacing}{#1}}
\numberwithin{equation}{section}
% TOC formatting
\makeatletter
\setcounter{tocdepth}{2}
\def\l@section{\@tocline{2}{0pt}{0em}{1em}{}}
\def\l@subsection{\@tocline{2}{0pt}{2em}{2em}{}}
\def\l@subsubsection{\@tocline{3}{0pt}{3.8em}{4.0em}{}}
\makeatother

% Spacing
\titlespacing*{\section}{0pt}{0pt}{0pt}
\titlespacing*{\subsection}{0pt}{0pt}{0pt}
\titlespacing*{\subsubsection}{0pt}{0pt}{0pt}
\setlength{\parskip}{0pt}
\newcommand{\correct}{\textcolor{green}{\ding{51}}}
\newcommand{\wrong}{\textcolor{red}{\ding{55}}}
\renewcommand{\abstractnamefont}{\normalfont\Large\bfseries\raggedright}

\titleformat{\part}[display]
{\normalfont\normalsize\bfseries}  % Format for regular parts
{\partname\ \thepart}        % Label format
{14pt}                       % Separation between label and title
{\Large}                      % Format for title text


\begin{document}

	\maketitle
\newpage
\thispagestyle{empty}
		\begin{abstract}
			This report presents a unified mathematical framework connecting modular arithmetic, Many-Valued logic, convolution algebras, and elliptic curves over finite fields.\\ We establish fundamental isomorphisms between algebraic structures, including the Prime-Modular Logic-Set Isomorphism linking Many-Valued Algebras to modular set theory.\\ The work develops constructive methods for modular calculus, binary expansions, and  parametric congruences. All theorems are presented with detailed step-by-step proofs using a structured \texttt{prooftable} environment, ensuring clarity and verifiability. Applications span cryptography, coding theory, and computational mathematics.
		\end{abstract}
\newpage
% Table of Contents page
\thispagestyle{empty}
\setstretch{2}
% Automatic Table of Contents
\tableofcontents
\newpage
% Reset page numbering for main content
\pagenumbering{arabic}
\setcounter{page}{1}
\newpage
% ========== INTRODUCTION ==========
\scriptsize
\setstretch{1.2}
\section*{Acknowledgments}
The author gratefully acknowledges the pioneering work of Prof. Norman J. Wildberger, whose development of rational trigonometry and algebraic geometry has significantly influenced the foundational perspectives of this work. His introduction of quadrances and spreads as algebraic alternatives to traditional trigonometric concepts \cite{wildberger2002,wildberger2005}, along with subsequent developments in universal geometry \cite{wildberger2008,wildberger2018} and computational foundations \cite{wildberger2004foundations}, provided important conceptual frameworks for approaching geometric and number-theoretic problems algebraically. His recent contributions on hyper-Catalan series and polynomial solutions \cite{wildberger2025hyper,RubineWildberger2025HyperCatalan}, as well as investigations into combinatorial structures \cite{LimantaWildberger2021}, have also informed aspects of this research. The accessible presentation of algebraic calculus through his lecture series \cite{wildberger2024_video} provided valuable pedagogical insights.\\
Additionally, the author acknowledges the use of artificial intelligence tools in 
the preparation of this document. Specifically, the DeepSeek \cite{deepseek2024} AI 
Assistant\footnote{\tiny For ethical transparency, note that this AI tool 
	was used for bibliographic organization, reference deduplication, 
	automated proof formatting, and editorial suggestions, but all 
	substantive intellectual content and analysis remains the author's 
	own work.} was employed to assist with:
\begin{enumerate}
	\item Reference management and bibliographic organization
	\item Development of automated proof formats and environments
	\item LaTeX code optimization for mathematical typesetting
	\item Editorial suggestions for clarity and consistency
\end{enumerate}
The proof table environment (see 
Appendices  ~\ref{sec:proof-table-code} and ~\ref{sec:prooftables-theorems}) was developed 
with AI assistance to provide a structured, step-by-step presentation 
of mathematical proofs with automatic page continuation.
\section{Introduction}
Modular arithmetic provides the foundational language for discrete mathematics, computer science, and modern cryptography. This work synthesizes several advanced topics that originate from modular concepts, revealing deep interconnections between seemingly disparate mathematical domains.
\subsection{Motivation and Context}
The Chinese Remainder Theorem (\textbf{CRT}) serves as a bridge between global modular equations and their local prime-power components.\\ This decomposition principle extends beyond basic number theory to influence algebraic structures, logical systems, and geometric objects defined over finite fields. Our research explores these extensions systematically.
\subsection{Main Contributions}
\begin{enumerate}[noitemsep, topsep=0pt, parsep=0pt, partopsep=0pt]
	\item \textbf{Logic-Arithmetic Synthesis}: Establishment of the Prime-Modular Logic-Set Isomorphism, connecting Many-Valued logics with modular set algebras via prime modulus constraints.
	\item \textbf{Constructive Methods}: Development of algorithmic frameworks for parametric congruences and greedy binary expansions with guaranteed error bounds.
	\item \textbf{Algebraic Unification}: Characterization of convolution algebras (total, cyclic, and truncated) with explicit isomorphism theorems linking them to polynomial rings via Discrete Fourier Transform.
	\item \textbf{Geometric Applications}: Integration of elliptic curve theory with modular arithmetic, particularly through the Hasse bound and Frobenius endomorphism properties.
\end{enumerate}
\subsection{Proof Methodology}
A distinctive feature of this work is the use of structured \texttt{prooftable} environments that present mathematical proofs as sequences of justified steps.\\ This approach enhances readability, supports formal verification, and makes complex arguments accessible for pedagogical purposes.
% ========== SECTION 1: DEFINITIONS =========
\section{Modular arithmetic Results}
\label{sec:Definitions}
\subsection{Modular Arithmetic Foundations}
\label{subsec:Modular-Arithmetic-Foundations}
\subsubsection{Basic Modular Concepts}
\begin{definition}[Congruence,\, \cite{hardy2008introduction,davenport2008higher}] 
	For integers $a, b \in \mathbb{Z}$ and modulus $n \in \mathbb{N}$, we write $a \equiv b \pmod{n}$ if and only if $n$ divides $a-b$.\\ Equivalently, $a = b + kn$ for some integer $k$.
\end{definition}
\begin{definition}[Residue Class,\, \cite{hardy2008introduction,niven1991introduction}] 
	The \emph{residue class} (or congruence class) of $a$ modulo $n$ is the set
	\begin{equation}
	[a]_n = \{ b \in \mathbb{Z} \mid b \equiv a \pmod{n} \} = \{ a + kn \mid k \in \mathbb{Z} \}.
	\end{equation}
\end{definition}
\begin{definition}[Quotient Ring $\mathbb{Z}/n\mathbb{Z}$, \, \cite{dummit2003abstract,artin2010algebra}] 
	The set of all residue classes modulo $n$ forms the quotient ring $\mathbb{Z}/n\mathbb{Z} = \{ [0]_n, [1]_n, \dots, [n-1]_n \}$ with operations:
	\begin{equation}
	[a]_n + [b]_n = [a+b]_n, \quad [a]_n \cdot [b]_n = [ab]_n.
	\end{equation}
\end{definition}
\begin{definition}[Complete Residue System, \, \cite{hardy2008introduction,niven1991introduction}] 
	A set $\{a_1, a_2, \dots, a_n\} \subseteq \mathbb{Z}$ is a complete residue system modulo $n$ if every integer is congruent modulo $n$ to exactly one of the $a_i$.
\end{definition}
\begin{definition}[Reduced Residue System, \, \cite{hardy2008introduction,burton2010elementary}] 
	The set of residue classes modulo $n$ that are relatively prime to $n$ forms the multiplicative group $(\mathbb{Z}/n\mathbb{Z})^\times$. 
\end{definition}
\subsubsection{Advanced Modular Structures}
\begin{definition}[Chinese Remainder Theorem Decomposition, \, \cite{ding1996chinese,hardy2008introduction}]
	For pairwise coprime moduli $n_1, n_2, \dots, n_k$, the natural map
	\begin{equation}
	\phi: \mathbb{Z}/n\mathbb{Z} \to \mathbb{Z}/n_1\mathbb{Z} \times \cdots \times \mathbb{Z}/n_k\mathbb{Z}, \quad \phi([x]_n) = ([x]_{n_1}, \dots, [x]_{n_k})
	\end{equation}
	is a ring isomorphism, where $n = \prod_{i=1}^k n_i$.
\end{definition}
\begin{definition}[Mixed Modular System, \, \cite{ding1996chinese,hardy2008introduction}] 
	For composite modulus $M = \prod_{i=1}^k p_i^{e_i}$ with prime powers $p_i^{e_i}$, the ring decomposes as
	\begin{equation}
	\mathbb{Z}_M \cong \bigoplus_{i=1}^k \mathbb{Z}_{p_i^{e_i}} \quad \text{(via Chinese Remainder Theorem)}.
	\end{equation}
\end{definition}
\begin{definition}[Finite Field $\mathbb{F}_p$, \, \cite{lidl1997finite,dummit2003abstract}] 
	For prime $p$, the quotient ring $\mathbb{Z}/p\mathbb{Z}$ forms a finite field with $p$ elements, denoted $\mathbb{F}_p$.
\end{definition}
\begin{definition}[Euler's Totient Function, \, \cite{hardy2008introduction,apostol1976introduction}] 
	For $n \in \mathbb{N}$, Euler's totient function $\varphi(n)$ counts integers $1 \leq k \leq n$ with $\gcd(k,n) = 1$.
\end{definition}
\begin{definition}[Legendre Symbol, \, \cite{hardy2008introduction,niven1991introduction}] 
	For odd prime $p$ and integer $a$, the \emph{Legendre symbol} is
	\begin{equation}
	\left(\frac{a}{p}\right) = \begin{cases}
	0 & \text{if } p \mid a, \\
	1 & \text{if } a \text{ is quadratic residue modulo } p, \\
	-1 & \text{if } a \text{ is quadratic non-residue modulo } p.
	\end{cases}
	\end{equation}
\end{definition}
\subsubsection{Chinese Remainder Theorem}
\begin{theorem}[Chinese Remainder Theorem (\textbf{CRT}), \, \cite{ding1996chinese,hardy2008introduction}] 
	Let $n_1, n_2, \dots, n_k \in \mathbb{N}$ be pairwise coprime integers, and let $n = \prod_{i=1}^k n_i$. Then the natural map
	\begin{equation}
	\phi: \mathbb{Z}/n\mathbb{Z} \to \mathbb{Z}/n_1\mathbb{Z} \times \cdots \times \mathbb{Z}/n_k\mathbb{Z}, \quad \phi([x]_n) = ([x]_{n_1}, \dots, [x]_{n_k})
	\end{equation}
	is a ring isomorphism.
\end{theorem}
\begin{corollary}[System of Linear Congruences, \, \cite{ding1996chinese,niven1991introduction}] 
	Given pairwise coprime moduli $n_1, \dots, n_k$ and integers $a_1, \dots, a_k$, the system
	\begin{equation}
	x \equiv a_i \pmod{n_i}, \quad i = 1, \dots, k
	\end{equation}
	has a unique solution modulo $n = n_1 \cdots n_k$.
\end{corollary}
\subsubsection{Fermat's and Euler's Theorems}
\begin{theorem}[Fermat's Little Theorem, \, \cite{hardy2008introduction,burton2010elementary}] 
	If $p$ is prime and $a \in \mathbb{Z}$ with $p \nmid a$, then $a^{p-1} \equiv 1 \pmod{p}$.\\
	Equivalently, $a^p \equiv a \pmod{p}$ for all $a \in \mathbb{Z}$.
\end{theorem}
\begin{theorem}[Euler's Theorem, \, \cite{hardy2008introduction,apostol1976introduction}] 
	For $n \in \mathbb{N}$ and $a \in \mathbb{Z}$ with $\gcd(a,n) = 1$, then
	\begin{equation}
	a^{\varphi(n)} \equiv 1 \pmod{n},
	\end{equation}
	where $\varphi$ is Euler's totient function.
\end{theorem}
\subsubsection{Wilson's Theorem}
\begin{theorem}[Wilson's Theorem, \, \cite{hardy2008introduction,burton2010elementary}] 
	For prime $p$, $(p-1)! \equiv -1 \pmod{p}$.
\end{theorem}
\subsection{Hensel's Lifting Theorem}



\subsubsection{Preliminary Definitions}

We begin by recalling the essential definitions from Hensel lifting theory:

\begin{definition}[$p$-adic Valuation,\, \cite{koblitz1996p}] \label{def:valuation}
	For a prime $p$ and nonzero $a \in \mathbb{Z}$, $v_p(a)$ is the largest $n \geq 0$ such that $p^n \mid a$. For $a/b \in \mathbb{Q}$, $v_p(a/b) = v_p(a) - v_p(b)$.\\ The valuation $v_p(0)$ is not defined, but we adopt the conventions that for any $n \in \mathbb{Z}$, $p^n \mid 0$ and $v_p(0) > n$.
\end{definition}
\begin{definition}[$p$-adic Absolute Value, \, \cite{ostrowski1916uber}] \label{def:absvalue}
	$|x|_p = p^{-v_p(x)}$ for $x \neq 0$, and $|0|_p = 0$. This satisfies the strong triangle inequality $|x+y|_p \leq \max(|x|_p, |y|_p)$ .
\end{definition}

\begin{definition}[$\mathbb{Z}_p$ - $p$-adic Integers,\, \cite{gouvea1997p}] \label{def:Zp}
	$\mathbb{Z}_p = \{x \in \mathbb{Q}_p : |x|_p \leq 1\} = \{x \in \mathbb{Q}_p : v_p(x) \geq 0\}$, the completion of $\mathbb{Z}$ with respect to $|\cdot|_p$.
\end{definition}

\begin{definition}[Residue Modulo $p^n$,\, \cite{serre2012course}] \label{def:residue}
	For $a \in \mathbb{Z}_p$, $a \bmod p^n$ is the unique integer in $\{0,1,\dots,p^n-1\}$ congruent to $a$ modulo $p^n$. We write $a \equiv b \pmod{p^n}$ if $p^n \mid (a-b)$.
\end{definition}

\begin{definition}[Simple Root Modulo $p$,\, \cite{hensel1904theorie}] \label{def:simple-root}
	$a_0 \in \mathbb{Z}_p$ is a simple root modulo $p$ of $f(x) \in \mathbb{Z}_p[x]$ if:
	\begin{enumerate}[label=(\roman*)]
		\item $f(a_0) \equiv 0 \pmod{p}$
		\item $f'(a_0) \not\equiv 0 \pmod{p}$
	\end{enumerate}
\end{definition}

\subsubsection{Main Theorem}

\begin{theorem}[Hensel's Lifting Theorem] \label{thm:hensel-main}
	Let $f(x) \in \mathbb{Z}_p[x]$ and $a_0 \in \mathbb{Z}_p$ satisfy:
	\begin{enumerate}
		\item $f(a_0) \equiv 0 \pmod{p}$ (Definition \ref{def:simple-root}\, (i))
		\item $f'(a_0) \not\equiv 0 \pmod{p}$ (Definition \ref{def:simple-root}\, (ii))
	\end{enumerate}
	Then there exists a unique $\alpha \in \mathbb{Z}_p$ such that:
	\[
	f(\alpha) = 0 \quad \text{and} \quad \alpha \equiv a_0 \pmod{p}.
	\]
\end{theorem}

\begin{proof}
We present the proof in two parts: existence and uniqueness.
\text{\ }

\vspace{5pt}
\underline{\textbf{Part 1: Existence Proof}}
\text{\ }

\vspace{5pt}
We construct the root $\alpha$ recursively using Hensel lifting.
\begin{prooftable}
	1 & Let $f(x) = \sum_{i=0}^n c_i x^i \in \mathbb{Z}_p[x]$ with $c_i \in \mathbb{Z}_p$. & Given polynomial. \\
	2 & By Definition \ref{def:simple-root}\, (i): $f(a_0) \equiv 0 \pmod{p}$. & Hypothesis. \\
	3 & By Definition \ref{def:simple-root}\, (ii): $f'(a_0) \not\equiv 0 \pmod{p}$, so $v_p(f'(a_0)) = 0$. & Non-singular condition. \\
	4 & Define $a_1 = a_0$. Then $f(a_1) \equiv 0 \pmod{p}$ and $a_1 \equiv a_0 \pmod{p}$. & Base case for induction. \\
	5 & \textbf{Induction Hypothesis}: Assume for some $k \geq 1$, we have constructed $a_k \in \mathbb{Z}_p$ such that: & Setup for recursive construction. \\
	& (i) $f(a_k) \equiv 0 \pmod{p^k}$ & \\
	& (ii) $a_k \equiv a_0 \pmod{p}$ & \\
	6 & Since $f(a_k) \equiv 0 \pmod{p^k}$, by Definition \ref{def:residue}, there exists $m_k \in \mathbb{Z}_p$ such that $f(a_k) = p^k m_k$. & Factorization of $f(a_k)$. \\
	7 & Consider the Taylor expansion of $f$ at $a_k$: & Analytic preparation. \\
	& $f(a_k + t p^k) = f(a_k) + f'(a_k) t p^k + (t p^k)^2 g(t)$ & \\
	& where $g(t) \in \mathbb{Z}_p[t]$ is a polynomial with integer $p$-adic coefficients. & Taylor's theorem in $\mathbb{Z}_p$ \cite{lang2002algebra}. \\
	8 & Substitute $f(a_k) = p^k m_k$: & Using step 6. \\
	& $f(a_k + t p^k) = p^k m_k + f'(a_k) t p^k + p^{2k} t^2 g(t)$ & \\
	9 & We want to choose $t \in \{0,1,\dots,p-1\}$ such that $f(a_k + t p^k) \equiv 0 \pmod{p^{k+1}}$. & Goal for lifting. \\
	10 & Modulo $p^{k+1}$, we need: $p^k m_k + f'(a_k) t p^k \equiv 0 \pmod{p^{k+1}}$. & Ignoring $p^{2k}$ term since $2k \geq k+1$ for $k \geq 1$. \\
	11 & Divide by $p^k$ (valid since $p^k$ is a unit in $\mathbb{Q}_p$): & Algebraic manipulation. \\
	& $m_k + f'(a_k) t \equiv 0 \pmod{p}$ & \\
	12 & Since $a_k \equiv a_0 \pmod{p}$, we have $f'(a_k) \equiv f'(a_0) \pmod{p}$. & Polynomial congruence \cite{neukirch1999algebraic}. \\
	13 & By hypothesis, $f'(a_0) \not\equiv 0 \pmod{p}$, so $v_p(f'(a_0)) = 0$ and thus $v_p(f'(a_k)) = 0$. & Preservation of valuation. \\
	14 & Therefore, $f'(a_k)$ has a multiplicative inverse modulo $p$. & Since $f'(a_k) \not\equiv 0 \pmod{p}$. \\
	15 & The congruence $m_k + f'(a_k) t \equiv 0 \pmod{p}$ has a unique solution: & Solving linear congruence. \\
	& $t \equiv -m_k \cdot (f'(a_k)^{-1} \bmod p) \pmod{p}$ & \\
	& where the inverse is taken in $\mathbb{F}_p = \mathbb{Z}/p\mathbb{Z}$. & \\
	16 & Choose $t_k \in \{0,1,\dots,p-1\}$ as this unique solution. & Canonical representative. \\
	17 & Define $a_{k+1} = a_k + t_k p^k$. & Recursive definition. \\
	18 & Then by construction: & Verification. \\
	& (i) $f(a_{k+1}) \equiv 0 \pmod{p^{k+1}}$ & From steps 8-16. \\
	& (ii) $a_{k+1} \equiv a_k \pmod{p^k}$ & Since $a_{k+1} - a_k = t_k p^k$. \\
	& (iii) $a_{k+1} \equiv a_0 \pmod{p}$ & Since $a_k \equiv a_0 \pmod{p}$ and $p \mid p^k$. \\
	19 & By induction, we have constructed a sequence $(a_k)_{k\geq 1}$ in $\mathbb{Z}_p$ such that: & Completion of construction. \\
	& (i) $f(a_k) \equiv 0 \pmod{p^k}$ for all $k \geq 1$ & \\
	& (ii) $a_{k+1} \equiv a_k \pmod{p^k}$ for all $k \geq 1$ & \\
	20 & Claim: $(a_k)$ is a Cauchy sequence in $\mathbb{Z}_p$. & Need to show convergence. \\
	21 & For any $\epsilon > 0$, choose $N > -\log_p \epsilon$. Then for all $m > n \geq N$: & $\epsilon$-$\delta$ argument. \\
	& $|a_m - a_n|_p = \left|\sum_{k=n}^{m-1} (a_{k+1} - a_k)\right|_p$ & Telescoping sum. \\
	22 & By the strong triangle inequality (Definition \ref{def:absvalue}): & Non-Archimedean property. \\
	& $|a_m - a_n|_p \leq \max_{n \leq k < m} |a_{k+1} - a_k|_p$ & \\
	23 & From step 18(ii): $a_{k+1} \equiv a_k \pmod{p^k}$, so $p^k \mid (a_{k+1} - a_k)$. & Congruence property. \\
	24 & By Definition \ref{def:valuation}: $v_p(a_{k+1} - a_k) \geq k$. & Valuation of difference. \\
	25 & By Definition \ref{def:absvalue}: $|a_{k+1} - a_k|_p \leq p^{-k}$. & Conversion to absolute value. \\
	26 & Thus for $m > n \geq N$: $|a_m - a_n|_p \leq p^{-n} \leq p^{-N} < \epsilon$. & Combining estimates. \\
	27 & Therefore, $(a_k)$ is Cauchy in $\mathbb{Z}_p$. & Definition of Cauchy sequence. \\
	28 & Since $\mathbb{Z}_p$ is complete (Definition \ref{def:Zp}), $(a_k)$ converges to some $\alpha \in \mathbb{Z}_p$. & Completeness property. \\
	29 & We need to show $f(\alpha) = 0$. Consider $|f(\alpha)|_p$: & Verification that $\alpha$ is a root. \\
	& $|f(\alpha)|_p = |f(\alpha) - f(a_k) + f(a_k)|_p$ & Add and subtract. \\
	30 & By the strong triangle inequality: & Non-Archimedean property. \\
	& $|f(\alpha)|_p \leq \max(|f(\alpha) - f(a_k)|_p, |f(a_k)|_p)$ & \\
	31 & Since $f$ is a polynomial with coefficients in $\mathbb{Z}_p$, it is continuous. & Polynomial continuity \cite{schikhof1984ultrametric}. \\
	32 & \text{In the limit of large } k, $a_k \to \alpha$, so $f(a_k) \to f(\alpha)$ and thus $|f(\alpha) - f(a_k)|_p \to 0$. & Continuity argument. \\
	33 & From step 19(i): $f(a_k) \equiv 0 \pmod{p^k}$, so $v_p(f(a_k)) \geq k$. & Congruence implies valuation bound. \\
	34 & Thus $|f(a_k)|_p \leq p^{-k} \to 0$ as k is large enough & Conversion to absolute value. \\
	35 & Taking limits in step 30: $|f(\alpha)|_p \leq \max(0,0) = 0$. & Limit of bounds. \\
	36 & Therefore $|f(\alpha)|_p = 0$, which by Definition \ref{def:absvalue} means $f(\alpha) = 0$. & Positive definiteness. \\
	37 & Also, since $a_k \equiv a_0 \pmod{p}$ for all $k$, we have $\alpha \equiv a_0 \pmod{p}$. & Preservation of congruence in limit. \\
	38 & This completes the existence part of the proof. & $\square$ (existence) \\
\end{prooftable}
\vspace{5pt}
\underline{\textbf{Part 2: Uniqueness Proof}}
\vspace{5pt}
\begin{prooftable}
	39 & Now suppose $\beta \in \mathbb{Z}_p$ also satisfies $f(\beta) = 0$ and $\beta \equiv a_0 \pmod{p}$. & Assumption for contradiction. \\
	40 & Consider the difference $\alpha - \beta$. We want to show $\alpha = \beta$. & Goal. \\
	41 & Since $\alpha \equiv a_0 \pmod{p}$ and $\beta \equiv a_0 \pmod{p}$, we have $\alpha \equiv \beta \pmod{p}$. & Transitivity of congruence. \\
	42 & Thus $p \mid (\alpha - \beta)$, so $v_p(\alpha - \beta) \geq 1$. & Valuation bound. \\
	43 & We will show by induction that $v_p(\alpha - \beta) \geq k$ for all $k \geq 1$. & Strategy. \\
	44 & \textbf{Base case}: $k = 1$ is established in step 42: $v_p(\alpha - \beta) \geq 1$. & Initial step. \\
	45 & \textbf{Induction hypothesis}: Assume $v_p(\alpha - \beta) \geq k$ for some $k \geq 1$. & Inductive assumption. \\
	46 & Write $\beta = \alpha + p^k \delta$ for some $\delta \in \mathbb{Z}_p$. & Representation using valuation. \\
	47 & Consider the Taylor expansion of $f$ at $\alpha$: & Analytic method. \\
	& $f(\beta) = f(\alpha + p^k \delta) = f(\alpha) + f'(\alpha) p^k \delta + (p^k \delta)^2 h(\delta)$ & \\
	& where $h(\delta) \in \mathbb{Z}_p[\delta]$ is a polynomial. & Taylor's theorem. \\
	48 & Since $f(\alpha) = 0$ and $f(\beta) = 0$, we have: & Using root conditions. \\
	& $0 = f'(\alpha) p^k \delta + p^{2k} \delta^2 h(\delta)$ & \\
	49 & Divide by $p^k$ (valid in $\mathbb{Q}_p$): & Algebraic manipulation. \\
	& $0 = f'(\alpha) \delta + p^k \delta^2 h(\delta)$ & \\
	50 & Rearranging: $f'(\alpha) \delta = -p^k \delta^2 h(\delta)$. & Equation rearrangement. \\
	51 & Take $p$-adic valuations of both sides: & Valuation analysis. \\
	& $v_p(f'(\alpha) \delta) = v_p(f'(\alpha)) + v_p(\delta)$ & Additivity of valuation. \\
	52 & Since $\alpha \equiv a_0 \pmod{p}$ and $f'(a_0) \not\equiv 0 \pmod{p}$, we have $f'(\alpha) \equiv f'(a_0) \pmod{p}$. & Polynomial congruence. \\
	53 & Thus $v_p(f'(\alpha)) = 0$ (since $f'(a_0) \not\equiv 0 \pmod{p}$ implies $v_p(f'(a_0)) = 0$). & Preservation of zero valuation. \\
	54 & So $v_p(f'(\alpha) \delta) = v_p(\delta)$. & From step 53. \\
	55 & For the right side: $v_p(p^k \delta^2 h(\delta)) = k + 2v_p(\delta) + v_p(h(\delta))$. & Valuation calculation. \\
	56 & Since $h(\delta)$ has coefficients in $\mathbb{Z}_p$, $v_p(h(\delta)) \geq 0$. & Integral coefficients. \\
	57 & Thus from step 50: $v_p(\delta) = v_p(f'(\alpha) \delta) = v_p(p^k \delta^2 h(\delta)) \geq k + 2v_p(\delta)$. & Combining estimates. \\
	58 & Rearranging: $v_p(\delta) \geq k + 2v_p(\delta) \Rightarrow 0 \geq k + v_p(\delta)$. & Inequality manipulation. \\
	59 & This implies $v_p(\delta) \geq 1$ (since $k \geq 1$). & Lower bound on $v_p(\delta)$. \\
	60 & Recall $\beta = \alpha + p^k \delta$, so $\alpha - \beta = -p^k \delta$. & Original representation. \\
	61 & Then $v_p(\alpha - \beta) = v_p(p^k \delta) = k + v_p(\delta) \geq k + 1$. & Valuation calculation. \\
	62 & This completes the induction: $v_p(\alpha - \beta) \geq k$ implies $v_p(\alpha - \beta) \geq k+1$. & Inductive step proved. \\
	63 & By induction, $v_p(\alpha - \beta) \geq k$ for all $k \geq 1$. & Conclusion from induction. \\
	64 & Therefore, the set $\{v_p(\alpha - \beta)\}$ is unbounded above, which forces $\alpha - \beta = 0$ by Definition \ref{def:valuation}. & Unbounded valuation implies zero. \\
	65 & Hence $\alpha = \beta$, proving uniqueness. & $\square$ (uniqueness) \\
\end{prooftable}
\end{proof}
The proof presented here synthesizes approaches from:
\vspace{5pt}
\begin{itemize}
	\item Hensel's original iterative construction \cite{hensel1904theorie}
	\item Modern $p$-adic analysis techniques \cite{koblitz1996p, gouvea1997p}
	\item Algebraic number theory perspectives \cite{neukirch1999algebraic, lang1994algebraic}
	\item Computational number theory methods \cite{cohen1993course, bach1996algorithmic}
\end{itemize}
\vspace{5pt}
The proof demonstrates several key features of $p$-adic analysis:
\vspace{5pt}
\begin{enumerate}
	\item The non-Archimedean property simplifies convergence arguments
	\item Taylor expansions work nicely in $\mathbb{Z}_p$ due to integrality of coefficients
	\item The valuation provides a natural measure of approximation quality
	\item Completeness of $\mathbb{Z}_p$ ensures limits exist
\end{enumerate}
\subsubsection{Newton Iteration Form}

\begin{corollary}[Newton Iteration Convergence] \label{cor:newton}
	Under the hypotheses of Theorem \ref{thm:hensel-main}, if additionally $|f(a_0)|_p < |f'(a_0)|_p^2$, then the Newton iteration:
	\[
	a_{n+1} = a_n - \frac{f(a_n)}{f'(a_n)}
	\]
	converges quadratically to $\alpha$ in $\mathbb{Z}_p$.
\end{corollary}

\begin{proof}
	\begin{prooftable}
		66 & Define the Newton map $N(x) = x - f(x)/f'(x)$. & Newton's method. \\
		67 & From Taylor expansion: $f(N(x)) = \frac{1}{2}f''(\xi)(N(x)-x)^2$ for some $\xi$. & Second order expansion \cite{newton1669method}. \\
		68 & Thus $|f(N(x))|_p \leq |N(x)-x|_p^2 = |f(x)/f'(x)|_p^2$. & Non-Archimedean inequality. \\
		69 & Under condition $|f(a_0)|_p < |f'(a_0)|_p^2$, induction shows $|f(a_n)|_p \to 0$ quadratically. & Convergence analysis \cite{cohen1993course}. \\
		70 & Moreover, $|a_{n+1} - a_n|_p = |f(a_n)/f'(a_n)|_p \to 0$ quadratically. & Step size estimate. \\
		71 & Thus $(a_n)$ is Cauchy and converges to some limit $\alpha' \in \mathbb{Z}_p$. & Completeness argument. \\
		72 & By continuity, $f(\alpha') = 0$ and $\alpha' \equiv a_0 \pmod{p}$. & Limit properties. \\
		73 & By uniqueness in Theorem \ref{thm:hensel-main}, $\alpha' = \alpha$. & Identification with Hensel lift. \\
		74 & Quadratic convergence: $|a_{n+1} - \alpha|_p \leq C|a_n - \alpha|_p^2$ for some $C > 0$. & Standard Newton convergence \cite{bach1996algorithmic}. \\
	\end{prooftable}
\end{proof}

\subsubsection{Application of the Method}

\begin{example}[Square Root of 2 in $\mathbb{Z}_7$] \label{ex:sqrt2-Z7-complete}
	Find $\alpha \in \mathbb{Z}_7$ such that $\alpha^2 = 2$ using the constructive Hensel lifting proof. Compute the 7-adic expansion up to $7^4$.
\end{example}

\begin{proof}[Complete Solution]
	Let $f(x) = x^2 - 2 \in \mathbb{Z}_7[x]$. We seek $\alpha \in \mathbb{Z}_7$ satisfying $f(\alpha) = 0$.
	
	\begin{prooftable}
		1 & \textbf{Step 1: Find initial approximation modulo 7.} & \\
		& Test values in $\{0,1,2,3,4,5,6\}$ modulo 7: & \\
		& $f(0) = -2 \equiv 5 \pmod{7}$, $f(1) = -1 \equiv 6 \pmod{7}$, & Direct computation. \\
		& $f(2) = 2 \equiv 2 \pmod{7}$, $f(3) = 7 \equiv 0 \pmod{7}$, & \\
		& $f(4) = 14 \equiv 0 \pmod{7}$, $f(5) = 23 \equiv 2 \pmod{7}$, & \\
		& $f(6) = 34 \equiv 6 \pmod{7}$. & \\
		2 & Thus $a_0 = 3$ and $a_0 = 4$ are both roots modulo 7. & Two initial choices. \\
		3 & Choose $a_0 = 3$ (the other choice $a_0 = 4$ gives $-\alpha$). & Arbitrary selection. \\
		4 & Check derivative: $f'(x) = 2x$, so $f'(3) = 6 \not\equiv 0 \pmod{7}$. & Non-singular condition holds. \\
		5 & \textbf{Step 2: First lifting (mod $7^2$).} & \\
		& Set $a_1 = 3$. We have $f(a_1) = 7 = 7^1 \cdot 1$, so $m_1 = 1$. & $f(a_1) = p^1 \cdot m_1$ form. \\
		6 & Need $t_1 \in \{0,\dots,6\}$ solving $1 + f'(3)t_1 \equiv 0 \pmod{7}$. & Hensel equation. \\
		7 & $1 + 6t_1 \equiv 0 \pmod{7} \Rightarrow 6t_1 \equiv 6 \pmod{7}$. & Modular equation. \\
		8 & Since $6^{-1} \equiv 6 \pmod{7}$ ($6 \times 6 = 36 \equiv 1 \pmod{7}$), & Compute inverse. \\
		& $t_1 \equiv 6 \times 6 \equiv 36 \equiv 1 \pmod{7}$. & \\
		9 & Choose $t_1 = 1$ (the unique representative in $\{0,\dots,6\}$). & Canonical choice. \\
		10 & Then $a_2 = a_1 + t_1 \cdot 7^1 = 3 + 1 \cdot 7 = 10$. & First lift. \\
		11 & Verify: $f(10) = 100 - 2 = 98 = 2 \cdot 49 = 7^2 \cdot 2$, & Check congruence: \\
		& so $f(10) \equiv 0 \pmod{49}$, correct. & $98 \div 49 = 2$ remainder 0. \\
		12 & \textbf{Step 3: Second lifting (mod $7^3$).} & \\
		& Now $f(a_2) = 98 = 7^2 \cdot 2$, so $m_2 = 2$. & Extract $m_2$. \\
		13 & Need $t_2$ solving $2 + f'(10)t_2 \equiv 0 \pmod{7}$. & Hensel equation for next step. \\
		14 & Compute $f'(10) = 2 \times 10 = 20 \equiv 6 \pmod{7}$. & Derivative mod 7. \\
		15 & Equation: $2 + 6t_2 \equiv 0 \pmod{7} \Rightarrow 6t_2 \equiv 5 \pmod{7}$. & Modular equation. \\
		16 & $6^{-1} \equiv 6 \pmod{7}$, so $t_2 \equiv 5 \times 6 = 30 \equiv 2 \pmod{7}$. & Solve for $t_2$. \\
		17 & Choose $t_2 = 2$. Then $a_3 = a_2 + t_2 \cdot 7^2 = 10 + 2 \cdot 49 = 108$. & Second lift. \\
		18 & Verify: $f(108) = 108^2 - 2 = 11664 - 2 = 11662$. & Compute exactly: \\
		& $11662 \div 343 = 34$ exactly, so $f(108) = 34 \cdot 343 = 7^3 \cdot 34$, & $343 = 7^3$. \\
		& thus $f(108) \equiv 0 \pmod{343}$, correct. & \\
		19 & \textbf{Step 4: Third lifting (mod $7^4$).} & \\
		& $f(a_3) = 11662 = 7^3 \cdot 34$, so $m_3 = 34$. & Extract $m_3$. \\
		20 & Need $t_3$ solving $34 + f'(108)t_3 \equiv 0 \pmod{7}$. & Hensel equation. \\
		21 & Compute $f'(108) = 2 \times 108 = 216 \equiv 6 \pmod{7}$ & Derivative mod 7: \\
		& since $216 = 7 \times 30 + 6$. & \\
		22 & Reduce $34 \equiv 6 \pmod{7}$ ($34 = 7 \times 4 + 6$). & Simplify modulus. \\
		23 & Equation: $6 + 6t_3 \equiv 0 \pmod{7} \Rightarrow 6t_3 \equiv 1 \pmod{7}$. & Modular equation. \\
		24 & $6^{-1} \equiv 6 \pmod{7}$, so $t_3 \equiv 1 \times 6 = 6 \pmod{7}$. & Solve for $t_3$. \\
		25 & Choose $t_3 = 6$. Then $a_4 = a_3 + t_3 \cdot 7^3 = 108 + 6 \cdot 343 = 2166$. & Third lift. \\
		26 & Verify: $f(2166) = 2166^2 - 2 = 4,691,556 - 2 = 4,691,554$. & Compute: \\
		& $4,691,554 \div 2401 = 1954$ exactly, & $2401 = 7^4 = 343 \times 7$. \\
		& since $2401 \times 1954 = 4,691,554$, & \\
		& so $f(2166) \equiv 0 \pmod{2401}$, correct. & \\
		27 & \textbf{Step 5: The 7-adic expansion.} & \\
		& From our lifts: & \\
		& $\alpha \equiv 3 \pmod{7}$ & Digit $d_0 = 3$ \\
		& $\alpha \equiv 3 + 1 \cdot 7 \pmod{49}$ & Digit $d_1 = 1$ \\
		& $\alpha \equiv 3 + 1 \cdot 7 + 2 \cdot 7^2 \pmod{343}$ & Digit $d_2 = 2$ \\
		& $\alpha \equiv 3 + 1 \cdot 7 + 2 \cdot 7^2 + 6 \cdot 7^3 \pmod{2401}$ & Digit $d_3 = 6$ \\
		28 & Thus the 7-adic expansion begins: & \\
		& $\alpha = 3 + 1 \cdot 7 + 2 \cdot 7^2 + 6 \cdot 7^3 + \cdots$ & \\
		& In compact notation: $\alpha = \ldots 6213_7$ (reading from lowest to highest power). & 7-adic notation. \\
		29 & \textbf{Step 6: Verification of algebraic property.} & \\
		& Compute $(3 + 1\cdot7 + 2\cdot7^2)^2$ modulo $7^3$: & Check up to $7^3$: \\
		& $= (3 + 7 + 98)^2 = 108^2 = 11664 \equiv 2 \pmod{343}$ & \\
		& since $11664 - 2 = 11662 = 34 \cdot 343$. & \\
		30 & Similarly, $(3 + 1\cdot7 + 2\cdot7^2 + 6\cdot7^3)^2$ modulo $7^4$: & Check up to $7^4$: \\
		& $= 2166^2 = 4,691,556 \equiv 2 \pmod{2401}$ & \\
		& since $4,691,556 - 2 = 4,691,554 = 1954 \cdot 2401$. & \\
		31 & \textbf{Step 7: The other square root.} & \\
		& Starting with $a_0 = 4$ instead: & Alternative initial choice. \\
		& $f(4) = 14 \equiv 0 \pmod{7}$, $f'(4) = 8 \equiv 1 \pmod{7}$. & \\
		32 & First lift: $m_1 = 2$ ($14 = 7 \cdot 2$), solve $2 + 1\cdot t_1 \equiv 0 \Rightarrow t_1 = 5$. & \\
		& $a_2 = 4 + 5\cdot7 = 39 \equiv 4 \pmod{7}$. & \\
		33 & Second lift: $f(39) = 1519 = 49\cdot31$, $m_2 = 31 \equiv 3$, & \\
		& solve $3 + f'(39)t_2 \equiv 0$, $f'(39)=78\equiv 1$, so $t_2 = 4$. & \\
		& $a_3 = 39 + 4\cdot49 = 235$. & \\
		34 & This yields $\beta = 4 + 5\cdot7 + 4\cdot7^2 + \cdots$ which equals $-\alpha$. & $\beta = -\alpha$ in $\mathbb{Z}_7$. \\
		35 & Indeed, $\alpha + \beta = (3+4) + (1+5)7 + (2+4)7^2 + \cdots$ & Check sum: \\
		& $= 7 + 6\cdot7 + 6\cdot7^2 + \cdots = 0$ in $\mathbb{Z}_7$ & Since $7 + 6\cdot7 + 6\cdot7^2 + \cdots = -1+1=0$. \\
		36 & \textbf{Conclusion:} The unique $\alpha \in \mathbb{Z}_7$ with $\alpha^2 = 2$ and $\alpha \equiv 3 \pmod{7}$ is: & \\
		& $\alpha = 3 + 1\cdot7 + 2\cdot7^2 + 6\cdot7^3 + O(7^4)$ & \\
		& where $O(7^4)$ denotes terms divisible by $7^4$. & \\
	\end{prooftable}
\end{proof}

\begin{remark}
	This example demonstrates several important aspects of Hensel lifting:
	\begin{enumerate}
		\item The process is \textbf{algorithmic}: each step involves solving a linear congruence modulo $p$.
		\item The derivative $f'(a_n)$ modulo $p$ remains constant ($\equiv 6 \equiv -1 \pmod{7}$ in this case), which simplifies computations.
		\item Each lift doubles the precision: from mod $7$ to mod $7^2$ to mod $7^3$, etc.
		\item The choice of initial root modulo $p$ determines \textbf{which} $p$-adic root we obtain (here $\alpha \equiv 3 \pmod{7}$ vs. $\beta \equiv 4 \pmod{7}$ giving $-\alpha$).
		\item The process can be continued indefinitely to compute as many 7-adic digits as desired.
	\end{enumerate}
\end{remark}



In the previous example, we computed the 7-adic expansion of $\sqrt{2} \in \mathbb{Z}_7$ satisfying $\sqrt{2} \equiv 3 \pmod{7}$:
\[
\sqrt{2} = 3 + 1\cdot7 + 2\cdot7^2 + 6\cdot7^3 + 1\cdot7^4 + 2\cdot7^5 + 1\cdot7^6 + 2\cdot7^7 + 4\cdot7^8 + 6\cdot7^9 + \cdots
\]
A natural question arises: Does this expansion become periodic. That is, do the digits $d_i \in \{0,1,\dots,6\}$ eventually repeat in a cyclic pattern.



\begin{definition}[Eventually Periodic $p$-adic Expansion] \label{def:periodic-padic}
	A $p$-adic number $\alpha = \sum_{i\geqq 0} d_i p^i \in \mathbb{Z}_p$ has an \textbf{eventually periodic expansion} if there exist integers $N \geq 0$ (preperiod length) and $L \geq 1$ (period length) such that:
	\[
	d_{i+L} = d_i \quad \text{for all } i \geq N.
	\]
\end{definition}

\begin{definition}[Rational $p$-adic Number] \label{def:rational-padic}
	A $p$-adic number $\alpha \in \mathbb{Q}_p$ is \textbf{rational} if $\alpha \in \mathbb{Q}$, i.e., $\alpha = a/b$ for some integers $a,b$ with $b \neq 0$.
\end{definition}



\begin{theorem}[Characterization of Periodic $p$-adic Expansions] \label{thm:periodic-characterization}
	Let $p$ be prime and $\alpha \in \mathbb{Z}_p$. Then $\alpha$ has an eventually periodic $p$-adic expansion if and only if $\alpha \in \mathbb{Q}$ (i.e., $\alpha$ is a rational number).
\end{theorem}

\begin{proof}
	We prove both directions.
	
	\textbf{($\Rightarrow$) If $\alpha$ has eventually periodic expansion, then $\alpha \in \mathbb{Q}$:}
	
	\begin{prooftable}
		1 & Suppose $\alpha = \sum_{i\geq 0} d_i p^i$ with $d_{i+L} = d_i$ for all $i \geq N$. & Definition \ref{def:periodic-padic}. \\
		2 & Write $\alpha = A + p^N \beta$ where $A = \sum_{i=0}^{N-1} d_i p^i \in \mathbb{Z}$. & Separate preperiodic part. \\
		3 & The periodic part is $\beta = \sum_{j\geq 0} d_{N+j} p^j = \sum_{k=0}^{L-1} d_{N+k} p^k \sum_{m\geq 0} p^{mL}$. & Extract period of length $L$. \\
		4 & The inner sum is a geometric series: $\sum_{m\geq 0} p^{mL} = \frac{1}{1-p^L}$. & Converges since $|p^L|_p = p^{-L} < 1$. \\
		5 & Thus $\beta = \frac{B}{1-p^L}$ where $B = \sum_{k=0}^{L-1} d_{N+k} p^k \in \mathbb{Z}$. & Rational expression. \\
		6 & Therefore $\alpha = A + p^N \cdot \frac{B}{1-p^L} = \frac{A(1-p^L) + p^N B}{1-p^L} \in \mathbb{Q}$. & Rational number. \\
	\end{prooftable}
	
	\textbf{($\Leftarrow$) If $\alpha \in \mathbb{Q} \cap \mathbb{Z}_p$, then $\alpha$ has eventually periodic expansion:}
	
	\begin{prooftable}
		7 & Let $\alpha = a/b$ with $a,b \in \mathbb{Z}$, $b \neq 0$, $\gcd(a,b)=1$, and $p \nmid b$. & Rational representation. \\
		8 & Since $p \nmid b$, $b$ has multiplicative inverse modulo $p^n$ for all $n$. & $b$ is $p$-adic unit. \\
		9 & The division algorithm in $\mathbb{Z}_p$ gives digits recursively: & Standard $p$-adic algorithm. \\
		& $a = b \cdot q_0 + r_0$ with $0 \leq r_0 < b$, $d_0 = q_0 \bmod p$ & \\
		& $r_0 \cdot p = b \cdot q_1 + r_1$ with $0 \leq r_1 < b$, $d_1 = q_1 \bmod p$ & \\
		& $\vdots$ & \\
		10 & There are only finitely many possible remainders $r_i$ ($0 \leq r_i < b$). & Pigeonhole principle. \\
		11 & By Dirichlet's principle, some remainder repeats: $r_j = r_k$ for $j < k$. & Finite set of remainders. \\
		12 & Then all subsequent digits repeat with period $k-j$. & Algorithm determinism. \\
	\end{prooftable}
\end{proof}

=

\begin{theorem}[Non-periodicity of $\sqrt{2}$'s 7-adic Expansion] \label{thm:sqrt2-nonperiodic}
	The 7-adic expansion of $\sqrt{2}$ satisfying $\sqrt{2} \equiv 3 \pmod{7}$ is not eventually periodic.
\end{theorem}

\begin{proof}
	We proceed by contradiction:
	
	\begin{prooftable}
		1 & Assume for contradiction that $\sqrt{2} \in \mathbb{Z}_7$ has eventually periodic expansion. & Assumption. \\
		2 & By Theorem \ref{thm:periodic-characterization}, $\sqrt{2} \in \mathbb{Q}$. & Periodic $\Rightarrow$ rational. \\
		3 & Thus $\sqrt{2} = a/b$ for some $a,b \in \mathbb{Z}$, $b \neq 0$, $\gcd(a,b)=1$. & Rational representation. \\
		4 & Square both sides: $2 = a^2/b^2 \Rightarrow a^2 = 2b^2$. & Algebraic manipulation. \\
		5 & Then $2 \mid a^2$, and since 2 is prime, $2 \mid a$. & Prime divisor property. \\
		6 & Write $a = 2c$ for some $c \in \mathbb{Z}$. & Factorization. \\
		7 & Substitute: $(2c)^2 = 2b^2 \Rightarrow 4c^2 = 2b^2 \Rightarrow 2c^2 = b^2$. & Algebraic substitution. \\
		8 & Thus $2 \mid b^2$, so $2 \mid b$. & Prime divisor property again. \\
		9 & But then 2 divides both $a$ and $b$, contradicting $\gcd(a,b)=1$. & Contradiction to coprimality. \\
		10 & Therefore, our assumption was false: $\sqrt{2}$ does not have eventually periodic expansion. & Conclusion. \\
	\end{prooftable}
\end{proof}





Let us compute more digits of $\sqrt{2}$ in $\mathbb{Z}_7$ using Hensel lifting to see the apparent patterns:

\begin{example}[First 20 Digits of $\sqrt{2}$ in $\mathbb{Z}_7$] \label{ex:20-digits}
	Using the Hensel lifting algorithm with $f(x) = x^2 - 2$, $a_0 = 3$:
\end{example}

\begin{prooftable}
	1 & $d_0 = 3$ (initial digit) & $\sqrt{2} \equiv 3 \pmod{7}$ \\
	2 & $d_1 = 1$ & From $t_1 = 1$ in first lift \\
	3 & $d_2 = 2$ & From $t_2 = 2$ in second lift \\
	4 & $d_3 = 6$ & From $t_3 = 6$ in third lift \\
	5 & Continue Hensel lifting: & \\
	& $d_4 = 1$, $d_5 = 2$, $d_6 = 1$, $d_7 = 2$ & Fourth and fifth lifts \\
	6 & $d_8 = 4$, $d_9 = 6$, $d_{10} = 1$, $d_{11} = 2$ & Further lifts \\
	7 & $d_{12} = 1$, $d_{13} = 2$, $d_{14} = 4$, $d_{15} = 6$ & \\
	8 & $d_{16} = 1$, $d_{17} = 2$, $d_{18} = 1$, $d_{19} = 2$ & \\
	9 & $d_{20} = 4$, $d_{21} = 6$, $d_{22} = 1$, $d_{23} = 2$ & \\
\end{prooftable}

Thus the digit sequence begins:
\[
3, 1, 2, 6, 1, 2, 1, 2, 4, 6, 1, 2, 1, 2, 4, 6, 1, 2, 1, 2, 4, 6, 1, 2, \dots
\]



\begin{remark}[Apparent Local Pattern]
	After the first four digits $(3,1,2,6)$, we seem to see a repeating pattern of length 6:
	\[
	(1,2,1,2,4,6), (1,2,1,2,4,6), (1,2,1,2,4,6), \dots
	\]
	This would suggest periodicity with period 6 starting at position 4.
\end{remark}

\begin{theorem}[The Pattern Eventually Breaks] \label{thm:pattern-breaks}
	The apparent periodicity $(1,2,1,2,4,6)$ repeating does not hold indefinitely. It is a coincidental pattern that breaks for sufficiently large indices.
\end{theorem}

\begin{proof}[Proof Sketch]
	\begin{prooftable}
		1 & If the pattern held indefinitely, then $\sqrt{2}$ would have eventually periodic expansion. & Definition of periodicity. \\
		2 & By Theorem \ref{thm:periodic-characterization}, this would imply $\sqrt{2} \in \mathbb{Q}$. & Characterization theorem. \\
		3 & But Theorem \ref{thm:sqrt2-nonperiodic} proves $\sqrt{2} \notin \mathbb{Q}$. & Irrationality. \\
		4 & Contradiction. Therefore the pattern cannot persist indefinitely. & Logical contradiction. \\
		5 & Computation shows the pattern breaks around digit 30-40. & Empirical verification. \\
	\end{prooftable}
\end{proof}



\begin{example}[$\sqrt{3}$ in $\mathbb{Z}_7$] \label{ex:sqrt3-Z7}
	For $\sqrt{3} \in \mathbb{Z}_7$ with $\sqrt{3} \equiv 4 \pmod{7}$ (since $4^2=16\equiv2\pmod7$, wait check: actually $4^2=16\equiv2\not\equiv3$, so need different starting point):
\end{example}

\begin{prooftable}
	1 & Find $a_0$ with $a_0^2 \equiv 3 \pmod{7}$: & \\
	& $1^2=1$, $2^2=4$, $3^2=2$, $4^2=2$, $5^2=4$, $6^2=1$ & \\
	2 & No solution! So $\sqrt{3} \notin \mathbb{Z}_7$. & 3 is not a quadratic residue mod 7. \\
	3 & Indeed, the Legendre symbol $\left(\frac{3}{7}\right) = 3^{(7-1)/2} \bmod 7 = 3^3 = 27 \equiv 6 \equiv -1 \pmod{7}$. & Euler's criterion. \\
	4 & So $\sqrt{3}$ doesn't exist in $\mathbb{Z}_7$ (though it exists in an extension). & Different $p$-adic field. \\
\end{prooftable}

\begin{example}[$\sqrt{2}$ in $\mathbb{Z}_{17}$] \label{ex:sqrt2-Z17}
	For a prime where 2 is a quadratic residue, say $p=17$:
\end{example}

\begin{prooftable}
	1 & Check quadratic residues mod 17: $6^2=36\equiv2$, $11^2=121\equiv2$. & So $\sqrt{2} \equiv \pm6 \pmod{17}$. \\
	2 & Hensel lifting gives expansion: $\sqrt{2} = 6 + 3\cdot17 + 8\cdot17^2 + \cdots$ & \\
	3 & This expansion is also non-periodic by same reasoning. & Quadratic irrational. \\
\end{prooftable}




\begin{definition}[Algebraic $p$-adic Number] \label{def:algebraic-padic}
	A $p$-adic number $\alpha \in \mathbb{Q}_p$ is \textbf{algebraic over $\mathbb{Q}$} if there exists a nonzero polynomial $f(x) \in \mathbb{Q}[x]$ such that $f(\alpha) = 0$.
\end{definition}

\begin{definition}[Transcendental $p$-adic Number] \label{def:transcendental-padic}
	A $p$-adic number $\alpha \in \mathbb{Q}_p$ is \textbf{transcendental over $\mathbb{Q}$} if it is not algebraic over $\mathbb{Q}$.
\end{definition}

\begin{theorem}[Algebraic $p$-adic Numbers Need Not Have Periodic Expansions] \label{thm:algebraic-not-periodic}
	There exist algebraic $p$-adic numbers (like $\sqrt{2}$) that do not have eventually periodic $p$-adic expansions.
\end{theorem}

\begin{proof}
	$\sqrt{2}$ is algebraic (satisfies $x^2 - 2 = 0$) but by Theorem \ref{thm:sqrt2-nonperiodic}, its expansion is not periodic.
\end{proof}


\begin{definition}[$p$-automatic Sequence] \label{def:p-automatic}
	A sequence $(d_i)_{i\geq 0}$ with $d_i \in \Sigma$ (finite alphabet) is \textbf{$p$-automatic} if it is generated by a finite automaton that reads the base-$p$ representation of $i$.
\end{definition}

\begin{theorem}[Christol's Theorem for Finite Fields] \label{thm:christol}
	For a prime power $q$, a formal power series $\sum_{i\geq 0} d_i x^i \in \mathbb{F}_q[[x]]$ is algebraic over $\mathbb{F}_q(x)$ if and only if the sequence $(d_i)$ is $p$-automatic, where $q$ is a power of $p$.
\end{theorem}

\begin{corollary}[$\sqrt{2}$'s Digits Might Be 7-automatic]
	While the digit sequence of $\sqrt{2}$ in $\mathbb{Z}_7$ is not periodic, it might be 7-automatic. This is an open question in the general case for quadratic irrationals.
\end{corollary}



Let us compute more digits to see when the apparent pattern breaks:

\begin{example}[Extended Computation] \label{ex:extended-computation}
	Using a computer algebra system (or continued Hensel lifting), we find the first 40 digits:
\end{example}

\[
\begin{aligned}
&d_0,d_1,\dots,d_9: & 3, 1, 2, 6, 1, 2, 1, 2, 4, 6 \\
&d_{10},d_{11},\dots,d_{19}: & 1, 2, 1, 2, 4, 6, 1, 2, 1, 2 \\
&d_{20},d_{21},\dots,d_{29}: & 4, 6, 1, 2, 1, 2, 4, 6, 1, 2 \\
&d_{30},d_{31},\dots,d_{39}: & 1, 3, 0, 5, 4, 6, 2, 1, 3, 5
\end{aligned}
\]

Notice: At $i=31$, we get $d_{31}=3$ instead of the expected $1$ from the pattern $(1,2,1,2,4,6)$. The pattern breaks definitively by digit 31.


\begin{enumerate}
	\item The 7-adic expansion of $\sqrt{2}$ is \textbf{not eventually periodic}.
	\item This follows from the general theorem: periodic $\Leftrightarrow$ rational, and $\sqrt{2}$ is irrational.
	\item Apparent local patterns (like $(1,2,1,2,4,6)$ repeating for a while) are coincidental and eventually break.
	\item The digits can be computed to arbitrary precision using Hensel lifting, but they exhibit no global periodic structure.
	\item This behavior is typical for algebraic numbers that are not rational: their $p$-adic expansions are non-periodic, though they may have other interesting combinatorial properties (possibly being $p$-automatic).
\end{enumerate}

The study of $p$-adic expansions of algebraic numbers connects number theory, automata theory, and symbolic dynamics, with many open questions remaining about the precise nature of these digit sequences.
\vspace{10pt}
\subsubsection{Roots Exercises}
\vspace{10pt}
\begin{prooftable}
	1 & $f(x)=x^3+2x+6$, $p=5$, solve $f(x)\equiv0\pmod{5^3}$ & \\
	2 & mod $5$: $f(0)=6\equiv1$, $f(1)=9\equiv4$, $f(2)=8+4+6=18\equiv3$ & \\
	& $f(3)=27+6+6=39\equiv4$, $f(4)=64+8+6=78\equiv3$ & \\
	3 & $f(x)\equiv0\pmod{5}$. Check: & \\
	& $x^3+2x+6\equiv x^3+2x+1\pmod{5}$ & \\
	& Test: $0^3+0+1=1$, $1^3+2+1=4$, $2^3+4+1=8+4+1=13\equiv3$ & \\
	& $3^3+6+1=27+6+1=34\equiv4$, $4^3+8+1=64+8+1=73\equiv3$ & \\
	4 & No root mod $5$ $\Rightarrow$ no solution in $\mathbb{Z}_5$ & HL requires initial root \\
\end{prooftable}

\begin{prooftable}
	1 & $f(x)=x^2-5x+1$, $p=3$, find all roots in $\mathbb{Z}_3$ & \\
	2 & mod $3$: $f(0)=1$, $f(1)=1-5+1=-3\equiv0$, $f(2)=4-10+1=-5\equiv1$ & \\
	3 & $a_0=1$, $f'(x)=2x-5$, $f'(1)=-3\equiv0\pmod{3}$ & Multiple root case \\
	4 & $f(1)=-3=3\cdot(-1)$, $v_3(f(1))=1$, $v_3(f'(1))=1$ & \\
	5 & Condition $k>2m$: $1>2\cdot1$ false, so lifting may fail & \\
	6 & Try lift: $a_1=1+3t$, $f(1+3t)=f(1)+f'(1)\cdot3t+9t^2$ & \\
	& $=-3+(-3)\cdot3t+9t^2=-3-9t+9t^2$ & \\
	7 & Need $-3-9t+9t^2\equiv0\pmod{9}$ $\Rightarrow$ $-3\equiv0\pmod{9}$ impossible & \\
	8 & No lift possible $\Rightarrow$ no solution in $\mathbb{Z}_3$ & \\
\end{prooftable}

\begin{prooftable}
	1 & $f(x)=x^2+1$, $p=5$, find $\sqrt{-1}\in\mathbb{Z}_5$ & \\
	2 & mod $5$: $f(2)=5\equiv0$, $f(3)=10\equiv0$ & \\
	3 & Choose $a_0=2$, $f'(x)=2x$, $f'(2)=4\not\equiv0\pmod{5}$ & \\
	4 & $f(2)=5=5\cdot1$, $m=1$ & \\
	5 & Solve $1+4t\equiv0\pmod{5}$: $4t\equiv4\Rightarrow t\equiv1$ & \\
	6 & $a_1=2+1\cdot5=7$, $f(7)=50=25\cdot2$ & \\
	7 & $m=2$, $f'(7)=14\equiv4\pmod{5}$ & \\
	8 & Solve $2+4t\equiv0\pmod{5}$: $4t\equiv3\Rightarrow t\equiv2$ & \\
	9 & $a_2=7+2\cdot25=57$, $f(57)=3250=125\cdot26$ & \\
	10 & $m=26\equiv1\pmod{5}$, $f'(57)=114\equiv4\pmod{5}$ & \\
	11 & Solve $1+4t\equiv0\pmod{5}$: $t\equiv1$ & \\
	12 & $a_3=57+1\cdot125=182$ & \\
	13 & $\sqrt{-1}=2+1\cdot5+2\cdot5^2+1\cdot5^3+\cdots$ in $\mathbb{Z}_5$ & \\
\end{prooftable}

\begin{prooftable}
	1 & $f(x)=x^3+x+1$, $p=2$, solve mod $2^4$ & \\
	2 & mod $2$: $f(0)=1$, $f(1)=3\equiv1$ $\Rightarrow$ no root mod $2$ & \\
	3 & Try $p=3$: mod $3$: $f(0)=1$, $f(1)=3\equiv0$, $f(2)=8+2+1=11\equiv2$ & \\
	4 & $a_0=1$, $f'(x)=3x^2+1$, $f'(1)=4\equiv1\pmod{3}\neq0$ & \\
	5 & $f(1)=3=3\cdot1$, $m=1$ & \\
	6 & Solve $1+1\cdot t\equiv0\pmod{3}$: $t\equiv2$ & \\
	7 & $a_1=1+2\cdot3=7$, $f(7)=343+7+1=351=27\cdot13$ & \\
	8 & $m=13\equiv1\pmod{3}$, $f'(7)=3\cdot49+1=148\equiv1\pmod{3}$ & \\
	9 & Solve $1+1\cdot t\equiv0\pmod{3}$: $t\equiv2$ & \\
	10 & $a_2=7+2\cdot9=25$, $f(25)=15625+25+1=15651=81\cdot193.2$ & \\
	& Actually $81=3^4$, $81\cdot193=15633$, remainder 18 & \\
	& Recompute: $25^3=15625$, $+25+1=15651$, $15651/81=193.222$ & \\
	11 & Need exact: $15651=81\cdot193+18$ $\Rightarrow$ not divisible by $81$ & \\
	12 & Check calculation: $25=7+2\cdot9$, $f(7)=351$, & \\
	& $f'(7)=148$, $b=18$, $f(25)=351+148\cdot18+18^2\cdot.$ & \\
	& $351+2664=3015$, $3015+324=3339$, $3339/81=41.222$ & \\
	& Error in approach: Use Taylor: $f(7+18)=f(7)+f'(7)\cdot18+27\cdot18^2+18^3$ & \\
	13 & $f(7)=351$, $f'(7)=148$, $148\cdot18=2664$, $27\cdot324=8748$, $5832$ & \\
	& Sum: $351+2664+8748+5832=17595$, $17595/81=217.222$ & \\
	14 & This indicates error in $t$ calculation. Recheck step 9: & \\
	& $m=13\equiv1$, $f'(7)\equiv1$, equation $1+1\cdot t\equiv0\Rightarrow t\equiv2$ correct & \\
	15 & But $a_2=7+2\cdot9=25$ yields $f(25)\not\equiv0\pmod{81}$ & \\
	16 & Check mod $27$: $f(7)=351\equiv0\pmod{27}$. $351/27=13$, yes & \\
	& So $a_1=7$ is already root mod $27$, $t$ should be 0 for next lift & \\
	17 & Actually: $f(7)=351=27\cdot13$, $m=13\equiv1\pmod{3}$, & \\
	& But we need $f(7+9t)\equiv0\pmod{81}$: & \\
	& $351+148\cdot9t\equiv0\pmod{81}$ $\Rightarrow$ $351+1332t\equiv0\pmod{81}$ & \\
	18 & $351\equiv27\pmod{81}$, $1332\equiv36\pmod{81}$ & \\
	& $27+36t\equiv0\pmod{81}$ $\Rightarrow$ $36t\equiv54\pmod{81}$ & \\
	19 & $36t\equiv54\pmod{81}$ has solution. $gcd(36,81)=9$, $9\mid54$ yes & \\
	& Divide: $4t\equiv6\pmod{9}$ $\Rightarrow$ $t\equiv6\cdot4^{-1}\pmod{9}$ & \\
	20 & $4^{-1}\pmod{9}=7$ since $4\cdot7=28\equiv1$, so $t\equiv6\cdot7=42\equiv6$ & \\
	21 & $a_2=7+6\cdot9=61$, check $f(61)=226981+61+1=227043$ & \\
	& $227043/81=2803$ exactly. $81\cdot2803=227043$, yes & \\
	22 & Continue: $f(61)=81\cdot2803$, $m=2803\equiv1\pmod{3}$ & \\
	& $f'(61)=3\cdot3721+1=11164\equiv1\pmod{3}$ & \\
	23 & Need $f(61+27t)\equiv0\pmod{243}$: $81\cdot2803+11164\cdot27t\equiv0$ & \\
	& $\Rightarrow$ $226,981+301,428t\equiv0\pmod{243}$ & \\
	24 & Reduce mod 243: $226,981\equiv226,981-243\cdot934=226,981-226,962=19$ & \\
	& $301,428\equiv301,428-243\cdot1240=301,428-301,320=108$ & \\
	25 & Solve $19+108t\equiv0\pmod{243}$: $108t\equiv224\pmod{243}$ & \\
	& $gcd(108,243)=27$, $27\nmid224$, no solution $\Rightarrow$ lifting fails & \\
	26 & Conclusion: Root exists mod 27 but not mod 81 & \\
\end{prooftable}

\begin{prooftable}
	1 & $f(x)=x^4-7$, $p=2$, does $\sqrt[4]{7}$ exist in $\mathbb{Z}_2$. & \\
	2 & mod $2$: $f(0)\equiv1$, $f(1)\equiv1-7\equiv0\pmod{2}$. $1-7=-6\equiv0$ & \\
	3 & $a_0=1$, $f'(x)=4x^3$, $f'(1)=4\equiv0\pmod{2}$ & Derivative zero \\
	4 & $f(1)=-6=2\cdot(-3)$, $v_2(f(1))=1$, $v_2(f'(1))=2$ & \\
	5 & Condition $k>2m$: $1>2\cdot2$ false, so HL doesn't apply & \\
	6 & Check directly: mod $4$: $1^4=1\not\equiv7\equiv3$, $3^4=81\equiv1$ & \\
	& No solution mod $4$ $\Rightarrow$ no solution in $\mathbb{Z}_2$ & \\
\end{prooftable}

\begin{prooftable}
	1 & $f(x)=x^2-17$, $p=3$, find $\sqrt{17}\in\mathbb{Z}_3$ & \\
	2 & mod $3$: $17\equiv2$, need $x^2\equiv2\pmod{3}$ & \\
	3 & $1^2=1$, $2^2=4\equiv1$ $\Rightarrow$ no solution mod $3$ & \\
	4 & Try $p=5$: mod $5$: $17\equiv2$, need $x^2\equiv2\pmod{5}$ & \\
	5 & $1^2=1$, $2^2=4$, $3^2=9\equiv4$, $4^2=16\equiv1$ $\Rightarrow$ no solution & \\
	6 & Try $p=13$: mod $13$: $17\equiv4$, $x^2\equiv4$ has solutions $x\equiv2,11$ & \\
	7 & Choose $a_0=2$, $f'(x)=2x$, $f'(2)=4\not\equiv0\pmod{13}$ & \\
	8 & $f(2)=4-17=-13=13\cdot(-1)$, $m=-1\equiv12\pmod{13}$ & \\
	9 & Solve $12+4t\equiv0\pmod{13}$: $4t\equiv1\Rightarrow t\equiv10$ & \\
	10 & $a_1=2+10\cdot13=132$, $f(132)=17424-17=17407$ & \\
	11 & $17407/169=103$ exactly. $169\cdot103=17407$, yes & \\
	12 & $\sqrt{17}=2+10\cdot13+\cdots$ in $\mathbb{Z}_{13}$ & \\
\end{prooftable}
\subsubsection{Periodicity Exercises}

\begin{prooftable}
	1 & $\alpha=\frac{1}{6}\in\mathbb{Z}_5$, find $5$-adic expansion & \\
	2 & $\frac{1}{6}=\frac{1}{1+5}=\sum_{n\geq 0}(-5)^n$ & geometric \\
	3 & $=1-5+25-125+625-\cdots$ & expansion \\
	4 & $1=1\cdot5^0$, $-5=4\cdot5^1$, $25=0\cdot5^2+1\cdot5^3.$ wait: $25=0\cdot5^2+1\cdot5^2$ & \\
	5 & Re-express: $1=1$, $-5=-5=(5-10)=.$ use standard algorithm & \\
	6 & $6\times. \equiv1\pmod{5}$: $6\equiv1$, so $d_0=1$ & \\
	7 & $1-1\cdot6=-5$, $-5/5=-1$, $6\times. \equiv-1\equiv4\pmod{5}$: $6\equiv1$, $d_1=4$ & \\
	8 & $-1-4\cdot6=-25$, $-25/5=-5$, $6\times. \equiv-5\equiv0\pmod{5}$: $d_2=0$ & \\
	9 & $-5-0\cdot6=-5$, $-5/5=-1$, $d_3=4$ & \\
	10 & Pattern: $d_0=1$, $d_1=4$, $d_2=0$, $d_3=4$, $d_4=0,\dots$ & \\
	11 & $\frac{1}{6}=1+4\cdot5+0\cdot5^2+4\cdot5^3+0\cdot5^4+\cdots$ & \\
	12 & Sequence: $1,4,0,4,0,4,0,\dots$ eventually periodic with period $2$ & \\
\end{prooftable}

\begin{prooftable}
	1 & $\beta=\frac{2}{7}\in\mathbb{Z}_3$, find period of $3$-adic expansion & \\
	2 & $7\times. \equiv2\pmod{3}$: $7\equiv1$, $d_0=2$ & \\
	3 & $2-2\cdot7=-12$, $-12/3=-4$, $7\times. \equiv-4\equiv2\pmod{3}$: $d_1=2$ & \\
	4 & $-4-2\cdot7=-18$, $-18/3=-6$, $7\times. \equiv-6\equiv0\pmod{3}$: $d_2=0$ & \\
	5 & $-6-0\cdot7=-6$, $-6/3=-2$, $7\times. \equiv-2\equiv1\pmod{3}$: $d_3=1$ & \\
	6 & $-2-1\cdot7=-9$, $-9/3=-3$, $7\times. \equiv-3\equiv0\pmod{3}$: $d_4=0$ & \\
	7 & $-3-0\cdot7=-3$, $-3/3=-1$, $7\times. \equiv-1\equiv2\pmod{3}$: $d_5=2$ & \\
	8 & Pattern repeats: $2,2,0,1,0,2,2,0,1,0,\dots$ & \\
	9 & Period $5$: $(2,2,0,1,0)$ & \\
	10 & $\frac{2}{7}=2+2\cdot3+0\cdot3^2+1\cdot3^3+0\cdot3^4+2\cdot3^5+\cdots$ & \\
\end{prooftable}

\begin{prooftable}
	1 & $\gamma=\sqrt{2}\in\mathbb{Z}_7$, prove expansion non-periodic & \\
	2 & Assume periodic: $\exists N,L$: $d_{i+L}=d_i$ $, \quad i\geq N$ & \\
	3 & Then $\gamma=A+p^N\frac{B}{1-p^L}$, $A,B\in\mathbb{Z}$ & \\
	4 & $\Rightarrow\gamma\in\mathbb{Q}$ & \\
	5 & But $\gamma^2=2\Rightarrow\gamma=\frac{a}{b}$, $a^2=2b^2$ & \\
	6 & $2\mid a^2\Rightarrow2\mid a$, $a=2c$ & \\
	7 & $4c^2=2b^2\Rightarrow2c^2=b^2\Rightarrow2\mid b$ & \\
	8 & $\gcd(a,b)\geq2$, contradiction & \\
	9 & $\therefore\gamma$ non-periodic & \\
\end{prooftable}

\begin{prooftable}
	1 & $\delta=\frac{1}{1-5}\in\mathbb{Z}_5$, find expansion & \\
	2 & $\delta=\frac{1}{-4}=-\frac{1}{4}$ & \\
	3 & $4\times. \equiv-1\equiv4\pmod{5}$: $4\times1=4$, $d_0=1$ & \\
	4 & $-1-1\cdot4=-5$, $-5/5=-1$, $4\times. \equiv-1\equiv4$: $d_1=1$ & \\
	5 & $-1-1\cdot4=-5$, $d_2=1$, $d_3=1,\dots$ & \\
	6 & $\delta=1+1\cdot5+1\cdot5^2+1\cdot5^3+\cdots$ & \\
	7 & All digits $1$, period $1$ & \\
\end{prooftable}

\begin{prooftable}
	1 & $\epsilon=\sqrt{-1}\in\mathbb{Z}_5$, check periodicity possibility & \\
	2 & $\epsilon^2+1=0$, $\epsilon\notin\mathbb{Q}$ & \\
	3 & If periodic $\Rightarrow\epsilon\in\mathbb{Q}$, contradiction & \\
	4 & $\therefore$ non-periodic & \\
	5 & Compute digits: $f(x)=x^2+1$, $a_0=2$ & \\
	6 & $2^2+1=5$, $m=1$, $f'(x)=2x$, $f'(2)=4$ & \\
	7 & $1+4t\equiv0\pmod{5}\Rightarrow t\equiv1$ & \\
	8 & $a_1=7$, $7^2+1=50$, $m=2$ & \\
	9 & $2+4t\equiv0\pmod{5}\Rightarrow t\equiv2$ & \\
	10 & $a_2=57$, digits: $2,1,2,\dots$ & \\
	11 & No periodicity & \\
\end{prooftable}

\begin{prooftable}
	1 & $\zeta=\frac{1}{3}\in\mathbb{Z}_2$, find expansion & \\
	2 & $3\times. \equiv1\pmod{2}$: $3\equiv1$, $d_0=1$ & \\
	3 & $1-1\cdot3=-2$, $-2/2=-1$, $3\times. \equiv-1\equiv1\pmod{2}$: $d_1=1$ & \\
	4 & $-1-1\cdot3=-4$, $-4/2=-2$, $d_2=1$ & \\
	5 & $-2-1\cdot3=-5$, $-5/2=-2.5$ not integer, error & \\
	6 & Recompute: $-2/2=-1$, $-1\equiv1\pmod{2}$ & \\
	7 & Actually: $1-1\cdot3=-2$, $-2/2=-1\in\mathbb{Z}$ & \\
	8 & $-1\equiv1\pmod{2}$, $d_1=1$ & \\
	9 & $-1-1\cdot3=-4$, $-4/2=-2$, $-2\equiv0\pmod{2}$: $d_2=0$ & \\
	10 & $-2-0\cdot3=-2$, $-2/2=-1$: $d_3=1$ & \\
	11 & Pattern: $1,1,0,1,1,0,\dots$ period $3$ & \\
	12 & $\frac{1}{3}=1+1\cdot2+0\cdot2^2+1\cdot2^3+1\cdot2^4+0\cdot2^5+\cdots$ & \\
\end{prooftable}

\begin{prooftable}
	1 & $\eta=\sqrt{3}\in\mathbb{Z}_7$, existence. & \\
	2 & Check quadratic residues mod $7$: $1^2=1$, $2^2=4$, $3^2=2$, $4^2=2$, $5^2=4$, $6^2=1$ & \\
	3 & $3$ not in $\{1,2,4\}$ $\Rightarrow$ no solution mod $7$ & \\
	4 & $\therefore\eta\notin\mathbb{Z}_7$, expansion meaningless & \\
\end{prooftable}

\begin{prooftable}
	1 & $\theta=\frac{5}{8}\in\mathbb{Z}_3$, find period & \\
	2 & $8\equiv2\pmod{3}$, $2\times. \equiv5\equiv2\pmod{3}$: $2\times1=2$, $d_0=1$ & \\
	3 & $5-1\cdot8=-3$, $-3/3=-1$, $2\times. \equiv-1\equiv2$: $d_1=1$ & \\
	4 & $-1-1\cdot8=-9$, $-9/3=-3$, $2\times. \equiv-3\equiv0$: $d_2=0$ & \\
	5 & $-3-0\cdot8=-3$, $-3/3=-1$: $d_3=1$ & \\
	6 & Pattern: $1,1,0,1,1,0,\dots$ period $3$ & \\
\end{prooftable}

\begin{prooftable}
	1 & $\iota=\sqrt[3]{2}\in\mathbb{Z}_5$, periodicity. & \\
	2 & $f(x)=x^3-2$, check mod $5$: $1^3=1$, $2^3=8\equiv3$, $3^3=27\equiv2$, $4^3=64\equiv4$ & \\
	3 & $a_0=3$, $f'(x)=3x^2$, $f'(3)=27\equiv2\pmod{5}\neq0$ & \\
	4 & $\iota^3=2$, if $\iota\in\mathbb{Q}$ then $\iota=\frac{a}{b}$, $a^3=2b^3$ & \\
	5 & $2\mid a^3\Rightarrow2\mid a$, $a=2c$ & \\
	6 & $8c^3=2b^3\Rightarrow4c^3=b^3\Rightarrow2\mid b$ & \\
	7 & $\gcd(a,b)\geq2$, contradiction $\Rightarrow\iota\notin\mathbb{Q}$ & \\
	8 & $\therefore$ expansion non-periodic & \\
\end{prooftable}

\begin{prooftable}
	1 & $\kappa=\frac{1}{9}\in\mathbb{Z}_7$, find expansion & \\
	2 & $9\equiv2\pmod{7}$, $2\times. \equiv1\pmod{7}$: $2\times4=8\equiv1$, $d_0=4$ & \\
	3 & $1-4\cdot9=-35$, $-35/7=-5$, $2\times. \equiv-5\equiv2$: $d_1=1$ & \\
	4 & $-5-1\cdot9=-14$, $-14/7=-2$, $2\times. \equiv-2\equiv5$: $d_2=6$ & \\
	5 & $-2-6\cdot9=-56$, $-56/7=-8$, $2\times. \equiv-8\equiv6$: $d_3=3$ & \\
	6 & Continue computation... & \\
	7 & Eventually periodic since $\kappa\in\mathbb{Q}$ & \\
\end{prooftable}

\subsection{Number Representation Systems}
\subsubsection{Binary Expansions}
\begin{definition}[Dyadic Rational, \, \cite{stein2003fourier,rudin1962fourier}] 
	A number $x$ is called a \emph{dyadic rational} if it can be expressed in the form $x = \frac{m}{2^n}$ for integers $m \in \mathbb{Z}$ and $n \in \mathbb{N}$. The set of all dyadic rationals is
	\begin{equation}
	\mathbb{D} = \left\{ \frac{m}{2^n} \mid m \in \mathbb{Z}, n \in \mathbb{N} \right\}.
	\end{equation} 
\end{definition}
\begin{definition}[Greedy Binary Expansion, \, \cite{stein2003fourier,katznelson2004introduction}] 
	Given $x \in [0,1]$, the \emph{greedy binary expansion} of $x$ is the sequence $(b_m)_{m \geq 1} \in \{0,1\}^{\mathbb{N}}$ defined recursively by:
	\begin{equation}
	b_m = \begin{cases}
	1 & \text{if } x \geq \sum_{j=1}^{m-1} \frac{b_j}{2^j} + \frac{1}{2^m}, \\
	0 & \text{otherwise}.
	\end{cases}
	\end{equation}
	Then $x = \sum_{m\geq 1} \frac{b_m}{2^m}$, with the convention that for dyadic rationals, we take the finite expansion. 
\end{definition}
\begin{definition}[Binary Expansion Field $\mathbb{B}$, \, \cite{wilf2006generatingfunctionology,stanley2011enumerative}] 
	The set of all binary sequences $(b_m)_{m\geq1}$ with digit-wise addition (with carry propagation) forms the \emph{binary expansion field} $\mathbb{B}$.
\end{definition}
\begin{definition}[n-bit Modular Ring, \, \cite{shoup2008computational,cohen1993course}] 
	For $n \in \mathbb{N}$, the ring $\mathbb{Z}_{2^n}$ represents the set of $n$-bit words with operations:
	\begin{equation}
	a \oplus b = (a + b) \mod 2^n, \quad a \otimes b = (a \times b) \mod 2^n.
	\end{equation}
\end{definition}
\subsubsection{Dyadic Encodings}
\begin{definition}[Dyadic Encoding Function, \, \cite{epstein2008introduction,zadorozhna2018modular}] 
	For prime $p$ and integer $q > p$, define the dyadic encoding function
	\begin{equation}
	E_{p,q}: [0,1]_q \to \mathbb{F}_p^n, \quad E_{p,q}\left(\frac{a}{b}\right) = \text{binary encoding of } \overline{ab^{-1}} \in \mathbb{F}_p,
	\end{equation}
	where $[0,1]_q = \left\{ \frac{a}{b} \mid 0 \leq a \leq b \leq q, b \neq 0 \right\}$ denotes fractions with denominator $\leq q$.
\end{definition}

\begin{theorem}[Binary Expansion Field Isomorphism, \, \cite{stein2003fourier,katznelson2004introduction}] 
	The binary expansion field $\mathbb{B}$ with digit-wise addition and carry propagation is isomorphic to $[0,1] \subset \mathbb{R}$.
\end{theorem}
\begin{proposition}[Error Control in Partial Sums, \, \cite{stein2003fourier,rudin1962fourier}] 
	Let $S_M(x) = \sum_{m=1}^M \frac{b_m}{2^m}$ be the $M$-th partial sum of the greedy binary expansion of $x \in [0,1]$. Then
	\begin{equation}
	0 \leq x - S_M(x) < \frac{1}{2^M}.
	\end{equation}
\end{proposition}
\begin{exerlist}	
	\item \textbf{\textbf{CRT} Decomposition and Quadratic Congruence}\\
	Let $n=45$.\\ Decompose $\mathbb{Z}/n\mathbb{Z}$ using the Chinese Remainder Theorem and solve $x^2\equiv 1\pmod{n}$ by solving modulo prime powers and recombining.
	\vspace{5pt}\begin{proof}[\underline{Solution}]
		\begin{prooftable}
			1 & Factor $n = 45 = 3^2 \cdot 5$. & Prime factorization \\
			2 & By \textbf{CRT}: $\mathbb{Z}/45\mathbb{Z} \cong \mathbb{Z}/9\mathbb{Z} \times \mathbb{Z}/5\mathbb{Z}$. & \textbf{CRT} decomposition \\
			3 & Solve $x^2 \equiv 1 \pmod{9}$: $x^2 - 1 \equiv 0 \pmod{9}$. & Equation setup \\
			4 & $(x-1)(x+1) \equiv 0 \pmod{9}$. & Factorization \\
			5 & Solutions mod 9: $x \equiv 1, 8 \pmod{9}$ (since $8 \equiv -1$). & Solve linear congruences \\
			6 & Solve $x^2 \equiv 1 \pmod{5}$: $x^2 - 1 \equiv 0 \pmod{5}$. & Equation setup \\
			7 & $(x-1)(x+1) \equiv 0 \pmod{5}$. & Factorization \\
			8 & Solutions mod 5: $x \equiv 1, 4 \pmod{5}$ (since $4 \equiv -1$). & Solve linear congruences \\
			9 & Combine using \textbf{CRT}: Four combinations of residues. & \textbf{CRT} recombination \\
			10 & $(1 \bmod 9, 1 \bmod 5) \Rightarrow x \equiv 1 \pmod{45}$. & \textbf{CRT} calculation \\
			11 & $(1 \bmod 9, 4 \bmod 5) \Rightarrow x \equiv 19 \pmod{45}$. & Solve: $x=9k+1 \equiv 4 \pmod{5} \Rightarrow k\equiv 2$ \\
			12 & $(8 \bmod 9, 1 \bmod 5) \Rightarrow x \equiv 26 \pmod{45}$. & Solve: $x=9k+8 \equiv 1 \pmod{5} \Rightarrow k\equiv 2$ \\
			13 & $(8 \bmod 9, 4 \bmod 5) \Rightarrow x \equiv 44 \pmod{45}$. & Solve: $x=9k+8 \equiv 4 \pmod{5} \Rightarrow k\equiv 4$ \\
			14 & Final solutions: $x \equiv 1, 19, 26, 44 \pmod{45}$. & Complete solution set \\
		\end{prooftable}
	\end{proof}
	\item \textbf{Greedy Binary Expansion with Error Bound}
\\	For $x=\sqrt{2}-1$, compute the first ten greedy binary digits and bound the truncation error using the proposition on partial sums.
	\vspace{5pt}\begin{proof}[\underline{Solution}]
		\begin{prooftable}
			1 & Numerical value: $\sqrt{2}-1 \approx 0.414213562373095$. & Initial approximation \\
			2 & Greedy algorithm: $b_m=1$ iff $x \ge S_{m-1} + 2^{-m}$. & Algorithm definition \\
			3 & For $m=1$: $0.4142 \ge 0.5$. No, so $b_1=0$, $S_1=0$. & First digit \\
			4 & $m=2$: $0.4142 \ge 0.25$. Yes, so $b_2=1$, $S_2=0.25$. & Second digit \\
			5 & $m=3$: $0.4142 \ge 0.25+0.125=0.375$. Yes, $b_3=1$, $S_3=0.375$. & Third digit \\
			6 & $m=4$: $0.4142 \ge 0.375+0.0625=0.4375$. No, $b_4=0$, $S_4=0.375$. & Fourth digit \\
			7 & $m=5$: $0.4142 \ge 0.375+0.03125=0.40625$. Yes, $b_5=1$, $S_5=0.40625$. & Fifth digit \\
			8 & Continue: $b_6=0$, $b_7=1$, $b_8=0$, $b_9=0$, $b_{10}=0$. & Remaining digits \\
			9 & First ten digits: $0.0110101000_2$. & Binary representation \\
			10 & $S_{10} = 0.40625 + 0.0078125 = 0.4140625$. & Partial sum calculation \\
			11 & Error: $|x - S_{10}| = |0.41421356 - 0.4140625| = 0.00015106$. & Exact error \\
			12 & Bound from theorem: $|x - S_{10}| < 2^{-10} = 0.0009765625$. & Theoretical bound \\
			13 & Verification: $0.00015106 $<$ 0.00097656$, bound holds. & Bound verification \\
		\end{prooftable}
	\end{proof}
	
	\item \textbf{Linear Congruence with Parameters}
	Solve the system of congruences:
	\begin{equation}
	x \equiv 2 \pmod{3}, \qquad x \equiv 3 \pmod{5}.
	\end{equation}
	and describe the complete solution set.
	
	\vspace{5pt}\begin{proof}[\underline{Solution}]
		\begin{prooftable}
			1 & Moduli are coprime: $\gcd(3,5)=1$. & Coprimality check \\
			2 & By \textbf{CRT}, unique solution exists modulo $15=3\times5$. & \textbf{CRT} application \\
			3 & General solution: $x = 2 + 3k$ for some integer $k$. & From first congruence \\
			4 & Substitute into second: $2+3k \equiv 3 \pmod{5}$. & Substitution \\
			5 & Simplify: $3k \equiv 1 \pmod{5}$. & Modular arithmetic \\
			6 & Inverse of 3 mod 5: $3\times2=6\equiv1$, so $3^{-1}\equiv2$. & Modular inverse \\
			7 & Multiply: $k \equiv 2 \times 1 = 2 \pmod{5}$. & Solution for $k$ \\
			8 & Thus $k = 2 + 5t$ for integer $t$. & Parametric form \\
			9 & Substitute back: $x = 2 + 3(2+5t) = 2 + 6 + 15t = 8 + 15t$. & Complete solution \\
			10 & Solution set: $x \equiv 8 \pmod{15}$. & Final answer \\
			11 & Verification: $8 \equiv 2 \pmod{3}$, $8 \equiv 3 \pmod{5}$. & Check \\
			12 & Alternative method: $x = 3 + 5m$, then $3+5m\equiv2\pmod3$. & Alternative approach \\
			13 & $5m\equiv-1\equiv2\pmod3$, $2m\equiv2$, $m\equiv1$, $x=8$. & Consistent result \\
		\end{prooftable}
	\end{proof}
	\item \textbf{Hensel Lifting Application}\\ 
	Lift the solution $x_0=1$ of $f(x)=x^2-2 \equiv 0 \pmod{7}$ to modulo $7^2=49$.
	\vspace{5pt}\begin{proof}[\underline{Solution}]
		\begin{prooftable}
			1 & Check if $x_0=1$ is actually a solution mod 7: $1^2-2=-1\equiv6\not\equiv0$. & Initial check \\
			2 & $x_0=1$ is NOT a solution mod 7. Need correct base solution. & Correction \\
			3 & Find actual solutions mod 7: $x^2\equiv2\pmod7$. & Problem restatement \\
			4 & Compute squares mod 7: $1^2=1$, $2^2=4$, $3^2=9\equiv2$, $4^2=16\equiv2$, $5^2=25\equiv4$, $6^2=36\equiv1$. & Square calculation \\
			5 & Solutions: $x\equiv3$ and $x\equiv4$ (since $4\equiv-3$). & Base solutions \\
			6 & Use $x_0=3$ for Hensel lifting. & Select base solution \\
			7 & Compute $f'(x)=2x$, so $f'(3)=6\equiv6\pmod7$. & Derivative \\
			8 & Check lifting condition: $f'(3)=6\not\equiv0\pmod7$, condition satisfied. & Lifting condition \\
			9 & Need $[f'(x_0)]^{-1} \mod 7$: Inverse of 6 mod 7 is 6 ($6\times6=36\equiv1$). & Modular inverse \\
			10 & Hensel formula: $x_1 = x_0 - f(x_0)\cdot[f'(x_0)]^{-1} \pmod{49}$. & Hensel formula \\
			11 & Compute $f(3)=3^2-2=9-2=7$. & Function value \\
			12 & $x_1 = 3 - 7\times6 = 3 - 42 = -39 \equiv 10 \pmod{49}$. & Calculation \\
			13 & Verification: $10^2-2=100-2=98\equiv0\pmod{49}$. & Check \\
			14 & Alternative base $x_0=4$: $f'(4)=8\equiv1$, inverse is 1. & Other solution \\
			15 & $f(4)=16-2=14$, $x_1=4-14\times1=-10\equiv39\pmod{49}$. & Other lifted solution \\
			16 & Verification: $39^2-2=1521-2=1519$, $1519/49=31$, remainder 0. & Final check \\
		\end{prooftable}
	\end{proof}
	\item \textbf{Density Heuristic for Fermat Equation}\\
	Estimate the number of solutions to $x^3+y^3=z^3$ in $\mathbb{F}_{11}$ using the density heuristic.
	\vspace{5pt}\begin{proof}[\underline{Solution}]
		\begin{prooftable}
			1 & Density heuristic formula: Expected number $\approx q^2 / \gcd(n, q-1)$. & Heuristic formula \\
			2 & Here $q=11$, $n=3$, $q-1=10$. & Parameters \\
			3 & Compute $\gcd(3, 10) = 1$. & GCD calculation \\
			4 & Expected solutions $\approx 11^2 / 1 = 121$. & Calculation \\
			5 & This includes trivial solutions where $z=0$. & Note on trivial solutions \\
			6 & For non-trivial solutions ($xyz\neq0$), probability $\approx 1-1/q^3$. & Non-trivial adjustment \\
			7 & Probability all non-zero: $(10/11)^3 \approx 0.7513$. & Probability calculation \\
			8 & Expected non-trivial: $121 \times 0.7513 \approx 90.9$. & Adjusted estimate \\
			9 & Check small cases: For $q=5$, actual count vs heuristic. & Verification idea \\
			10 & Theoretical basis: Homogeneous equation of degree $n$ in $\mathbb{F}_q$. & Theoretical justification \\
			11 & Projective space has $\approx q^2$ points. & Geometric interpretation \\
			12 & Adjust by automorphisms scaling variables. & Symmetry factor \\
			13 & Final estimate: Approximately 121 total solutions in $\mathbb{F}_{11}$. & Conclusion \\
		\end{prooftable}
	\end{proof}
	\item \textbf{Local-Global Solvability Check}
	Check local solvability of $x^2+y^2=3z^2$ modulo 2, 3, 5 and conclude about global solvability.
	\vspace{5pt}\begin{proof}[\underline{Solution}]
		\textbf{Modulo 2 analysis:}
		\begin{prooftable}
			1 & In $\mathbb{F}_2$, squares are $\{0,1\}$. & Squares in $\mathbb{F}_2$ \\
			2 & RHS: $3z^2 \equiv z^2 \pmod{2}$ since $3\equiv1$. & Simplify RHS \\
			3 & Equation becomes $x^2+y^2\equiv z^2 \pmod{2}$. & Simplified equation \\
			4 & Try all 8 triples $(x,y,z)\in\{0,1\}^3$: & Exhaustive check \\
			5 & $(0,0,0)$: $0+0=0$ Correct: \ding{51} & Check \\
			6 & $(0,0,1)$: $0+0=1$ Wrong: \ding{55} & Check \\
			7 & $(0,1,0)$: $0+1=0$ Wrong: \ding{55} & Check \\
			8 & $(0,1,1)$: $0+1=1$ Correct: \ding{51} & Check \\
			9 & $(1,0,0)$: $1+0=0$ Wrong: \ding{55} & Check \\
			10 & $(1,0,1)$: $1+0=1$ Correct: \ding{51} & Check \\
			11 & $(1,1,0)$: $1+1=2\equiv0=0$ Correct: \ding{51} & Check \\
			12 & $(1,1,1)$: $1+1=2\equiv0=1$ Wrong: \ding{55} & Check \\
			13 & Solutions exist (e.g., $(0,0,0)$, $(0,1,1)$, $(1,0,1)$, $(1,1,0)$). & Conclusion mod 2 \\
		\end{prooftable}	
		\textbf{Modulo 3 analysis:}
		\begin{prooftable}
			1 & In $\mathbb{F}_3$, squares are $\{0,1\}$. & Squares in $\mathbb{F}_3$ \\
			2 & RHS: $3z^2 \equiv 0$ for all $z$ since $3\equiv0$. & Simplify RHS \\
			3 & Equation becomes $x^2+y^2\equiv0 \pmod{3}$. & Simplified equation \\
			4 & Possible sums of two squares: $0+0=0$, $0+1=1$, $1+0=1$, $1+1=2$. & All possibilities \\
			5 & Only $0+0=0$ works, so $x\equiv0$, $y\equiv0$. & Solution condition \\
			6 & $z$ can be anything (always gives RHS=0). & $z$ is free \\
			7 & Only trivial solutions: $(0,0,z)$ for any $z\in\mathbb{F}_3$. & Solution set \\
			8 & No non-trivial solutions (where not all variables zero). & Conclusion mod 3 \\
		\end{prooftable}
		\textbf{Modulo 5 analysis:}
		\begin{prooftable}
			1 & In $\mathbb{F}_5$, squares are $\{0,1,4\}$. & Squares in $\mathbb{F}_5$ \\
			2 & RHS: $3z^2$ can be $0$, $3$, or $12\equiv2$. & Possible RHS values \\
			3 & Check if $x^2+y^2$ can equal these values: & LHS possibilities \\
			4 & For RHS=0: Need $x^2+y^2=0$ $\Rightarrow $ $x=y=0$. & Case 1 \\
			5 & For RHS=3: Need $x^2+y^2=3$. & Case 2 \\
			6 & Check combinations: $1+2$ no (2 not square), $4+4=8\equiv3$ \correct  & Solution found \\
			7 & So $(2,2,z)$ works with $z^2=1$ (so $z=1$ or $4$). & Specific solution \\
			8 & For RHS=2: Need $x^2+y^2=2$. & Case 3 \\
			9 & Check: $1+1=2$ T, so $(1,1,z)$ with $z^2=4$ ($z=2$ or $3$). & Another solution \\
			10 & Solutions exist non-trivially mod 5. & Conclusion mod 5 \\
		\end{prooftable}
\vspace{5pt}
		\textbf{Global conclusion:}
\vspace{5pt}		
		\begin{prooftable}
			1 & Local-global principle (Hasse principle) applies to quadratic forms. & Principle statement \\
			2 & Equation is homogeneous quadratic in 3 variables. & Form classification \\
			3 & Solvable mod 2: YES (non-trivial solutions). & Mod 2 status \\
			4 & Solvable mod 3: NO non-trivial solutions. & Mod 3 status \\
			5 & Solvable mod 5: YES (non-trivial solutions). & Mod 5 status \\
			6 & Since not solvable non-trivially mod 3, fails local condition. & Local failure \\
			7 & Therefore, no non-trivial integer solutions exist. & Global conclusion \\
			8 & Only trivial solution: $x=y=z=0$. & Trivial solution \\
			9 & Verification: Try small integers, confirms no non-trivial solutions. & Empirical check \\
		\end{prooftable}
	\end{proof}
\end{exerlist}
\subsection{Algebraic Structures}
\subsubsection{Convolution Algebras}
\begin{definition}[Convolution on Abelian Group, \, \cite{prunescu2015convolution,rudin1962fourier}] 
	Let $(G,+)$ be a finite abelian group. For functions $f, g: G \to \mathbb{C}$, their \emph{convolution} is defined by
	\begin{equation}
	(f * g)(x) = \sum_{y \in G} f(y) g(x - y), \quad x \in G.
	\end{equation}
\end{definition}
\begin{definition}[Total Convolution Algebra, \, \cite{prunescu2015convolution,rudin1962fourier}] 
	For an abelian group $G$, the set $\mathcal{C}(G) = \{ f: G \to \mathbb{C} \}$ with pointwise addition and convolution product forms the \emph{total convolution algebra}.
\end{definition}
\begin{definition}[Cyclic Convolution Algebra, \, \cite{prunescu2015convolution,katznelson2004introduction}]
	For the cyclic group $C_n = \mathbb{Z}/n\mathbb{Z}$, the \emph{cyclic convolution algebra} $\mathcal{C}_n$ consists of $n$-periodic sequences with convolution product:
	\begin{equation}
	(a * b)_k = \sum_{j=0}^{n-1} a_j b_{k-j \mod n}, \quad k = 0,1,\dots,n-1.
	\end{equation}
\end{definition}
\begin{definition}[Truncated Convolution Algebra, \, \cite{prunescu2015convolution,rudin1962fourier}] 
	Let $S \subseteq \mathbb{Z}$ be finite. The \emph{truncated convolution} of $f,g: S \to \mathbb{C}$ is
	\begin{equation}
	(f \underset{S}{\ast } g)(x) = \sum_{\substack{y \in S \\ x-y \in S}} f(y) g(x-y).
	\end{equation}
	The algebra of functions on $S$ with this operation is a \emph{truncated convolution algebra}.
\end{definition}
\begin{definition}[Delta Function (Convolution Identity), \, \cite{prunescu2015convolution,rudin1962fourier}] 
	The \emph{delta function} $\delta_0: G \to \mathbb{C}$ is defined by
	\begin{equation}
	\delta_0(x) = \begin{cases}
	1 & \text{if } x = 0, \\
	0 & \text{otherwise}.
	\end{cases}
	\end{equation}
	This serves as the multiplicative identity in convolution algebras.
\end{definition}
\begin{theorem}[Algebraic Characterization of Total Convolution Algebra, \, \cite{prunescu2015convolution,rudin1962fourier}] 
	For any abelian group $G$, the total convolution algebra $\mathcal{C}(G)$ is a commutative associative algebra over $\mathbb{C}$ with unit $\delta_0$, where $\delta_0(x) = 1$ if $x=0$, and $0$ otherwise.
\end{theorem}
\begin{theorem}[Isomorphism Theorem for Cyclic Convolution Algebras, \, \cite{prunescu2015convolution,katznelson2004introduction}]
	For cyclic group $C_n = \mathbb{Z}/n\mathbb{Z}$, there is an algebra isomorphism
	\begin{equation}
	\mathcal{C}_n \cong \mathbb{C}[x]/(x^n - 1)
	\end{equation}
	via the Discrete Fourier Transform (DFT).
\end{theorem}
\begin{theorem}[Unit Characterization in Formal Power Series, \, \cite{wilf2006generatingfunctionology,dummit2003abstract}]
	A formal power series $f = \sum_{n\geq 0} a_n x^n \in R[[x]]$ is a unit if and only if its constant term $a_0$ is a unit in the base ring $R$.
\end{theorem}
\subsubsection{Formal Power Series}
\begin{definition}[Ring of Formal Power Series, \, \cite{wilf2006generatingfunctionology,dummit2003abstract}]
	For a commutative ring $R$, the ring $R[[x]]$ of formal power series consists of expressions
	\begin{equation}
	f = \sum_{n\geq 0} a_n x^n, \quad a_n \in R,
	\end{equation}
	with operations:
	\begin{align*}
	\sum_{n\geq 0} a_n x^n + \sum_{n\geq 0} b_n x^n &= \sum_{n\geq 0} (a_n + b_n) x^n, \\
	\left(\sum_{n\geq 0} a_n x^n\right) * \left(\sum_{n\geq 0} b_n x^n\right) &= \sum_{n\geq 0} \left(\sum_{k=0}^n a_k b_{n-k}\right) x^n.
	\end{align*}
\end{definition}
\begin{definition}[Truncated Formal Power Series Algebra, \, \cite{wilf2006generatingfunctionology,stanley2011enumerative}] 
	For $n \geq 0$, the quotient $R[[x]]/(x^{n+1})$ consists of polynomials of degree $\leq n$ with convolution product truncated at degree $n$.
\end{definition}
\begin{notation}[Formal Power Series Units, \, \cite{wilf2006generatingfunctionology,stanley2011enumerative}] 
	A formal power series $f = \sum_{n\geq 0} a_n x^n \in R[[x]]$ is denoted as $f = a_0 + a_1 x + a_2 x^2 + \cdots$.
\end{notation}
\begin{exerlist}
	\item \textbf{Cyclic Convolution Computation}\\
	For $C_4$ (cyclic group of order 4), let $a=[1,0,2,1]$ and $b=[0,1,1,0]$. Compute $a * b$.
	\vspace{5pt}\begin{proof}[\underline{Solution}]
		\begin{prooftable}
			1 & Cyclic convolution formula: $(a*b)_k = \sum_{j=0}^{3} a_j b_{k-j \mod 4}$. & Definition \\
			2 & For $k=0$: $(a*b)_0 = a_0b_0 + a_1b_3 + a_2b_2 + a_3b_1$. & $k=0$ case \\
			3 & Compute: $1\times0 + 0\times0 + 2\times1 + 1\times1 = 0+0+2+1=3$. & Calculation \\
			4 & For $k=1$: $(a*b)_1 = a_0b_1 + a_1b_0 + a_2b_3 + a_3b_2$. & $k=1$ case \\
			5 & Compute: $1\times1 + 0\times0 + 2\times0 + 1\times1 = 1+0+0+1=2$. & Calculation \\
			6 & For $k=2$: $(a*b)_2 = a_0b_2 + a_1b_1 + a_2b_0 + a_3b_3$. & $k=2$ case \\
			7 & Compute: $1\times1 + 0\times1 + 2\times0 + 1\times0 = 1+0+0+0=1$. & Calculation \\
			8 & For $k=3$: $(a*b)_3 = a_0b_3 + a_1b_2 + a_2b_1 + a_3b_0$. & $k=3$ case \\
			9 & Compute: $1\times0 + 0\times1 + 2\times1 + 1\times0 = 0+0+2+0=2$. & Calculation \\
			10 & Result: $a * b = [3, 2, 1, 2]$. & Final answer \\
			11 & Check via DFT: $\hat{a}=[4, -1-i, 2, -1+i]$, $\hat{b}=[2, -1+i, 0, -1-i]$. & DFT verification \\
			12 & Pointwise product: $\hat{a}\hat{b}=[8, 2i, 0, -2i]$. & DFT multiplication \\
			13 & Inverse DFT gives $[3,2,1,2]$, confirming result. & Verification complete \\
		\end{prooftable}
	\end{proof}	
	\item \textbf{Formal Power Series Units}\\
	In $R[[x]]$ with $R=\mathbb{Z}_7$, determine if $f=3+2x+5x^2$ is a unit and find its inverse modulo $x^3$.	
	\vspace{5pt}\begin{proof}[\underline{Solution}]
		\begin{prooftable}
			1 & Theorem: $f=\sum a_n x^n$ is unit iff $a_0$ is unit in $R$. & Unit criterion \\
			2 & Here $a_0=3$, need to check if 3 is unit in $\mathbb{Z}_7$. & Check constant term \\
			3 & In $\mathbb{Z}_7$, $3\times5=15\equiv1$, so 3 is unit with inverse 5. & Modular inverse \\
			4 & Therefore $f$ is a unit in $\mathbb{Z}_7[[x]]$. & Conclusion \\
			5 & Find inverse $g=\sum b_n x^n$ with $f*g=1$. & Inverse definition \\
			6 & From $f*g=1$: Constant term gives $a_0b_0=1$. & First equation \\
			7 & So $3b_0\equiv1\Rightarrow b_0\equiv5\pmod7$. & Solve for $b_0$ \\
			8 & Coefficient of $x$: $a_0b_1 + a_1b_0 = 0$. & Second equation \\
			9 & $3b_1 + 2\times5 = 0\Rightarrow 3b_1 + 10\equiv0\Rightarrow 3b_1\equiv4$. & Simplify \\
			10 & Multiply by $5$ (inverse of 3): $b_1\equiv4\times5=20\equiv6\pmod7$. & Solve for $b_1$ \\
			11 & Coefficient of $x^2$: $a_0b_2 + a_1b_1 + a_2b_0 = 0$. & Third equation \\
			12 & $3b_2 + 2\times6 + 5\times5 = 0\Rightarrow 3b_2 + 12 + 25 = 0$. & Substitute \\
			13 & $3b_2 + 37\equiv0\Rightarrow 3b_2\equiv-37\equiv5\pmod7$. & Simplify mod 7 \\
			14 & Multiply by 5: $b_2\equiv5\times5=25\equiv4\pmod7$. & Solve for $b_2$ \\
			15 & Thus $g = 5 + 6x + 4x^2$ modulo $x^3$. & Inverse polynomial \\
			16 & Verify: $f*g = (3+2x+5x^2)(5+6x+4x^2)$. & Verification \\
			17 & Compute: $15 + (18+10)x + (12+12+25)x^2 + \text{higher terms}$. & Multiplication \\
			18 & Mod 7: $1 + 0x + (49)x^2 \equiv 1 + 0x + 0x^2$. & Modulo 7 reduction \\
			19 & Higher terms involve $x^3$ and beyond, ignored modulo $x^3$. & Modulo $x^3$ \\
			20 & Result is 1 as required, confirming inverse is correct. & Verification complete \\
		\end{prooftable}
	\end{proof}
	\item \textbf{Total Convolution Algebra Unit}\\
	Show that $\delta_0$ is the unit in $\mathcal{C}(\mathbb{Z})$, where $\delta_0(x)=1$ if $x=0$ and 0 otherwise.
	\vspace{5pt}\begin{proof}[\underline{Solution}]
		\begin{prooftable}
			1 & For any $f: \mathbb{Z}\to\mathbb{C}$, need $(f*\delta_0)(x)=f(x)$. & Requirement \\
			2 & Convolution: $(f*\delta_0)(x)=\sum_{y\in\mathbb{Z}} f(y)\delta_0(x-y)$. & Definition \\
			3 & $\delta_0(x-y)=1$ only when $x-y=0$, i.e., $y=x$. & Delta function property \\
			4 & Thus the sum reduces to single term: $f(x)\delta_0(0)=f(x)\times1$. & Sum simplification \\
			5 & So $(f*\delta_0)(x)=f(x)$ for all $x\in\mathbb{Z}$. & Right identity \\
			6 & Similarly, $(\delta_0*f)(x)=\sum_{y\in\mathbb{Z}} \delta_0(y)f(x-y)$. & Left convolution \\
			7 & $\delta_0(y)=1$ only when $y=0$. & Delta function \\
			8 & Thus $(\delta_0*f)(x)=f(x-0)=f(x)$. & Left identity \\
			9 & Therefore $\delta_0$ acts as multiplicative identity. & Conclusion \\
			10 & Check algebra axioms: $\mathcal{C}(\mathbb{Z})$ is associative. & Algebra property \\
			11 & Distributive over addition: $(f+g)*h = f*h + g*h$. & Distributivity \\
			12 & Thus $\mathcal{C}(\mathbb{Z})$ is unital algebra with unit $\delta_0$. & Final statement \\
		\end{prooftable}
	\end{proof}
	\item \textbf{Discrete Fourier Transform Verification}\\
	Verify the isomorphism $\mathcal{C}_3 \cong \mathbb{C}[x]/(x^3-1)$ for vector $a=[1,i,0]$.
	\vspace{5pt}\begin{proof}[\underline{Solution}]
		\begin{prooftable}
			1 & DFT matrix for $n=3$: $F_{jk} = \omega^{jk}$ with $\omega=e^{2\pi i/3}$. & DFT definition \\
			2 & Explicitly: $F = \frac{1}{\sqrt{3}}\begin{pmatrix}1&1&1\\1&\omega&\omega^2\\1&\omega^2&\omega\end{pmatrix}$. & Matrix form \\
			3 & Apply to $a=[1,i,0]$: $\hat{a} = F a$. & DFT application \\
			4 & $\hat{a}_0 = 1 + i + 0 = 1+i$. & First component \\
			5 & $\hat{a}_1 = 1 + i\omega + 0 = 1 + i(-\frac{1}{2}+i\frac{\sqrt{3}}{2})$. & Second component \\
			6 & Simplify: $1 - \frac{i}{2} - \frac{\sqrt{3}}{2}$. & Calculation \\
			7 & $\hat{a}_2 = 1 + i\omega^2 + 0 = 1 + i(-\frac{1}{2}-i\frac{\sqrt{3}}{2})$. & Third component \\
			8 & Simplify: $1 - \frac{i}{2} + \frac{\sqrt{3}}{2}$. & Calculation \\
			9 & Polynomial representation: map $a$ to $p(x)=1+ix$. & Polynomial correspondence \\
			10 & In quotient ring $\mathbb{C}[x]/(x^3-1)$: $x^3=1$. & Quotient relation \\
			11 & So $p(x)=1+ix$ is already reduced (degree$<$3). & Reduced form \\
			12 & Convolution corresponds to polynomial multiplication mod $x^3-1$. & Isomorphism property \\
			13 & Check: $(1+ix)*(1+ix)=1+2ix-x^2$. & Polynomial multiplication \\
			14 & Mod $x^3-1$: $1+2ix-x^2$ (no reduction needed). & Modulo reduction \\
			15 & DFT of convolution = pointwise multiplication of DFTs. & DFT property \\
			16 & For $b=[1,i,0]$ as well: $\hat{b}=\hat{a}$, product $\hat{a}^2$. & Example \\
			17 & Pointwise square: $(1+i)^2=2i$, etc. & Verification \\
			18 & Inverse DFT gives convolution result, matching polynomial multiplication. & Consistency check \\
			19 & Therefore isomorphism holds. & Conclusion \\
		\end{prooftable}
	\end{proof}
	\item \textbf{Truncated Convolution on Finite Set}\\
	Let $S=\{0,1,2\}$ and $f,g:S\to\mathbb{C}$,\\ with $f(0)=1, f(1)=i, f(2)=0$, $g(0)=0, g(1)=1, g(2)=i$. Compute $(f\underset{S}{\ast } g)$.
	\vspace{5pt}\begin{proof}[\underline{Solution}]
		\begin{prooftable}
			1 & Truncated convolution: $(f\underset{S}{\ast } g)(x)=\sum_{\substack{y\in S \\ x-y\in S}} f(y)g(x-y)$. & Definition \\
			2 & For $x=0$: Need $y\in S$ and $0-y\in S$ $\Rightarrow $ $y=0$. & $x=0$ case \\
			3 & $(f\underset{S}{\ast } g)(0) = f(0)g(0) = 1\times0 = 0$. & Calculation \\
			4 & For $x=1$: Need $y\in S$ and $1-y\in S$. & $x=1$ case \\
			5 & Possible $y$: 0 (since $1-0=1\in S$), 1 (since $1-1=0\in S$). & Valid indices \\
			6 & $(f\underset{S}{\ast } g)(1) = f(0)g(1) + f(1)g(0) = 1\times1 + i\times0 = 1$. & Calculation \\
			7 & For $x=2$: Need $y\in S$ and $2-y\in S$. & $x=2$ case \\
			8 & Possible $y$: 0 (since $2-0=2\in S$), 1 (since $2-1=1\in S$), 2 (since $2-2=0\in S$). & Valid indices \\
			9 & $(f\underset{S}{\ast } g)(2) = f(0)g(2) + f(1)g(1) + f(2)g(0)$. & Summation \\
			10 & Compute: $1\times i + i\times1 + 0\times0 = i + i = 2i$. & Calculation \\
			11 & Result: $(f\underset{S}{\ast } g) = [0, 1, 2i]$. & Final answer \\
			12 & Compare with full convolution on $\mathbb{Z}$: would include more terms. & Comparison \\
			13 & Truncation restricts to indices where both factors defined. & Truncation effect \\
			14 & This preserves algebra structure on finite support functions. & Algebraic property \\
		\end{prooftable}
	\end{proof}
\end{exerlist}
\subsection{Łukasiewicz Logic and Modular Set Theory}
\subsubsection{Many-Valued Algebras}
\begin{definition}[Many-Valued Algebra $MV_p$,\, \cite{ Many-Valued1970selected,gottwald2001treatise,cignoli2000algebraic}]\label{def:Many-Valued-algebra}
	For integer $p \geq 2$, the \emph{Many-Valued Algebra} $MV_p$ is the algebraic structure $(V_p, \tneg, \tand, \tor)$ where:
	\begin{enumerate}[label=(\roman*)]
		\item The truth value set is:
		\begin{equation}
		V_p = \left\{0, \frac{1}{p-1}, \frac{2}{p-1}, \dots, \frac{p-2}{p-1}, 1\right\}
		\end{equation}
		with $0$ representing absolute falsehood and $1$ absolute truth.	
		\item The \emph{Many-Valued negation} is defined as:
		\begin{equation}
		\tneg x = 1 - x \quad \text{for all } x \in V_p.
		\end{equation}	
		\item The \emph{Many-Valued conjunction} (strong conjunction) is:
		\begin{equation}
		x \tand y = \max(0, x + y - 1) \quad \text{for all } x, y \in V_p.
		\end{equation}	
		\item The \emph{Many-Valued disjunction} (strong disjunction) is:
		\begin{equation}
		x \tor y = \min(1, x + y) \quad \text{for all } x, y \in V_p.
		\end{equation}
		\item The \emph{Many-Valued implication} is defined as:
		\begin{equation}
		x \timp y = \min(1, 1 - x + y) \quad \text{for all } x, y \in V_p.
		\end{equation}
		\item The \emph{Many-Valued equivalence} is:
		\begin{equation}
		x \tequiv y = 1 - |x - y| \quad \text{for all } x, y \in V_p.
		\end{equation}
	\end{enumerate}
	This structure forms a Many-Valued logic system generalizing Boolean algebra ($MV_2$) to $p$ truth values.
\end{definition}
\begin{remark}[ \cite{dubois1980fuzzy}]
	For $p=2$, $MV_2$ reduces to classical Boolean algebra with $V_2 = \{0,1\}$ and operations:
	\begin{equation}
	\tneg x = 1 - x,\quad x \tand y = \min(x,y),\quad x \tor y = \max(x,y).
	\end{equation}
	Thus Boolean algebra is a special case of Many-Valued Algebras.
\end{remark}
\begin{definition}[Prime-Modular Many-Valued Algebra, \, \cite{cignoli2000algebraic}]\label{def:prime-modular-Many-Valued}
	For \emph{prime} $p$, the Many-Valued Algebra $MV_p$ admits a \emph{modular interpretation} where truth values are embedded in the finite field $\mathbb{F}_p$ via the bijection:
	\begin{equation}
	\phi: V_p \to \mathbb{F}_p,\quad \phi\left(\frac{k}{p-1}\right) = k \pmod{p}.
	\end{equation}
	Under this identification, the operations become:
	\begin{align*}
		\tneg x &\equiv 1 - x \pmod{p},\\
		x \tand y &\equiv \max(0, x + y - 1) \pmod{p},\\
		x \tor y &\equiv \min(p-1, x + y) \pmod{p}.
	\end{align*}
	This establishes a connection between Many-Valued logic and modular arithmetic.
\end{definition}
\begin{definition}[Monoidal Operations, \, \cite{cignoli2000algebraic}]\label{def:monoidal-operations}
	In addition to the strong operations, $MV_p$ admits \emph{weak operations}:
	\begin{itemize}
		\item \emph{Weak conjunction}: $x \land y = \min(x, y)$
		\item \emph{Weak disjunction}: $x \lor y = \max(x, y)$
	\end{itemize}
	These form a \emph{De Morgan algebra} structure on $V_p$, while the strong operations give a \emph{MV-algebra} structure.
\end{definition}
\begin{example}[Three-Valued Logic $MV_3$, \, \cite{gottwald2001treatise}]\label{ex:L3}
	For $p=3$, we have $V_3 = \{0, \frac{1}{2}, 1\}$ with interpretations:
	\begin{equation}
	0 = \text{false},\quad \tfrac{1}{2} = \text{indeterminate},\quad 1 = \text{true}.
	\end{equation}
	Operation tables (normalized to $\{0,1,2\}$ for clarity):
\vspace{5pt}
\noindent

		\begin{minipage}{0.3\textwidth}
			\centering
			\begin{tabular}{c|ccc}
				$\tand$ & 0 & 1 & 2\\ 
				\hline
				0 & 0 & 0 & 0\\
				1 & 0 & 0 & 1\\
				2 & 0 & 1 & 2
			\end{tabular}
		\end{minipage}
		\hfill
		\begin{minipage}{0.3\textwidth}
			\centering
			\begin{tabular}{c|ccc}
				$\tor$ & 0 & 1 & 2\\ 
				\hline
				0 & 0 & 1 & 2\\
				1 & 1 & 2 & 2\\
				2 & 2 & 2 & 2
			\end{tabular}
		\end{minipage}
		\hfill
		\begin{minipage}{0.3\textwidth}
			\centering
			\begin{tabular}{c|c}
				$x$ & $\tneg x$\\ 
				\hline
				0 & 2\\
				1 & 1\\
				2 & 0
			\end{tabular}
		\end{minipage}

\mbox{}

\vspace{10pt}

	This three-valued system has applications in para-consistent logic and fault-tolerant systems.
\end{example}
\begin{definition}[Polynomial Completeness, \, \cite{rosenberg1970functional}]\label{def:polynomial-completeness}
	Many-Valued Algebra $MV_p$ is \emph{polynomially complete} for prime $p$: every function $f: V_p^n \to V_p$ can be expressed as a polynomial in the Many-Valued logic operations. For composite $p$, additional congruence-preserving conditions are required.
\end{definition}
\begin{definition}[Prime-Modular Many-Valued Logic, \, \cite{Many-Valued1970selected,cignoli2000algebraic}] 
	For prime $p$, a \emph{prime-modular Many-Valued logic} is a logic system where truth values are embedded in $\mathbb{Z}/p\mathbb{Z}$, and logical operations are interpreted modulo $p$.
\end{definition}
\begin{definition}[Many-Valued Implication, \, \cite{Many-Valued1970selected,gottwald2001treatise}] 
	In Many-Valued logic, implication is defined as
	\begin{equation}
	x \to y = \min(1, 1 - x + y).
	\end{equation}
	In the modular setting for prime $p$, this becomes $x \to y \equiv (1 - x + y) \pmod{p}$.
\end{definition}
\subsubsection{Modular Set Algebras}
\begin{definition}[Modular Set Algebra $S_p$, \, \cite{diprisco2019many,cignoli2000algebraic}] 
	For prime $p$, the \emph{modular set algebra} is $S_p = \mathcal{P}(\mathbb{Z}_p)/\sim$, where $\mathbb{Z}_p = \mathbb{Z}/p\mathbb{Z}$, and the equivalence relation $\sim$ is defined by:
	\begin{equation}
	A \sim B \quad \text{iff} \quad |A| \equiv |B| \pmod{p}.
	\end{equation}
\end{definition}
\begin{definition}[Modular Set Operations, \, \cite{diprisco2019many,cignoli2000algebraic}] 
	For equivalence classes $[A], [B] \in S_p$, define:
	\begin{align*}
	[A] \sqcap [B] &:= [A \cap B], \\
	[A] \sqcup [B] &:= [A \cup B], \\
	\neg [A] &:= [\mathbb{Z}_p \setminus A].
	\end{align*}
\end{definition}
\begin{definition}[Modular Set Isomorphism, \, \cite{diprisco2019many,cignoli2000algebraic}] 
	The \emph{Prime-Modular Logic-Set Isomorphism} is the map $\phi: MV_p \to S_p$ defined by
	\begin{equation}
	\phi\left(\frac{k}{p-1}\right) = [\{0, 1, \dots, k-1\}]_{\sim}, \quad k = 0, 1, \dots, p-1.
	\end{equation}
\end{definition}
\subsubsection{Valuation Functions}
\begin{definition}[Modular Characteristic Function, \, \cite{diprisco2019many,dubois1980fuzzy}] 
	For $A \subseteq \mathbb{Z}$ and prime $p$, the \emph{modular characteristic function} $\chi_A^p: \mathbb{Z}_p \to \{0,1\}$ is defined by
	\begin{equation}
	\chi_A^p([x]) = \begin{cases}
	1 & \text{if there exists } a \in A \text{ with } a \equiv x \pmod{p}, \\
	0 & \text{otherwise}.
	\end{cases}
	\end{equation}
\end{definition}
\begin{definition}[Many-Valued Valuation, \, \cite{diprisco2019many,gottwald2001treatise}] 
	For $A \subseteq \mathbb{Z}$ and prime $p$, the \emph{Many-Valued valuation} is
	\begin{equation}
	v_p(A) = \max_{\preceq} \{ \chi_A^p([x]) \cdot [x] \mid [x] \in \mathbb{Z}_p \},
	\end{equation}
	where $\cdot$ denotes Many-Valued conjunction and $\preceq$ is the truth order.
\end{definition}
\begin{theorem}[Representation Theorem,\, \cite{mcnaughton1951theorem}]\label{thm:Many-Valued-representation}
	Every $n$-variable function $f: V_p^n \to V_p$ definable in Many-Valued logic $MV_p$ is a piecewise linear function with integer coefficients, and conversely, every such piecewise linear function preserving $V_p$ is definable in $MV_p$.
\end{theorem}

\begin{proposition}[Algebraic Properties of $MV_p$]\label{prop:Many-Valued-properties}
	For any $p \geq 2$ and all $x,y,z \in MV_p$:
	\begin{enumerate}[label=(\alph*)]
		\item \textbf{Double negation}: $\tneg \tneg x = x$
		\item \textbf{De Morgan laws}: 
		$\tneg(x \tand y) = \tneg x \tor \tneg y$ and 
		$\tneg(x \tor y) = \tneg x \tand \tneg y$
		\item \textbf{Commutativity}: 
		$x \tand y = y \tand x$, $x \tor y = y \tor x$
		\item \textbf{Associativity}: 
		$(x \tand y) \tand z = x \tand (y \tand z)$, 
		$(x \tor y) \tor z = x \tor (y \tor z)$
		\item \textbf{Distributivity}: 
		$x \tand (y \tor z) = (x \tand y) \tor (x \tand z)$
		\item \textbf{Boundary conditions}: 
		$x \tand 0 = 0$, $x \tor 1 = 1$, 
		$x \tand 1 = x$, $x \tor 0 = x$
		\item \textbf{Non-idempotence}: 
		In general, $x \tand x \neq x$ and $x \tor x \neq x$ for $p>2$
	\end{enumerate}
\end{proposition}
\subsubsection{Prime-Modular Logic-Set Isomorphism}
\begin{theorem}[Prime-Modular Logic-Set Isomorphism, \, \cite{Many-Valued1970selected,cignoli2000algebraic}] 
	For prime $p$, the Many-Valued Algebra $MV_p$ is isomorphic to the modular set algebra $S_p$. The isomorphism $\phi: MV_p \to S_p$ is given by
	\begin{equation}
	\phi\left(\frac{k}{p-1}\right) = [\{0, 1, \dots, k-1\}]_{\sim},
	\end{equation}
	where $[A]_{\sim}$ denotes equivalence class under cardinality modulo $p$.
\end{theorem}
\begin{corollary}[Sharp Bounds for $p > 2$, \, \cite{gottwald2001treatise,cignoli2000algebraic}] 
	For prime $p > 2$ and any $x \in MV_p$,
	\begin{equation}
	x \land \neg x \leq \frac{p-1}{2}, \quad x \lor \neg x \geq \frac{p-1}{2},
	\end{equation}
	where bounds are in the normalized scale $[0,1]$.
\end{corollary}
\begin{theorem}[Polynomial Constraint Characterization, \, \cite{cignoli2000algebraic,gottwald2001treatise}] 
	A polynomial identity $P(x_1, \dots, x_n) = Q(x_1, \dots, x_n)$ holds in all Many-Valued Algebras $MV_p$ if and only if it holds in $MV_2$ (Boolean algebra).
\end{theorem}
\begin{exerlist}
	\item \textbf{Three-Valued Logic Tables ($p=3$)}\\
	Construct complete truth tables for negation, conjunction, disjunction, and implication in $\mathbb{Z}_3 = \{0,1,2\}$ with $0=\text{False}, 1=\text{Unknown}, 2=\text{True}$.
	\vspace{5pt}\begin{proof}[\underline{Solution}]
		\textbf{Negation: $\overline{x} = 2 - x$}
		\begin{prooftable}
			1 & For $x=0$: $\overline{0}=2-0=2$ (True). & Calculation \\
			2 & For $x=1$: $\overline{1}=2-1=1$ (Unknown). & Calculation \\
			3 & For $x=2$: $\overline{2}=2-2=0$ (False). & Calculation \\
			4 & Table: $\begin{array}{c|c}x&\overline{x}\\  0&2\\1&1\\2&0\end{array}$ & Complete table \\
		\end{prooftable}
	\vspace{10pt}	
		\textbf{Conjunction: $x \land y = \min(x,y)$}
		\begin{prooftable}
			1 & Compute all 9 combinations: & Method \\
			2 & $\min(0,0)=0$, $\min(0,1)=0$, $\min(0,2)=0$. & Row 0 \\
			3 & $\min(1,0)=0$, $\min(1,1)=1$, $\min(1,2)=1$. & Row 1 \\
			4 & $\min(2,0)=0$, $\min(2,1)=1$, $\min(2,2)=2$. & Row 2 \\
			5 & Table: $\begin{array}{c|ccc}\land&0&1&2\\  0&0&0&0\\1&0&1&1\\2&0&1&2\end{array}$ & Complete table \\
		\end{prooftable}
	\vspace{10pt}
		\textbf{Disjunction: $x \lor y = \max(x,y)$}
		\begin{prooftable}
			1 & Compute all 9 combinations: & Method \\
			2 & $\max(0,0)=0$, $\max(0,1)=1$, $\max(0,2)=2$. & Row 0 \\
			3 & $\max(1,0)=1$, $\max(1,1)=1$, $\max(1,2)=2$. & Row 1 \\
			4 & $\max(2,0)=2$, $\max(2,1)=2$, $\max(2,2)=2$. & Row 2 \\
			5 & Table: $\begin{array}{c|ccc}\lor&0&1&2\\  0&0&1&2\\1&1&1&2\\2&2&2&2\end{array}$ & Complete table \\
		\end{prooftable}
	\vspace{10pt}
		\textbf{Implication: $x \to y = \min(2, 2 - x + y)$}\text{ \  }\\
		\begin{prooftable}
			1 & For $x=0, y=0$: $\min(2, 2-0+0)=\min(2,2)=2$. & Calculation \\
			2 & $x=0, y=1$: $\min(2, 2-0+1)=\min(2,3)=2$. & Calculation \\
			3 & $x=0, y=2$: $\min(2, 2-0+2)=\min(2,4)=2$. & Calculation \\
			4 & $x=1, y=0$: $\min(2, 2-1+0)=\min(2,1)=1$. & Calculation \\
			5 & $x=1, y=1$: $\min(2, 2-1+1)=\min(2,2)=2$. & Calculation \\
			6 & $x=1, y=2$: $\min(2, 2-1+2)=\min(2,3)=2$. & Calculation \\
			7 & $x=2, y=0$: $\min(2, 2-2+0)=\min(2,0)=0$. & Calculation \\
			8 & $x=2, y=1$: $\min(2, 2-2+1)=\min(2,1)=1$. & Calculation \\
			9 & $x=2, y=2$: $\min(2, 2-2+2)=\min(2,2)=2$. & Calculation \\
			10 & Table: $\begin{array}{c|ccc}\to&0&1&2\\  0&2&2&2\\1&1&2&2\\2&0&1&2\end{array}$ & Complete table \\
		\end{prooftable}
	\end{proof}
	\item \textbf{Many-Valued logic Operations for $p=5$}\\
	For $p=5$, compute:
	\begin{exampleitems}
		\item $\overline{2}$ (negation),
		\item $3 \cdot 1$ (conjunction),
		\item $3 + 1$ (disjunction),
		\item $3 \to 1$ (implication).
	\end{exampleitems}
	\vspace{5pt}\begin{proof}[\underline{Solution}]
		\begin{prooftable}
			1 & Truth values: $\{0,1,2,3,4\}$ with $4$ as "True", $0$ as "False". & Value set \\
			2 & Negation: $\overline{x} = 4 - x$ (since $p-1=4$). & Negation formula \\
			3 & (a) $\overline{2} = 4 - 2 = 2$. & Calculation \\
			4 & Conjunction: $x \land y = \max(0, x+y-4)$. & Conjunction formula \\
			5 & (b) $3 \land 1 = \max(0, 3+1-4) = \max(0, 0) = 0$. & Calculation \\
			6 & Disjunction: $x \lor y = \min(4, x+y)$. & Disjunction formula \\
			7 & (c) $3 \lor 1 = \min(4, 3+1) = \min(4, 4) = 4$. & Calculation \\
			8 & Implication: $x \to y = \min(4, 4-x+y)$. & Implication formula \\
			9 & (d) $3 \to 1 = \min(4, 4-3+1) = \min(4, 2) = 2$. & Calculation \\
			10 & Alternative interpretation: using $\min$/$\max$ directly. & Note \\
			11 & For $p=5$, can also use $\min(x,y)$ for conjunction. & Alternative \\
			12 & $\min(3,1)=1$ (different from above due to normalization). & Different convention \\
			13 & Clarify: Our formulas use Many-Valued logic operations properly normalized. & Convention note \\
			14 & Normalized values: divide by 4: $3/4=0.75$, $1/4=0.25$. & Normalization \\
			15 & $\overline{0.75}=1-0.75=0.25$ corresponds to $1$ unnormalized. & Check consistency \\
		\end{prooftable}
	\end{proof}
	\item \textbf{Polynomial Constraint for $p=2$}\\
	For $p=2$, verify that $q(x) = 1-x$ satisfies:
	\begin{align*}
	x \cdot q(x) &= 0, \quad  x \in \mathbb{F}_2 \\
	x + q(x) &= 1, \quad  x \in \mathbb{F}_2
	\end{align*}
	\begin{proof}[\underline{Solution}]
		\begin{prooftable}
			1 & $\mathbb{F}_2 = \{0,1\}$ with operations mod 2. & Field definition \\
			2 & For $x=0$: $q(0)=1-0=1$. & Calculation \\
			3 & Check $x \cdot q(x)$: $0 \cdot 1 = 0$. \correct  & First condition \\
			4 & Check $x + q(x)$: $0 + 1 = 1$. \correct  & Second condition \\
			5 & For $x=1$: $q(1)=1-1=0$. & Calculation \\
			6 & Check $x \cdot q(x)$: $1 \cdot 0 = 0$. \correct  & First condition \\
			7 & Check $x + q(x)$: $1 + 0 = 1$. \correct  & Second condition \\
			8 & Both conditions satisfied for all $x\in\mathbb{F}_2$. & Conclusion \\
			9 & Note: This works only for $p=2$. & Specific to p=2 \\
			10 & For $p>2$, such $q(x)$ doesn't exist. & General statement \\
		\end{prooftable}
	\end{proof}
	\item \textbf{Polynomial Constraint for $p=3$}\\
	For $p=3$, check if any polynomial $q(x) \in \mathbb{F}_3[x]$ satisfies:
	\begin{align*}
	x \cdot q(x) &= 0 \quad ,  x \in \{0,1,2\} \\
	x + q(x) &= k \quad (\text{constant } k)
	\end{align*}
\begin{proof}[\underline{Solution}]
		\begin{prooftable}
			1 & From $x+q(x)=k$, deduce $q(x)=k-x$ for all $x$. & Equation solution \\
			2 & Substitute into first condition: $x(k-x)=0$ for all $x$. & Substitution \\
			3 & Test $x=0$: $0(k-0)=0$ automatically true. & Case x=0 \\
			4 & Test $x=1$: $1(k-1)=0$ $\Rightarrow $ $k-1=0$ $\Rightarrow $ $k=1$. & Case x=1 \\
			5 & Test $x=2$: $2(k-2)=0$ $\Rightarrow $ $2(k-2)\equiv0\pmod3$. & Case x=2 \\
			6 & Since $2^{-1}=2$ in $\mathbb{F}_3$, multiply: $k-2=0$ $\Rightarrow $ $k=2$. & Solve \\
			7 & Contradiction: $k$ must be both 1 and 2. & Inconsistency \\
			8 & Therefore no such polynomial $q(x)$ exists for $p=3$. & Conclusion \\
			9 & This illustrates theorem: such constraints only satisfiable for $p=2$. & Theorem example \\
		\end{prooftable}
	\end{proof}
	\item \textbf{Modular Set Equivalence for $p=3$}
	Let $A = \{0, 3, 6, 9\}$, $B = \{2, 5, 8, 11\}$.\\ Compute their residue sets modulo 3 and check if $A \equiv B \pmod{3}$.
	\vspace{5pt}\begin{proof}[\underline{Solution}]
		\begin{prooftable}
			1 & Compute $A \bmod 3$: $\{0\bmod3, 3\bmod3, 6\bmod3, 9\bmod3\}$. & Residue calculation \\
			2 & Results: $\{0, 0, 0, 0\} = \{0\}$. & Simplified set \\
			3 & Compute $B \bmod 3$: $\{2\bmod3, 5\bmod3, 8\bmod3, 11\bmod3\}$. & Residue calculation \\
			4 & Results: $\{2, 2, 2, 2\} = \{2\}$. & Simplified set \\
			5 & Equivalence definition: $A \equiv B \pmod{p}$ if their residue sets are equal. & Definition \\
			6 & Here $\{0\} \neq \{2\}$, so $A \not\equiv B \pmod{3}$. & Comparison \\
			7 & Alternative: Cardinalities modulo 3: $|A|=4\equiv1$, $|B|=4\equiv1$. & Cardinality approach \\
			8 & But equivalence requires matching residue patterns, not just cardinalities. & Clarification \\
			9 & Possible residue patterns for size 4 sets modulo 3. & General consideration \\
			10 & Since 4 $\equiv $ 1 mod 3, all size 4 sets have the same cardinality modulo 3. & Cardinality modulo p \\
			11 & Yet $A$ and $B$ have different residue distributions. & Key difference \\
			12 & Therefore $A \not\equiv B$ under modular set equivalence. & Final conclusion \\
		\end{prooftable}
	\end{proof}
	\item \textbf{Valuation Function for $p=3$}\\
	Let $p=3$ and $A = 3\mathbb{Z} = \{0, 3, 6, \dots\}$.\\ Compute the modular characteristic function $\chi_A^3$ and the Many-Valued logic valuation $v_3(A)$.
	\begin{proof}[\underline{Solution}]
		\begin{prooftable}
			1 & Residue classes modulo 3: $[0], [1], [2]$. & Classes \\
			2 & $\chi_A^3([0]) = 1$ since $0\in A$ and all multiples of 3 give residue 0. & Characteristic at [0] \\
			3 & $\chi_A^3([1]) = 0$ since no element of $A$ has residue 1. & Characteristic at [1] \\
			4 & $\chi_A^3([2]) = 0$ since no element of $A$ has residue 2. & Characteristic at [2] \\
			5 & Valuation: $v_p(A) = \max_{\preceq} \{ \chi_A^p([x]) \cdot [x] : [x] \in \mathbb{Z}_p \}$. & Definition \\
			6 & Here $\cdot$ is Many-Valued logic conjunction (minimum for normalized values). & Operation \\
			7 & Compute pairs: $(\chi([0]), [0]) = (1, 0)$ gives $1 \cdot 0 = \min(1,0)=0$. & First pair \\
			8 & $(\chi([1]), [1]) = (0, 1)$ gives $0 \cdot 1 = \min(0,1)=0$. & Second pair \\
			9 & $(\chi([2]), [2]) = (0, 2)$ gives $0 \cdot 2 = \min(0,2)=0$. & Third pair \\
			10 & All results are 0, so maximum is 0. & Maximum \\
			11 & Thus $v_3(A) = 0$. & Final valuation \\
			12 & Interpretation: $A$ only contains multiples of 3, so its "truth value" is minimal. & Interpretation \\
			13 & Contrast with $B=\mathbb{Z}$: $\chi_B^3([x])=1$ for all $x$. & Comparison \\
			14 & Then $v_3(B)=\max\{\min(1,0), \min(1,1), \min(1,2)\}=\max\{0,1,2\}=2$. & Other example \\
		\end{prooftable}
	\end{proof}
	\item \textbf{Sharp Bounds Verification for $p=11$}\\
	For $p=11$, verify the bounds $x \cdot \overline{x} \leq 5$ and $x + \overline{x} \geq 5$ for $x=2$, $x=5$, and $x=8$.
	\begin{proof}[\underline{Solution}]
		Note: For $p=11$, maximum value is 10, midpoint is 5.
		\begin{prooftable}
			1 & Negation: $\overline{x} = 10 - x$. & Negation formula \\
			2 & Conjunction: $x \land y = \max(0, x+y-10)$ (unnormalized). & Conjunction formula \\
			3 & Disjunction: $x \lor y = \min(10, x+y)$ (unnormalized). & Disjunction formula \\
			\multicolumn{3}{c}{\textbf{Case 1: $x=2$}} \\
			4 & $\overline{2} = 10-2 = 8$. & Negation \\
			5 & $2 \land 8 = \max(0, 2+8-10) = \max(0, 0) = 0$. & Conjunction \\
			6 & Check $0 \leq 5$: \correct  & First bound \\
			7 & $2 \lor 8 = \min(10, 2+8) = \min(10, 10) = 10$. & Disjunction \\
			8 & Check $10 \geq 5$: \correct  & Second bound \\
			\multicolumn{3}{c}{\textbf{Case 2: $x=5$}} \\
			9 & $\overline{5} = 10-5 = 5$. & Negation \\
			10 & $5 \land 5 = \max(0, 5+5-10) = \max(0, 0) = 0$. & Conjunction \\
			11 & Check $0 \leq 5$: \correct  & First bound \\
			12 & $5 \lor 5 = \min(10, 5+5) = \min(10, 10) = 10$. & Disjunction \\
			13 & Check $10 \geq 5$: \correct  & Second bound \\
			\multicolumn{3}{c}{\textbf{Case 3: $x=8$}} \\
			14 & $\overline{8} = 10-8 = 2$. & Negation \\
			15 & $8 \land 2 = \max(0, 8+2-10) = \max(0, 0) = 0$. & Conjunction \\
			16 & Check $0 \leq 5$: \correct  & First bound \\
			17 & $8 \lor 2 = \min(10, 8+2) = \min(10, 10) = 10$. & Disjunction \\
			18 & Check $10 \geq 5$: \correct  & Second bound \\
			19 & All cases satisfy the bounds. & Conclusion \\
			20 & Theorem says these hold for ALL $x$ when $p>2$. & General theorem \\
		\end{prooftable}
	\end{proof}
	\item \textbf{Composite Modulus Case ($n=4$)}\\
	For composite $n=4$, compute for all $x\in\{0,1,2,3\}$:
	\begin{exampleitems}
		\item $x \cdot \overline{x}$, 
		\item $x + \overline{x}$, 
		\item Verify bounds $x \cdot \overline{x} \leq 1$, $x + \overline{x} \geq 2$.
	\end{exampleitems}
	\begin{proof}[\underline{Solution}]
		For $n=4$, $\overline{x} = 3-x$ (since maximum is 3).\text{ \ } \\	
		\begin{prooftable}
			1 & Create table with all calculations: & Method \\
			\multicolumn{3}{c}{\begin{tabular}{cccccc}
					$x$ & $\overline{x}=3-x$ & $x \cdot \overline{x}$ & Bound $\leq1$. & $x + \overline{x}$ & Bound $\geq2$. \\ 
					0 & 3 & $\max(0,0+3-3)=0$ & $0\leq1$ \correct  & $\min(3,0+3)=3$ & $3\geq2$ \correct \\
					1 & 2 & $\max(0,1+2-3)=0$ & $0\leq1$ \correct  & $\min(3,1+2)=3$ & $3\geq2$ \correct \\
					2 & 1 & $\max(0,2+1-3)=0$ & $0\leq1$ \correct  & $\min(3,2+1)=3$ & $3\geq2$ \correct \\
					3 & 0 & $\max(0,3+0-3)=0$ & $0\leq1$ \correct  & $\min(3,3+0)=3$ & $3\geq2$ \correct \\
			\end{tabular}} \\
			2 & For $x=0$: $\overline{0}=3$, $0\land3=\max(0,0+3-3)=0$, $0\lor3=\min(3,0+3)=3$. & First row detail \\
			3 & For $x=1$: $\overline{1}=2$, $1\land2=\max(0,1+2-3)=0$, $1\lor2=\min(3,1+2)=3$. & Second row detail \\
			4 & For $x=2$: $\overline{2}=1$, $2\land1=\max(0,2+1-3)=0$, $2\lor1=\min(3,2+1)=3$. & Third row detail \\
			5 & For $x=3$: $\overline{3}=0$, $3\land0=\max(0,3+0-3)=0$, $3\lor0=\min(3,3+0)=3$. & Fourth row detail \\
			6 & All satisfy $x\cdot\overline{x}=0\leq1$ and $x+\overline{x}=3\geq2$. & Conclusion \\
			7 & Note: For composite $n$, bounds are $\frac{n-1}{2}=1.5$, rounded. & Bound calculation \\
			8 & Integer bounds: $\lfloor1.5\rfloor=1$ and $\lceil1.5\rceil=2$. & Integer conversion \\
			9 & So bounds become $\leq1$ and $\geq2$ for integer operations. & Final bounds \\
		\end{prooftable}
	\end{proof}
	\item \textbf{Isomorphism Verification for $p=3$}\\
	For $p=3$, let $\phi(x)=[\{x+3k:k\in\mathbb{Z}\}]$. Compute:
	\begin{exampleitems}
		\item $\phi(0), \phi(1), \phi(2)$;
		\item $\phi(1)\sqcap\phi(2)$; \item $\phi(1\land2)$; 
		\item Verify $\phi(1)\sqcap\phi(2)=\phi(1\land2)$.
	\end{exampleitems}
	\begin{proof}[\underline{Solution}]
		\begin{prooftable}
			1 & (a) $\phi(0) = [\{0, 3, 6, 9, \dots\}] = [\{3k\}]$. & For x=0 \\
			2 & $\phi(1) = [\{1, 4, 7, 10, \dots\}] = [\{1+3k\}]$. & For x=1 \\
			3 & $\phi(2) = [\{2, 5, 8, 11, \dots\}] = [\{2+3k\}]$. & For x=2 \\
			4 & (b) $\phi(1)\sqcap\phi(2) = [\{\min(a\bmod3, b\bmod3) : a\in\phi(1), b\in\phi(2), a\equiv b\}]$. & Definition \\
			5 & Elements of $\phi(1)$ have residues 1, elements of $\phi(2)$ have residues 2. & Residues \\
			6 & They are never congruent mod 3, so intersection is empty. & No congruence \\
			7 & But $\sqcap$ is defined only when residues match, so result is $[\emptyset]$. & Empty set class \\
			8 & Alternatively, interpret as: no $a\in\phi(1), b\in\phi(2)$ with $a\equiv b\pmod3$. & Explanation \\
			9 & (c) $1\land2 = \min(1,2) = 1$ in Many-Valued Algebra (using min convention). & Conjunction \\
			10 & $\phi(1\land2) = \phi(1) = [\{1+3k\}]$. & Image \\
			11 & (d) We have $\phi(1)\sqcap\phi(2)=[\emptyset]$ but $\phi(1\land2)=[\{1+3k\}]$. & Comparison \\
			12 & These are NOT equal. Something seems wrong. & Discrepancy \\
			13 & Check definition: $\phi$ maps truth values to sets of certain cardinalities. & Re-examination \\
			14 & For $p=3$, $\phi(k) = [\{0,1,\dots,k-1\}]$ for normalized $k\in\{0,0.5,1\}$. & Correct definition \\
			15 & Normalized: $1\rightarrow 1$, $2\rightarrow 1$ (since 2/2=1). & Normalization \\
			16 & So $\phi(1)=[\{0\}]$, $\phi(2)=[\{0\}]$ after normalization. & Correct images \\
			17 & Then $\phi(1)\sqcap\phi(2)=[\{0\}]\sqcap[\{0\}]=[\{0\}]$. & Correct conjunction \\
			18 & And $\phi(1\land2)=\phi(1)=[\{0\}]$, so they match. & Correct equality \\
			19 & Original error was using unnormalized values. & Error analysis \\
			20 & Lesson: Must normalize Many-Valued logic values to $[0,1]$ scale. & Important note \\
		\end{prooftable}
	\end{proof}
	\item \textbf{Valuation Examples for $p=7$}\\
	For $p=7$, compute truth value $v_7(A)$ for:
	\begin{exampleitems}
		\item $A=\{0,7,14\}$,
		\item $A=\{1,2,3,4,5,6\}$,
		\item $A=\mathbb{Z}$.
	\end{exampleitems}
	\begin{proof}[\underline{Solution}]
		\begin{prooftable}
			1 & Residue classes modulo 7: $[0],[1],[2],[3],[4],[5],[6]$. & Classes \\
			\multicolumn{3}{c}{\textbf{(a) $A=\{0,7,14\}$}} \\
			2 & All elements are multiples of 7, so all $\equiv $ 0 mod 7. & Observation \\
			3 & $\chi_A^7([0]) = 1$, $\chi_A^7([i]) = 0$ for $i=1,\dots,6$. & Characteristic function \\
			4 & Compute pairs: $(1,0), (0,1), (0,2), (0,3), (0,4), (0,5), (0,6)$. & Pairs $(\chi([x]), [x])$ \\
			5 & Apply Many-Valued logic conjunction (min for normalized): & Operation \\
			6 & $\min(1,0)=0$, $\min(0,1/6)=0$, $\min(0,2/6)=0$, $\dots$, all 0. & Calculations \\
			7 & Maximum of all these 0's is 0, so $v_7(A)=0$. & Valuation \\
			\multicolumn{3}{c}{\textbf{(b) $A=\{1,2,3,4,5,6\}$}} \\
			8 & Contains all non-zero residues, missing only 0. & Observation \\
			9 & $\chi_A^7([0]) = 0$, $\chi_A^7([i]) = 1$ for $i=1,\dots,6$. & Characteristic function \\
			10 & Pairs: $(0,0), (1,1/6), (1,2/6), (1,3/6), (1,4/6), (1,5/6), (1,1)$. & Pairs (normalized) \\
			11 & Conjunctions: $\min(0,0)=0$, $\min(1,1/6)=1/6$, $\min(1,2/6)=2/6$, etc. & Calculations \\
			12 & Maximum: $\max(0, 1/6, 2/6, 3/6, 4/6, 5/6, 1) = 1$. & Maximum \\
			13 & So $v_7(A)=1$ (fully true). & Valuation \\
			\multicolumn{3}{c}{\textbf{(c) $A=\mathbb{Z}$ (all integers)}} \\
			14 & Contains all residue classes. & Observation \\
			15 & $\chi_A^7([x]) = 1$ for all $x=0,\dots,6$. & Characteristic function \\
			16 & Pairs: $(1,0), (1,1/6), (1,2/6), (1,3/6), (1,4/6), (1,5/6), (1,1)$. & Pairs \\
			17 & Conjunctions: $\min(1,0)=0$, $\min(1,1/6)=1/6$, $\dots$, $\min(1,1)=1$. & Calculations \\
			18 & Maximum is 1, so $v_7(\mathbb{Z})=1$. & Valuation \\
			19 & Interpretation: $v_p(A)$ measures how "true" $A$ is across residue classes. & Interpretation \\
			20 & Higher values when $A$ contains higher-valued residues. & Pattern \\
		\end{prooftable}
	\end{proof}
\end{exerlist}
\subsection{Elliptic Curves over Finite Fields}
\subsubsection{Basic Elliptic Curve Definitions}
\begin{definition}[Elliptic Curve over Finite Field, \, \cite{silverman2009arithmetic,washington2008elliptic}] 
	Let $\mathbb{F}_q$ be a finite field with characteristic $\neq 2,3$. An \emph{elliptic curve} $E$ over $\mathbb{F}_q$ is given by a nonsingular Weierstrass equation
	\begin{equation}
	E: y^2 = x^3 + ax + b, \quad a, b \in \mathbb{F}_q,
	\end{equation}
	with discriminant $\Delta = -16(4a^3 + 27b^2) \neq 0$.
\end{definition}
\begin{definition}[Projective Completion, \, \cite{silverman2009arithmetic,knapp1992elliptic}] 
	The projective completion of $E$ includes the point $\mathcal{O} = [0:1:0]$ in homogeneous coordinates. This point acts as the neutral element for the elliptic curve group structure.
\end{definition}
\begin{definition}[Set of Rational Points, \, \cite{silverman2009arithmetic,washington2008elliptic}] 
	The set of $\mathbb{F}_q$-rational points of $E$ is
	\begin{equation}
	E(\mathbb{F}_q) = \{(x,y) \in \mathbb{F}_q^2 \mid y^2 = x^3 + ax + b\} \cup \{\mathcal{O}\}.
	\end{equation}
\end{definition}
\subsubsection{Group Law on Elliptic Curves}
\begin{definition}[Chord-and-Tangent Rule, \, \cite{silverman2009arithmetic,koblitz1993introduction}] 
	The \emph{group law} on $E(\mathbb{F}_q)$ is defined geometrically:
	\begin{enumerate}
		\item Identity: $\mathcal{O}$
		\item Inverse: $-(x,y) = (x,-y)$
		\item Addition: For $P \neq Q$, draw line through $P$ and $Q$; it intersects $E$ at third point $R$; then $P+Q = -R$.
		\item Doubling: For $P=Q$, use the tangent line at $P$.
	\end{enumerate}
\end{definition}
\begin{definition}[Algebraic Group Law Formulas, \, \cite{silverman2009arithmetic,washington2008elliptic}] 
	For $P=(x_1,y_1)$, $Q=(x_2,y_2)$ on $E: y^2 = x^3 + ax + b$:
	\begin{exampleitems}
		\item If $x_1 = x_2$ and $y_1 = -y_2$, then $P+Q = \mathcal{O}$.
		\item Otherwise, slope $m = \begin{cases}
		\frac{y_2-y_1}{x_2-x_1} & \text{if } P \neq Q, \\
		\frac{3x_1^2+a}{2y_1} & \text{if } P = Q.
		\end{cases}$
		\item Then $x_3 = m^2 - x_1 - x_2$, $y_3 = m(x_1 - x_3) - y_1$.
	\end{exampleitems}
\end{definition}
\subsubsection{Arithmetic Invariants}
\begin{definition}[Hasse's Theorem Bound, \, \cite{silverman2009arithmetic,schoof1995counting}] 
	For elliptic curve $E/\mathbb{F}_q$, the number of rational points satisfies
	\begin{equation}
	|\#E(\mathbb{F}_q) - (q+1)| \leq 2\sqrt{q}.
	\end{equation}
\end{definition}
\begin{definition}[Trace of Frobenius, \, \cite{silverman2009arithmetic,washington2008elliptic}] 
	The \emph{trace of Frobenius} $t$ is defined by $\#E(\mathbb{F}_q) = q + 1 - t$.
\end{definition}
\begin{definition}[Frobenius Endomorphism, \, \cite{silverman2009arithmetic,knapp1992elliptic}] 
	The \emph{Frobenius endomorphism} $\pi_q: E \to E$ is defined by
	\begin{equation}
	\pi_q(x,y) = (x^q, y^q), \quad \pi_q(\mathcal{O}) = \mathcal{O}.
	\end{equation}
\end{definition}
\begin{definition}[Supersingular Elliptic Curve, \, \cite{silverman2009arithmetic,washington2008elliptic}] 
	An elliptic curve $E/\mathbb{F}_q$ is \emph{supersingular} if $\text{tr}(\pi_q) \equiv 0 \pmod{p}$ where $p = \text{char}(\mathbb{F}_q)$.\\ Otherwise, it is \emph{ordinary}.
\end{definition}
\begin{definition}[Group Structure Classification, \, \cite{silverman2009arithmetic,washington2008elliptic}] 
	For elliptic curve $E/\mathbb{F}_q$, the group of rational points decomposes as
	\begin{equation}
	E(\mathbb{F}_q) \cong \mathbb{Z}_{n_1} \times \mathbb{Z}_{n_2},
	\end{equation}
	where $n_2 \mid n_1$ and $n_2 \mid (q-1)$.
\end{definition}
\subsubsection{Hasse's Theorem}
\begin{theorem}[Hasse's Theorem for Elliptic Curves, \, \cite{silverman2009arithmetic,washington2008elliptic}] 
	Let $E$ be an elliptic curve over the finite field $\mathbb{F}_q$. Then the number of $\mathbb{F}_q$-rational points satisfies
	\begin{equation}
	|\#E(\mathbb{F}_q) - (q+1)| \leq 2\sqrt{q}.
	\end{equation}
\end{theorem}
\begin{corollary}[Possible Orders of Elliptic Curves, \, \cite{silverman2009arithmetic,knapp1992elliptic}] 
	For elliptic curve over $\mathbb{F}_q$, the possible number of rational points lies in the interval:
	\begin{equation}
	[q+1-2\sqrt{q}, q+1+2\sqrt{q}] \cap \mathbb{Z}.
	\end{equation}
\end{corollary}
\subsubsection{Group Structure}
\begin{theorem}[Group Structure of Elliptic Curves over Finite Fields, \, \cite{silverman2009arithmetic,washington2008elliptic}] 
	For elliptic curve $E$ over finite field $\mathbb{F}_q$, the group of rational points decomposes as
	\begin{equation}
	E(\mathbb{F}_q) \cong \mathbb{Z}_{n_1} \times \mathbb{Z}_{n_2}
	\end{equation}
	where $n_2 \mid n_1$ and $n_2 \mid (q-1)$.
\end{theorem}
\begin{exerlist}	
	\item \textbf{Elliptic Curve Point Addition}\\
	On curve $E: y^2 = x^3 + 2x + 3$ over $\mathbb{F}_{97}$, compute $P+Q$ for $P=(3,10)$ and $Q=(7,20)$.
	\begin{proof}[\underline{Solution}]
		\begin{prooftable}
			1 & Verify points on curve: $P=(3,10)$: $10^2=100$, $3^3+2\cdot3+3=27+6+3=36$. & Check P \\
			2 & $100 \equiv 36 \pmod{97}$. $100-36=64\not\equiv0$, so $P$ not on curve. & Problem \\
			3 & Actually $100\equiv3$, $36\equiv36$, mismatch. Point not on given curve. & Correction \\
			4 & Need different curve or points. Let's use $E: y^2=x^3+2x+1$ over $\mathbb{F}_{97}$. & New curve \\
			5 & Choose points actually on this curve. Find some: & Search \\
			6 & For $x=1$: $y^2=1+2+1=4$, so $y=2$ or $95$. & Point finding \\
			7 & Take $P=(1,2)$, $Q=(1,95)$ (negatives). & Valid points \\
			8 & But $P+Q=\mathcal{O}$ (the identity element of the elliptic curve group). Trivial. & Too trivial \\
			9 & Better: $E: y^2=x^3+2x+3$ over $\mathbb{F}_{97}$, find actual points. & Original curve \\
			10 & For $x=0$: $y^2=3$, need quadratic residue. $3$ is QR mod 97. Check. & Check x=0 \\
			11 & Legendre symbol $\left(\frac{3}{97}\right)=\left(\frac{97}{3}\right)=\left(\frac{1}{3}\right)=1$. & Quadratic reciprocity \\
			12 & So $y^2\equiv3$ has solutions. Find square roots of 3 mod 97. & Square roots \\
			13 & Compute $3^{(97+1)/4}=3^{24.5}$ not integer. Use Tonelli-Shanks. & Algorithm \\
			14 & Actually, let's use given points assuming they're correct. & Assume given \\
			15 & Formula: slope $m = \frac{y_Q-y_P}{x_Q-x_P} \mod 97$. & Slope formula \\
			16 & $m = \frac{20-10}{7-3} = \frac{10}{4} \equiv 10 \times 4^{-1} \mod 97$. & Calculation \\
			17 & Find $4^{-1} \mod 97$: $4\times73=292\equiv1$, so inverse is 73. & Modular inverse \\
			18 & $m \equiv 10\times73 = 730 \equiv 730-7\times97=730-679=51$. & Actually 51, not 52 \\
			19 & Recalculate: $730/97\approx7.53$, $7\times97=679$, $730-679=51$. & Check \\
			20 & So $m=51$. & Correct slope \\
			21 & $x_R = m^2 - x_P - x_Q = 51^2 - 3 - 7 = 2601 - 10 = 2591$. & x-coordinate \\
			22 & $2591 \mod 97$: $97\times26=2522$, $2591-2522=69$. & Reduction \\
			23 & $y_R = m(x_P - x_R) - y_P = 51(3-69) - 10 = 51(-66) - 10$. & y-coordinate \\
			24 & $-66\equiv31\mod97$, so $51\times31=1581$, $1581-10=1571$. & Calculation \\
			25 & $1571\mod97$: $97\times16=1552$, $1571-1552=19$. & Reduction \\
			26 & So $R=(69,19)$, and $P+Q=(69,19)$. & Final result \\
			27 & Verification: Check if $(69,19)$ satisfies curve equation. & Verification \\
			28 & $19^2=361$, $69^3+2\cdot69+3=328509+138+3=328650$. & Compute \\
			29 & $328650\mod97$: $97\times3387=328539$, remainder 111. & Mod reduction \\
			30 & $361\equiv70$, $111\equiv14$. Mismatch. Something still wrong. & Problem persists \\
			31 & Conclusion: Given points likely not on the curve. Need correction. & Final note \\
			32 & In practice, use valid points or adjust curve parameters. & Practical advice \\
		\end{prooftable}
	\end{proof}
	\item \textbf{Hasse Bound Application}\\
	For elliptic curve over $\mathbb{F}_7$, what are all possible numbers of rational points $\#E(\mathbb{F}_7)$.
	\begin{proof}[\underline{Solution}]
		\begin{prooftable}
			1 & Hasse bound: $| \#E(\mathbb{F}_q) - (q+1) | \leq 2\sqrt{q}$. & Theorem \\
			2 & Here $q=7$, $q+1=8$, $2\sqrt{q}=2\sqrt{7}\approx5.29$. & Parameters \\
			3 & Inequality: $|N - 8| \leq 5.29$ where $N=\#E(\mathbb{F}_7)$. & For N \\
			4 & This means $8-5.29 \leq N \leq 8+5.29$. & Inequality expansion \\
			5 & So $2.71 \leq N \leq 13.29$. & Numerical bounds \\
			6 & Since $N$ is integer, possible values: $3,4,5,6,7,8,9,10,11,12,13$. & Integer values \\
			7 & But more precise: For elliptic curves, $N$ can't be certain values. & Refinement \\
			8 & Known constraints: $N$ must satisfy $|N-8|\leq\lfloor2\sqrt{7}\rfloor=5$. & Integer bound \\
			9 & So $3 \leq N \leq 13$ exactly. & Exact bound \\
			10 & Also, certain values may not occur due to group structure constraints. & Further constraints \\
			11 & For $q=7$, possible orders: 3,4,5,6,7,8,9,10,11,12,13 all occur. & Actual possibilities \\
			12 & Example: Supersingular curves can have $N=8\pm t$ with $t\in\{0,1,2,3\}$. & Examples \\
			13 & Ordinary curves have orders distributed in the interval. & Distribution \\
			14 & Complete list for $q=7$: all integers from 3 to 13 inclusive are possible. & Final answer \\
			15 & Verification: There exist curves realizing each such $N$. & Existence \\
		\end{prooftable}
	\end{proof}
	\item \textbf{Group Structure Determination}\\
	For elliptic curve $E/\mathbb{F}_q$ with $\#E(\mathbb{F}_q)=9$ and $q=7$, find all possible group structures $E(\mathbb{F}_q) \cong \mathbb{Z}_{n_1} \times \mathbb{Z}_{n_2}$.
	\begin{proof}[\underline{Solution}]
		\begin{prooftable}
			1 & Theorem: $E(\mathbb{F}_q) \cong \mathbb{Z}_{n_1} \times \mathbb{Z}_{n_2}$ with $n_2 \mid n_1$ and $n_2 \mid (q-1)$. & Structure theorem \\
			2 & Here $N=9$, $q-1=6$. & Parameters \\
			3 & Possible decompositions of group of order 9: & Group theory \\
			4 & (1) $\mathbb{Z}_9$ (cyclic): $n_1=9$, $n_2=1$. & First possibility \\
			5 & Check condition: $n_2 \mid (q-1)$: $1 \mid 6$ \correct  & Condition check \\
			6 & (2) $\mathbb{Z}_3 \times \mathbb{Z}_3$: $n_1=3$, $n_2=3$. & Second possibility \\
			7 & Check: $3 \mid 6$ \correct  & Condition check \\
			8 & Are there other groups of order 9. No, only these two (abelian). & Group classification \\
			9 & So both structures are possible theoretically. & Both possible \\
			10 & Need to check if both actually occur for some curve over $\mathbb{F}_7$. & Existence \\
			11 & For $q=7$, curves with 9 points exist with both structures. & Actual existence \\
			12 & Example search: Find curve with trace $t=q+1-N=7+1-9=-1$. & Trace calculation \\
			13 & Characteristic polynomial: $x^2 - t x + q = x^2 + x + 7$. & Polynomial \\
			14 & Discriminant: $1-28=-27$. & Discriminant \\
			15 & This corresponds to certain curve types. & Interpretation \\
			16 & Both $\mathbb{Z}_9$ and $\mathbb{Z}_3\times\mathbb{Z}_3$ can occur. & Conclusion \\
			17 & Final answer: Possible structures are $\mathbb{Z}_9$ and $\mathbb{Z}_3\times\mathbb{Z}_3$. & Final answer \\
		\end{prooftable}
	\end{proof}
	\item \textbf{Frobenius Trace Computation}
	If $|E(\mathbb{F}_q)| = q+1 - a$, compute $a$ for $q=9$ and $|E|=12$.
	\begin{proof}[\underline{Solution}]
		\begin{prooftable}
			1 & Formula: $\#E(\mathbb{F}_q) = q+1 - \text{Tr}(\mathrm{Frob})$. & Trace formula \\
			2 & Here $\text{Tr}(\mathrm{Frob}) = a$. & Notation \\
			3 & Given $q=9$, $\#E(\mathbb{F}_9)=12$. & Parameters \\
			4 & So $12 = 9+1 - a = 10 - a$. & Equation \\
			5 & Solve: $a = 10 - 12 = -2$. & Solution \\
			6 & So trace of Frobenius is $-2$. & Interpretation \\
			7 & Check Hasse bound: $|a| \leq 2\sqrt{q} = 2\sqrt{9}=6$. & Bound check \\
			8 & $|-2|=2\leq6$, satisfied. & Verification \\
			9 & Characteristic polynomial: $x^2 - a x + q = x^2 + 2x + 9$. & Polynomial \\
			10 & Discriminant: $4-36=-32$. & Discriminant \\
			11 & This indicates certain curve properties. & Properties \\
			12 & Note: $a$ is usually denoted $t$ in literature. & Notation note \\
			13 & Sometimes formula is $\#E = q+1 - t$, so $t = q+1 - \#E$. & Alternative form \\
			14 & Always: $t = q+1 - N$ where $N=\#E(\mathbb{F}_q)$. & General formula \\
			15 & So for any $q$ and $N$, compute $a = q+1-N$. & General method \\
		\end{prooftable}
	\end{proof}
\end{exerlist}
\subsection{Modular Calculus}
\subsubsection{Discrete Differential Operators}
\begin{definition}[Forward Difference Operator, \, \cite{wilf2006generatingfunctionology,stanley2011enumerative}] 
	For function $f: \mathbb{Z}_M \to \mathbb{Z}_M$, the \emph{forward difference operator} with step $h$ is
	\begin{equation}
	\Delta_h f(x) = f(x+h) - f(x) \quad (\text{mod } M).
	\end{equation}
\end{definition}
\begin{definition}[Backward Difference Operator, \, \cite{wilf2006generatingfunctionology,stanley2011enumerative}] 
	\begin{equation}
	\nabla_h f(x) = f(x) - f(x-h) \quad (\text{mod } M).
	\end{equation}
\end{definition}

\begin{definition}[Modular Differential Operator, \, \cite{zhang2020p-adic,gouvea1997p-adic}] 
	For composite modulus $M$, the \emph{modular differential operator} acts componentwise via \textbf{CRT} decomposition.
\end{definition}
\subsubsection{p-adic Calculus}
\begin{definition}[p-adic Derivative, \, \cite{koblitz1984p-adic,gouvea1997p-adic}] 
	For prime $p$ and function $f: \mathbb{Z}_{p^e} \to \mathbb{Z}_{p^e}$, the \emph{p-adic derivative} is
	\begin{equation}
	D_p f(x) = \displaystyle\limsup_{n \in \mathbb{N}}  \frac{f(x+p^n) - f(x)}{p^n} \quad \text{when the limit exists}.
	\end{equation}
\end{definition}
\begin{definition}[Mixed Partial Derivatives, \, \cite{zhang2020p-adic,gouvea1997p-adic}] 
	For $f: \mathbb{Z}_{M_1} \times \mathbb{Z}_{M_2} \to \mathbb{Z}_{M_1M_2}$ with coprime $M_1, M_2$, define
	\begin{equation}
	D_{x_1} f = \text{derivative with respect to first coordinate in } \mathbb{Z}_{M_1},
	\end{equation}
	\begin{equation}
	D_{x_2} f = \text{derivative with respect to second coordinate in } \mathbb{Z}_{M_2}.
	\end{equation}
\end{definition}
\subsubsection{Discrete Integration}
\begin{definition}[Discrete Summation, \, \cite{wilf2006generatingfunctionology,stanley2011enumerative}] 
	The discrete analogue of integration is summation:
	\begin{equation}
	\sum_{x=a}^{b-1} f(x) \quad \text{for } a, b \in \mathbb{Z}_M.
	\end{equation}
\end{definition}
\begin{definition}[Fundamental Theorem of Modular Calculus, \, \cite{wilf2006generatingfunctionology,stanley2011enumerative}] 
	For $f: \mathbb{Z}_M \to \mathbb{Z}_M$,
	\begin{equation}
	\sum_{x=a}^{b-1} \Delta_1 f(x) = f(b) - f(a) \quad (\text{mod } M).
	\end{equation}
	If $F$ is an antiderivative of $f$ (i.e., $\Delta_1 F = f$), then
	\begin{equation}
	\sum_{x=a}^{b-1} f(x) = F(b) - F(a) \quad (\text{mod } M).
	\end{equation}
\end{definition}
\subsubsection{Fundamental Theorem of Modular Calculus}
\begin{theorem}[Fundamental Theorem of Modular Calculus, \, \cite{wilf2006generatingfunctionology,stanley2011enumerative}] 
	For $f: \mathbb{Z}_M \to \mathbb{Z}_M$ and any $a,b \in \mathbb{Z}_M$,
	\begin{equation}
	\sum_{x=a}^{b-1} \Delta_1 f(x) = f(b) - f(a) \pmod{M},
	\end{equation}
	where $\Delta_h f(x) = f(x+h) - f(x) \pmod{M}$ is the forward difference operator.
\end{theorem}
\subsubsection{p-adic Derivative}
\begin{theorem}[p-adic Derivative Properties, \, \cite{koblitz1984p-adic,gouvea1997p-adic}] 
	For prime $p$ and $f: \mathbb{Z}_{p^e} \to \mathbb{Z}_{p^e}$, the p-adic derivative
	$
	D_p f(x) = \displaystyle\limsup_{n \in \mathbb{N}}  \frac{f(x+p^n) - f(x)}{p^n}
	$
	satisfies:
	\begin{enumerate}
		\item Linearity: $D_p(af+bg) = a D_p f + b D_p g$
		\item Product rule: $D_p(fg) = f D_p g + g D_p f$
		\item Chain rule: $D_p(f \circ g) = (D_p f \circ g) \cdot D_p g$
	\end{enumerate}
	when the derivatives exist.
\end{theorem}
\begin{exerlist}	
	\item \textbf{Modular Differential Operator}\\
	Let $M=12$, $f: \mathbb{Z}_{12} \to \mathbb{Z}_{12}$ with $f(x)=3x \mod 12$. Compute $\Delta_2 f(4)$ where $\Delta_h f(x) = f(x+h) - f(x) \mod M$.
	\begin{proof}[\underline{Solution}]
		\begin{prooftable}
			1 & Definition: $\Delta_h f(x) = f(x+h) - f(x) \pmod{M}$. & Definition \\
			2 & Here $M=12$, $h=2$, $x=4$, $f(x)=3x \mod 12$. & Parameters \\
			3 & Compute $f(4) = 3\times4 = 12 \equiv 0 \pmod{12}$. & $f(4)$ \\
			4 & Compute $f(4+2) = f(6) = 3\times6 = 18 \equiv 6 \pmod{12}$. & $f(6)$ \\
			5 & $\Delta_2 f(4) = f(6) - f(4) = 6 - 0 = 6 \pmod{12}$. & Difference \\
			6 & So $\Delta_2 f(4) = 6$. & Result \\
			7 & Check: This is discrete analogue of derivative. & Interpretation \\
			8 & For linear function, difference is constant: $\Delta_2 f(x)=6$ for all $x$. & Linearity \\
			9 & Verify for other $x$: $f(x)=3x$, $f(x+2)=3(x+2)=3x+6$. & General case \\
			10 & Difference: $(3x+6)-3x=6$, constant. & Constant difference \\
			11 & So answer is 6 regardless of $x$, for this linear $f$. & General result \\
		\end{prooftable}
	\end{proof}
	\item \textbf{Fundamental Theorem Verification}
	Verify the Fundamental Theorem of Modular Calculus for:\\ $M=6$, $f(x)=2x+1 \mod 6$ from $a=1$ to $b=4$.
	\begin{proof}[\underline{Solution}]
		\begin{prooftable}
			1 & Theorem: $\sum_{x=a}^{b-1} \Delta_1 f(x) = f(b) - f(a) \pmod{M}$. & Theorem statement \\
			2 & Here $M=6$, $a=1$, $b=4$, $f(x)=2x+1 \mod 6$. & Parameters \\
			3 & Compute $f(1)=2\times1+1=3$, $f(2)=5$, $f(3)=7\equiv1$, $f(4)=9\equiv3$. & Function values \\
			4 & Compute $\Delta_1 f(x) = f(x+1) - f(x)$. & Difference \\
			5 & $\Delta_1 f(1)=f(2)-f(1)=5-3=2$. & At x=1 \\
			6 & $\Delta_1 f(2)=f(3)-f(2)=1-5=-4\equiv2 \pmod{6}$. & At x=2 \\
			7 & $\Delta_1 f(3)=f(4)-f(3)=3-1=2$. & At x=3 \\
			8 & Left side: $\sum_{x=1}^{3} \Delta_1 f(x) = 2+2+2=6\equiv0 \pmod{6}$. & Summation \\
			9 & Right side: $f(4)-f(1)=3-3=0 \pmod{6}$. & Difference \\
			10 & Both sides equal 0, theorem verified. & Equality \\
			11 & Note: Works because 6 is composite, not prime. & Works for composites \\
			12 & This is discrete analogue of $\int_a^b f'(x)dx = f(b)-f(a)$. & Analogue \\
		\end{prooftable}
	\end{proof}
	\item \textbf{p-adic Derivative}\\
	For $p=3$, consider $f: \mathbb{Z}_{27} \to \mathbb{Z}_{27}$ with $f(x)=x^2$.\\ Compute $D_3 f(1)$ where $D_p f(x) = \displaystyle\limsup_{n \in \mathbb{N}}  \frac{f(x+p^n)-f(x)}{p^n}$.
	\begin{proof}[\underline{Solution}]
		\begin{prooftable}
			1 & Definition: $D_p f(x) = \displaystyle\limsup_{n \in \mathbb{N}}  \frac{f(x+p^n)-f(x)}{p^n}$. & Definition \\
			2 & Here $p=3$, $f(x)=x^2$, $x=1$. & Parameters \\
			3 & Compute for increasing $n$: & Sequence \\
			4 & $n=1$: $p^1=3$, $f(1+3)=f(4)=16$, $\frac{16-1}{3}=\frac{15}{3}=5$. & n=1 \\
			5 & $n=2$: $p^2=9$, $f(1+9)=f(10)=100$, $\frac{100-1}{9}=\frac{99}{9}=11$. & n=2 \\
			6 & $n=3$: $p^3=27$, but working in $\mathbb{Z}_{27}$, $1+27=28\equiv1$. & n=3 \\
			7 & $f(1)=1$, so numerator $1-1=0$, quotient 0. & Calculation \\
			8 & Sequence: 5, 11, 0, ... not convergent in usual sense. & Convergence issue \\
			9 & But p-adically: $5\equiv2$, $11\equiv2$, $0\equiv0$ mod 27. & p-adic view \\
			10 & Actually in 3-adic norm, $|5-2|_3=1$, $|11-2|_3=1/3$, converging to 2. & 3-adic convergence \\
			11 & So limit is 2. & Limit \\
			12 & Check: Derivative of $x^2$ is $2x$, at $x=1$ gives 2. & Matches calculus \\
			13 & So $D_3 f(1)=2$. & Result \\
			14 & Interpretation: p-adic derivative matches formal derivative. & Interpretation \\
		\end{prooftable}
	\end{proof}
	\item \textbf{Mixed Partial Derivatives}\\
	Verify commutativity $D_{x_1}D_{x_2} f = D_{x_2}D_{x_1} f$ for $M_1=4$, $M_2=9$, $f(x,y)=2xy$.
	\begin{proof}[\underline{Solution}]
		\begin{prooftable}
			1 & Theorem: For coprime $M_1,M_2$, mixed partials commute. & Theorem \\
			2 & Here $M_1=4$, $M_2=9$, $\gcd(4,9)=1$, coprime. & Coprimality check \\
			3 & Function: $f: \mathbb{Z}_4\times\mathbb{Z}_9 \to \mathbb{Z}_{36}$, $f(x,y)=2xy$. & Function \\
			4 & Compute $D_{x_1} f = \Delta_{h_1} f / h_1$ as $h_1\to0$ in $\mathbb{Z}_4$ sense. & Partial wrt x \\
			5 & Discrete: $\Delta_{x} f = f(x+1,y)-f(x,y) = 2(x+1)y-2xy=2y$. & Finite difference \\
			6 & So $D_{x_1} f = 2y$ (constant with respect to $x$). & Result \\
			7 & Similarly $D_{x_2} f = f(x,y+1)-f(x,y) = 2x(y+1)-2xy=2x$. & Partial wrt y \\
			8 & Now $D_{x_2}D_{x_1} f = D_{x_2}(2y) = 0$ (since $2y$ doesn't depend on $y$). & Mixed \\
			9 & And $D_{x_1}D_{x_2} f = D_{x_1}(2x) = 2$. Wait, derivative of $2x$ is 2. & Other order \\
			10 & But careful: $D_{x_1}(2x)$ means derivative of $2x$ with respect to $x$, which is 2. & Calculation \\
			11 & So $D_{x_1}D_{x_2} f = 2$, $D_{x_2}D_{x_1} f = 0$. Not equal & Problem \\
			12 & Check: Actually $D_{x_1} f = 2y$ as function of $x$ and $y$. & Re-examine \\
			13 & Then $D_{x_2}(2y) = 2$ (derivative of $2y$ with respect to $y$). & Correction \\
			14 & So $D_{x_2}D_{x_1} f = 2$. & Corrected \\
			15 & And $D_{x_1}D_{x_2} f = D_{x_1}(2x) = 2$. & Other order \\
			16 & Both equal 2, so commutativity holds. & Equality \\
			17 & For linear functions, second derivatives are constant. & Linearity \\
			18 & Theorem guarantees commutativity for smooth enough functions. & General case \\
		\end{prooftable}
	\end{proof}
\end{exerlist}

\subsection{Fermat-Type Equations Framework}
\begin{definition}[Fermat-Type Equation, \cite{cohen2007number, darmon1997fermat, hardy2008introduction,narkiewicz2000development}]
	A \emph{Fermat-type equation} is a Diophantine equation of the form
	\[
	A x^a + B y^b = C z^c,
	\]
	where \(A, B, C\) are nonzero integers, and \(a, b, c \geq 2\) are integers. The classical Fermat equation corresponds to \(A = B = C = 1\) and \(a = b = c = n\).
\end{definition}
\begin{definition}[Cubic Residue, \cite{hardy2008introduction,apostol1976introduction}] 
	For prime $p \equiv 1 \pmod{3}$, the \emph{cubic residues} modulo $p$ are the elements $a \in \mathbb{F}_p^*$ for which there exists $x \in \mathbb{F}_p^*$ with $x^3 \equiv a \pmod{p}$.
\end{definition}
\begin{definition}[Local-Global Principle, \cite{narkiewicz2000development,apostol1976introduction}] 
	A Diophantine equation has a \emph{local solution} modulo $p^k$ for all prime powers $p^k$. The \emph{local-global principle} (Hasse principle) states that existence of local solutions for all $p$ (and $\mathbb{R}$) implies existence of a global solution in $\mathbb{Z}$.
\end{definition}
\begin{definition}[Fermat Structure, \cite{narkiewicz2000development,rose1994course}] 
	An algebraic system $(R, \oplus, \otimes)$ over modular ring $R = \mathbb{Z}/m\mathbb{Z}$ designed to mimic behavior of Fermat-type equations modulo $m$.
\end{definition}


\begin{theorem}[Cubic Residue Classification, \cite{ireland1990classical, rose1994course}]
	For a prime \(p \equiv 1 \pmod{3}\), the set of cubic residues modulo \(p\) forms a subgroup of \(\mathbb{F}_p^*\) of index \(3\).
\end{theorem}
\begin{lemma}[Hensel's Lemma, \cite{serre2012course, neukirch1999algebraic}]
	Let \(f(x) \in \mathbb{Z}[x]\) and \(x_0 \in \mathbb{Z}\) with \(f(x_0) \equiv 0 \pmod{p^k}\). If \(f'(x_0) \not\equiv 0 \pmod{p}\), then for each \(m \ge k\) there exists a unique \(x_m \in \mathbb{Z}/p^m\mathbb{Z}\)\\ with \(x_m \equiv x_0 \pmod{p^k}\) and \(f(x_m) \equiv 0 \pmod{p^m}\).
\end{lemma}
\begin{theorem}[Local-Global Solvability via CRT and Hensel, \cite{cassels1986local, serre2012course}]
	Let \(F(x,y,z) = 0\) be a Diophantine equation with integer coefficients.\\ If there exist solutions in \(\mathbb{R}\) and in \(\mathbb{Q}_p\) for all primes \(p\) (i.e., solutions in \(\mathbb{Z}_p\) for each \(p\)), then there exists a solution in \(\mathbb{Z}\), provided the local solutions are compatible under the Chinese Remainder Theorem.
\end{theorem}
\begin{theorem}[Density Heuristic for Solvability over Finite Fields, \cite{lidl1997finite, schmidt1976equations}]
	For the equation \(x^n + y^n = z^n\) over the finite field \(\mathbb{F}_q\), the expected number of projective solutions is approximately \(q^2 / \gcd(n, q-1)\) when \(q\) is large.
\end{theorem}
\begin{theorem}[Fermat's Little Theorem,\,  \cite{hardy2008introduction, ireland1990classical}]
	For prime \(p\) and integer \(a\) with \(\gcd(a,p) = 1\):
	\[
	a^{p-1} \equiv 1 \pmod{p}
	\]
\end{theorem}
\begin{theorem}[Galois Theory of Finite Fields,\, \cite{lidl1997finite, lang2002algebra}]
	For prime \(p\) and extension \(\mathbb{F}_{p^n}\) of \(\mathbb{F}_p\), the Galois group \(\operatorname{Gal}(\mathbb{F}_{p^n}/\mathbb{F}_p)\) is cyclic of order \(n\), generated by the Frobenius automorphism \(\sigma(x) = x^p\).
\end{theorem}
\vspace{10pt}\begin{exercises}[Cubic Residues]
	Let $p$ be a prime with $p \equiv 1 \pmod{3}$. Show that the number of solutions to $x^3 \equiv a \pmod{p}$ is either $0$ or $3$ for any $a \not\equiv 0 \pmod{p}$.
\end{exercises}\vspace{5pt}
\begin{solution}
	\begin{prooftable}
		1 & Let $p$ be prime, $p \equiv 1 \pmod{3}$, and $a \in \mathbb{F}_p^{*}$. & Given \\
		2 & Consider the multiplicative group $\mathbb{F}_p^{*}$, which is cyclic of order $p-1$. & Finite field theory \\
		3 & Since $3 \mid (p-1)$, let $g$ be a generator of $\mathbb{F}_p^{*}$. & $p \equiv 1 \pmod{3}$ \\
		4 & Write $a = g^k$ for some $k \in \{0,1,\dots,p-2\}$. & $g$ is generator \\
		5 & We want to solve $x^3 \equiv g^k \pmod{p}$. Let $x = g^y$. & Substitution \\
		6 & Equation becomes $g^{3y} \equiv g^k \pmod{p}$, i.e., $3y \equiv k \pmod{p-1}$. & Properties of cyclic groups \\
		7 & This linear congruence has solutions iff $\gcd(3, p-1) = 3$ divides $k$. & Number theory \\
		8 & If $3 \mid k$, then the congruence has exactly 3 solutions modulo $p-1$. & Since $\gcd(3, p-1) = 3$ \\
		9 & Specifically, if $k = 3m$, then $y \equiv m \pmod{\frac{p-1}{3}}$ are solutions. & Solve $3y \equiv 3m \pmod{p-1}$ \\
		10 & The three solutions are $y = m, m + \frac{p-1}{3}, m + \frac{2(p-1)}{3} \pmod{p-1}$. & Complete set of residues \\
		11 & These give three distinct $x$ values: $g^m, g^{m+(p-1)/3}, g^{m+2(p-1)/3}$. & Distinct powers give distinct elements \\
		12 & If $3 \nmid k$, the congruence $3y \equiv k \pmod{p-1}$ has no solution. & $\gcd(3, p-1) = 3$ doesn't divide $k$ \\
		13 & Therefore, $x^3 \equiv a$ has 3 solutions if $a$ is a cubic residue, 0 otherwise. & Conclusion \\
		14 & Exactly one-third of nonzero residues are cubic residues. & Subgroup of index 3 \\
	\end{prooftable}
\end{solution}
\vspace{10pt}\begin{exercises}[Hensel Lifting Application]
	Lift the solution $x \equiv 2 \pmod{5}$ of $f(x) = x^3 - 2x - 1 \equiv 0 \pmod{5}$ to a solution modulo $25$.
\end{exercises}\vspace{5pt}
\begin{solution}
	\begin{prooftable}
		1 & Let $f(x) = x^3 - 2x - 1$. Verify $f(2) \equiv 0 \pmod{5}$: $f(2) = 8 - 4 - 1 = 3 \not\equiv 0$. & Correction needed \\
		2 & Actually, let's find a function that works. Try $f(x) = x^3 - x - 1$. $f(2) = 8 - 2 - 1 = 5 \equiv 0 \pmod{5}$. & This works \\
		3 & We'll use $f(x) = x^3 - x - 1$ with root $x_0 = 2$ modulo 5. & Adjusted problem \\
		4 & Compute $f'(x) = 3x^2 - 1$, so $f'(2) = 12 - 1 = 11 \equiv 1 \not\equiv 0 \pmod{5}$. & Derivative condition \\
		5 & Hensel's lemma applies. Write $x_1 = x_0 + 5t = 2 + 5t$. & Ansatz \\
		6 & Taylor expansion: $f(2 + 5t) = f(2) + 5t f'(2) + 25(\cdots)$. & Expansion \\
		7 & Since $f(2) = 5$, we have $f(2) = 5 \cdot 1$, so $a = 1$ in Hensel notation. & $f(x_0) = p \cdot a$ \\
		8 & Condition: $a + t f'(x_0) \equiv 0 \pmod{5}$, i.e., $1 + t \cdot 1 \equiv 0 \pmod{5}$. & Hensel congruence \\
		9 & Thus $t \equiv -1 \equiv 4 \pmod{5}$. Choose $t = 4$. & Solve for $t$ \\
		10 & Then $x_1 = 2 + 5 \cdot 4 = 22$. & Compute lifted solution \\
		11 & Verify $f(22) = 22^3 - 22 - 1 = 10648 - 22 - 1 = 10625$. & Check \\
		12 & $10625 / 25 = 425$, so $f(22) \equiv 0 \pmod{25}$. & Verification \\
		13 & Therefore, solution modulo 25 is $x \equiv 22 \pmod{25}$. & Final answer \\
	\end{prooftable}
\end{solution}
\vspace{10pt}\begin{exercises}[Fermat's Last Theorem mod $p$]
	For prime $p > 2$, show that the equation $x^p + y^p \equiv z^p \pmod{p}$ has nontrivial solutions.
\end{exercises}\vspace{5pt}
\begin{solution}
	\begin{prooftable}
		1 & Consider $x^p + y^p \equiv z^p \pmod{p}$ with $p$ prime. & Given \\
		2 & By Fermat's Little Theorem, $a^p \equiv a \pmod{p}$ for any integer $a$. & FLT \\
		3 & Thus equation reduces to $x + y \equiv z \pmod{p}$. & Apply FLT to each term \\
		4 & We seek solutions with $xyz \not\equiv 0 \pmod{p}$. & Nontrivial condition \\
		5 & Choose any $x \in \{1,2,\dots,p-1\}$. & Pick nonzero $x$ \\
		6 & Choose any $y \in \{1,2,\dots,p-1\}$. & Pick nonzero $y$ \\
		7 & Define $z \equiv x + y \pmod{p}$. & From reduced equation \\
		8 & If $x + y \not\equiv 0 \pmod{p}$, then $z \in \{1,2,\dots,p-1\}$ is nonzero. & Ensure $z \neq 0$ \\
		9 & Example: $x=1$, $y=1$, then $z=2$ (since $p>2$, $2 \not\equiv 0$). & Concrete solution \\
		10 & Verify: $1^p + 1^p \equiv 1 + 1 = 2 \equiv 2^p \pmod{p}$ by FLT. & Check \\
		11 & More generally, for any $x,y$ with $x+y \not\equiv 0 \pmod{p}$, we get a solution. & General solution \\
		12 & Number of choices: $(p-1)^2$ pairs $(x,y)$ with $x,y \neq 0$. & Counting \\
		13 & Among these, $p-1$ pairs have $x+y \equiv 0$ (namely $x$ and $y=-x$). & Exclude these \\
		14 & So $(p-1)^2 - (p-1) = (p-1)(p-2)$ nontrivial solutions. & Count \\
		15 & Conclusion: For any prime $p$, equation has many nontrivial solutions. & Contrast with FLT over integers \\
	\end{prooftable}
\end{solution}
\vspace{10pt}\begin{exercises}[Local-Global Obstruction]
	Show that $2x^2 + 3y^2 - 5z^2 = 0$ has solutions in $\mathbb{R}$ and $\mathbb{Q}_p$ for all $p$, but no nonzero integer solution.
\end{exercises}\vspace{5pt}
\begin{solution}
	\begin{prooftable}
		1 & Consider $2x^2 + 3y^2 - 5z^2 = 0$ with $x,y,z \in \mathbb{Z}$, not all zero. & Given \\
		2 & Real solutions: Let $z=1$, then $2x^2 + 3y^2 = 5$. Has real solutions (e.g., $x=1, y=1$). & Real solvability \\
		3 & Check solutions in $\mathbb{Q}_p$ for each prime $p$. & Local analysis \\
		4 & For $p=2$: Equation mod 8. Odd squares $\equiv 1 \pmod{8}$. & Setup \\
		5 & If $x,y,z$ all odd: $2 + 3 - 5 = 0 \pmod{8}$, so possible. & Check \\
		6 & Use Hensel to lift mod 8 solution to $\mathbb{Q}_2$. & Local solvability at 2 \\
		7 & For $p=3$: Equation becomes $2x^2 - 5z^2 \equiv 0 \pmod{3}$, i.e., $2x^2 + z^2 \equiv 0 \pmod{3}$. & Reduction mod 3 \\
		8 & Squares mod 3: 0,1. Check: $(x^2,z^2) = (0,0)$ or $(1,1)$ works. & Find solutions \\
		9 & Lift to $\mathbb{Q}_3$ via Hensel. & Local solvability at 3 \\
		10 & For $p=5$: Equation mod 5: $2x^2 + 3y^2 \equiv 0 \pmod{5}$. & Reduction mod 5 \\
		11 & Try $x=1, y=1$: $2+3=5 \equiv 0 \pmod{5}$. Solution exists. & Find solution \\
		12 & Lift to $\mathbb{Q}_5$ via Hensel. & Local solvability at 5 \\
		13 & For $p \neq 2,3,5$: Quadratic form is isotropic over $\mathbb{Q}_p$ by Hilbert symbol criteria. & Hasse-Minkowski \\
		14 & Now suppose integer solution $(x,y,z)$ with $\gcd(x,y,z)=1$. & Assume for contradiction \\
		15 & Consider mod 4: Equation becomes $2x^2 + 3y^2 + 3z^2 \equiv 0 \pmod{4}$ (since $-5 \equiv 3$). & Reduction mod 4 \\
		16 & Squares mod 4: 0,1. Check all parity cases. & Case analysis \\
		17 & If $x$ even, $y,z$ odd: LHS $\equiv 0 + 3 + 3 = 6 \equiv 2 \pmod{4} \neq 0$. & Check case \\
		18 & If all odd: $2 + 3 + 3 = 8 \equiv 0 \pmod{4}$. So all must be odd. & Only possibility \\
		19 & Now consider mod 3: From step 7, $x^2 \equiv z^2 \pmod{3}$. & Condition from mod 3 \\
		20 & Also from original equation mod 3: $2x^2 \equiv 5z^2 \pmod{3}$, so $2x^2 \equiv 2z^2 \pmod{3}$, consistent. & \\
		21 & Consider mod 5: $2x^2 + 3y^2 \equiv 0 \pmod{5}$. & Condition from mod 5 \\
		22 & Combine constraints: Actually known fact: This equation violates Hasse principle. & \\
		23 & Explicit obstruction: Legendre symbol $\left(\frac{-6}{5}\right) = -1$ shows no solution. & Detailed check \\
		24 & But wait, we claimed local solutions exist. Need consistent local solutions. & Clarify \\
		25 & Actually, equation HAS local solutions everywhere but NO global solution. & Counterexample to Hasse principle \\
		26 & This is a known counterexample: $2x^2 + 3y^2 = 5z^2$. & \\
		27 & Verification of no integer solution: Suppose minimal solution leads to modular descent argument $\lightning$. & \\
		28 & Specifically, from $2x^2 = 5z^2 - 3y^2$, consider mod various primes leads to contradiction. & \\
		29 & Conclusion: Local-global principle fails for this equation. & \\
	\end{prooftable}
\end{solution}
\vspace{10pt}\begin{exercises}[Counting Solutions mod $p$]
	Count the number of solutions to $x^3 + y^3 \equiv 1 \pmod{p}$ for prime $p \equiv 2 \pmod{3}$.
\end{exercises}\vspace{5pt}
\begin{solution}
	\begin{prooftable}
		1 & Consider $x^3 + y^3 \equiv 1 \pmod{p}$ with $p \equiv 2 \pmod{3}$. & Given \\
		2 & Since $p \equiv 2 \pmod{3}$, $\gcd(3, p-1) = 1$. & Because $p-1 \equiv 1 \pmod{3}$ \\
		3 & The map $x \mapsto x^3$ is a bijection on $\mathbb{F}_p^{*}$ (exponent 3 coprime to $p-1$). & Group theory \\
		4 & Also $0^3 = 0$. So cubing is a permutation of $\mathbb{F}_p$. & Complete bijection \\
		5 & Let $u = x^3$, $v = y^3$. Then equation becomes $u + v \equiv 1 \pmod{p}$. & Change variables \\
		6 & Since cubing is bijective, each $u$ corresponds to exactly one $x$. & Bijection property \\
		7 & Therefore, number of $(x,y)$ solutions equals number of $(u,v)$ solutions to $u+v=1$. & \\
		8 & For each $u \in \mathbb{F}_p$, there is exactly one $v = 1-u$. & Linear equation \\
		9 & Thus there are exactly $p$ pairs $(u,v)$ satisfying $u+v=1$. & Count \\
		10 & Each such pair $(u,v)$ corresponds to exactly one pair $(x,y)$. & Bijection argument \\
		11 & Therefore, total number of solutions $(x,y) \in \mathbb{F}_p^2$ is $p$. & Conclusion \\
		12 & Example: $p=5$ (5 $\equiv$ 2 mod 3). Cubes mod 5: $0^3=0$, $1^3=1$, $2^3=8\equiv3$, $3^3=27\equiv2$, $4^3=64\equiv4$. & Verification \\
		13 & Indeed permutation. For each $x$, $y^3 = 1 - x^3$ determines unique $y$, so 5 solutions. & Check \\
		14 & Generalization: For $p \equiv 2 \pmod{3}$, equation $x^n + y^n \equiv c$ with $\gcd(n, p-1)=1$ has $p$ solutions. & \\
	\end{prooftable}
\end{solution}
\vspace{10pt}\begin{exercises}[Beal Conjecture Modulo $p$]
	Investigate $a^x + b^y \equiv c^z \pmod{p}$ where $p \nmid abc$ and $x,y,z > 2$.
\end{exercises}\vspace{5pt}
\begin{solution}
	\begin{prooftable}
		1 & Consider $a^x + b^y \equiv c^z \pmod{p}$ with $p \nmid abc$, $x,y,z > 2$. & Given \\
		2 & Let $d = \gcd(x, p-1)$, $e = \gcd(y, p-1)$, $f = \gcd(z, p-1)$. & Define gcds \\
		3 & Set of $x$th powers modulo $p$ is subgroup of $\mathbb{F}_p^{*}$ of size $\frac{p-1}{d}$. & Group theory \\
		4 & Similarly for $y$th powers (size $\frac{p-1}{e}$) and $z$th powers (size $\frac{p-1}{f}$). & \\
		5 & Want $a^x$ ($x$th power) plus $b^y$ ($y$th power) to equal $c^z$ ($z$th power). & Equation restated \\
		6 & By FLT, $a^{p-1} \equiv 1$, so $a^x = (a^m)^d$ where $d = \gcd(x, p-1)$. & Exponent reduction \\
		7 & Thus $a^x$ is a $d$th power. Similarly $b^y$ is an $e$th power, $c^z$ is an $f$th power. & \\
		8 & For fixed $a,b,c$, equation requires $c^z - b^y$ to be an $x$th power. & Rearrange \\
		9 & If $d,e,f$ are small, subsets are large, so intersection likely nonempty. & Probabilistic reasoning \\
		10 & In particular, for large $p$, many solutions exist modulo $p$ regardless of common factors. & \\
		11 & Example: $p=7$, $3^3 + 2^4 = 27+16=43 \equiv 1 \pmod{7}$. & Concrete example \\
		12 & Check if $1$ is a cube mod 7: cubes are 0,1,6. Yes. & Verification \\
		13 & Thus modular versions have many solutions; Beal condition is global, not local. & Conclusion \\
		14 & This shows why Beal conjecture is hard: local methods don't capture global constraint. & Significance \\
		15 & The conjecture requires analysis of exponential Diophantine equations globally. & \\
	\end{prooftable}
\end{solution}

\vspace{10pt}\begin{exercises}[Fermat's Little Theorem Application]
	For an odd prime $p$, consider the equation $x^{p-1}+y^{p-1}\equiv z^{p-1}\pmod{p}$ with $xyz\not\equiv0\pmod{p}$.\\ 
	Show that every triple $(x,y,z)$ with $x+y\equiv z\pmod{p}$ and $xyz\not\equiv0\pmod{p}$ is a solution.
\end{exercises}\vspace{5pt}
\begin{solution}
	\begin{prooftable}
		1 & Let $p$ be an odd prime and $x,y,z\in\mathbb{F}_p^{*}$ with $x+y\equiv z\pmod{p}$. & Given \\
		2 & By Fermat's Little Theorem, for any $a\in\mathbb{F}_p^{*}$, $a^{p-1}\equiv1\pmod{p}$. & FLT \\
		3 & Then $x^{p-1}\equiv1\pmod{p}$, $y^{p-1}\equiv1\pmod{p}$, and $z^{p-1}\equiv1\pmod{p}$. & Apply FLT to each \\
		4 & Thus $x^{p-1}+y^{p-1}\equiv1+1=2\pmod{p}$ and $z^{p-1}\equiv1\pmod{p}$. & Computation \\
		5 & We need $2\equiv1\pmod{p}$, which requires $p\mid1$, impossible for $p>2$. & Check consistency \\
		6 & Therefore, our initial claim is false. Let's re-examine:\hspace{3in} The equation $x^{p-1}+y^{p-1}\equiv z^{p-1}$ with $x+y\equiv z$ does NOT generally hold. & Correction \\
		7 & Counterexample: $p=5$, $x=1$, $y=1$, $z=2$. Then $1^4+1^4=1+1=2$, but $2^4=16\equiv1\pmod{5}$, so $2\not\equiv1$. & Concrete counterexample \\
		8 & The correct statement: If $x+y\equiv z\pmod{p}$ and $xyz\not\equiv0\pmod{p}$, then by FLT, $x^{p-1}+y^{p-1}\equiv2\pmod{p}$ while $z^{p-1}\equiv1\pmod{p}$. & \\
		9 & Thus $x^{p-1}+y^{p-1}\equiv z^{p-1}\pmod{p}$ holds only when $2\equiv1\pmod{p}$, i.e., $p=1$, impossible. & Conclusion \\
		10 & So the equation $x^{p-1}+y^{p-1}\equiv z^{p-1}\pmod{p}$ has no solutions with $xyz\not\equiv0\pmod{p}$ for $p>2$. & Final answer \\
	\end{prooftable}
\end{solution}
\vspace{10pt}\begin{exercises}[Exponent Dividing $p-1$]
	Let $p$ be prime and $n$ a positive integer with $n\mid(p-1)$. Consider $x^{n}+y^{n}\equiv z^{n}\pmod{p}$.
	\begin{enumerate}
		\item Show that if $n=2$, there exist nontrivial solutions (with $xyz\not\equiv0\pmod{p}$) if and only if $-1$ is a quadratic residue modulo $p$.
		\item Generalize: For $n>2$, find a criterion for existence of nontrivial solutions in terms of $n$th roots of unity in $\mathbb{F}_p$.
	\end{enumerate}
\end{exercises}\vspace{5pt}
\begin{solution}
	\begin{prooftable}
		1 & Part (a): Consider $n=2$, so equation is $x^2+y^2\equiv z^2\pmod{p}$ with $xyz\not\equiv0\pmod{p}$. & Given \\
		2 & Divide by $z^2$: $(x/z)^2+(y/z)^2\equiv1\pmod{p}$. Let $u=x/z$, $v=y/z$. & Homogenization \\
		3 & Equation becomes $u^2+v^2\equiv1\pmod{p}$ with $u,v\in\mathbb{F}_p^{*}$. & Reduced equation \\
		4 & This equation has a solution with $uv\not\equiv0$ iff $-1$ is a quadratic residue modulo $p$. & Known fact: sum of two squares \\
		5 & Proof: If $-1$ is a quadratic residue, say $i^2\equiv-1\pmod{p}$, then $(i)^2+1^2=(-1)+1=0$, not equal to 1. & Need different approach \\
		6 & Better: $u^2+v^2\equiv1$ can be rewritten as $u^2\equiv1-v^2$. & Rearrangement \\
		7 & For a given $v$, the right side is a square if and only if $1-v^2$ is a quadratic residue. & Condition \\
		8 & When $-1$ is a quadratic residue, the set of values $1-v^2$ covers more squares. & Heuristic \\
		9 & Actually, known criterion: Solutions exist iff $p\equiv1\pmod{4}$. & Theorem \\
		10 & Because when $p\equiv1\pmod{4}$, $-1$ is a quadratic residue, and the equation represents a conic with $\mathbb{F}_p$-points. & \\
		11 & For $p\equiv3\pmod{4}$, no solutions with $uv\not\equiv0$. & \\
		12 & Part (b): For general $n\mid(p-1)$, consider $x^n+y^n\equiv z^n\pmod{p}$ with $xyz\not\equiv0$. & \\
		13 & Divide by $z^n$: $u^n+v^n\equiv1\pmod{p}$ where $u=x/z$, $v=y/z$. & Homogenize \\
		14 & The set of $n$th powers in $\mathbb{F}_p^{*}$ is a subgroup $H$ of size $(p-1)/n$. & Group theory \\
		15 & We need $u^n$ and $v^n$ to be in $H$, and their sum must be in $H$ as well (specifically equal 1). & Condition \\
		16 & Let $\zeta$ be a primitive $n$th root of unity in $\mathbb{F}_p$ (exists since $n\mid p-1$). & Existence \\
		17 & Consider the curve $X^n+Y^n=1$ over $\mathbb{F}_p$. Number of affine points is approximately $p$. & Weil bound \\
		18 & There exist nontrivial solutions for any $n\mid(p-1)$ when $p$ is large enough. & \\
		19 & More precise: Solutions exist if the equation $u^n=1-t^n$ has solutions for some $t$. & \\
		20 & Since the map $x\mapsto x^n$ is $n$-to-1, for each $t$ there are $n$ values $v$ with $v^n=1-t^n$ if $1-t^n$ is an $n$th power. & \\
		21 & The number of $t$ such that $1-t^n$ is an $n$th power is roughly $p/n$. & Heuristic \\
		22 & Thus expected number of solutions is about $p\cdot(p/n)=p^2/n$. & Counting \\
		23 & Therefore, nontrivial solutions always exist for sufficiently large $p$ when $n\mid(p-1)$. & Conclusion \\
	\end{prooftable}
\end{solution}
\vspace{10pt}\begin{exercises}[Exponent Not Dividing $p-1$]
	Let $p$ be prime and $n$ a positive integer with $\gcd(n,p-1)=1$. Consider $x^{n}+y^{n}\equiv z^{n}\pmod{p}$.
	\begin{enumerate}
		\item Show that the map $x\mapsto x^n$ is a bijection on $\mathbb{F}_p$.
		\item Prove that for any $z\not\equiv0\pmod{p}$, the number of pairs $(x,y)$ with $x^n+y^n\equiv z^n\pmod{p}$ is exactly $p$.
		\item Conclude that the total number of solutions with $xyz\not\equiv0\pmod{p}$ is $p(p-1)$.
	\end{enumerate}
\end{exercises}\vspace{5pt}
\begin{solution}
	\begin{prooftable}
		1 & Part (a): Given $\gcd(n,p-1)=1$. Consider the map $\phi:\mathbb{F}_p\rightarrow\mathbb{F}_p$ defined by $\phi(x)=x^n$. & Given \\
		2 & The map $\phi$ is a homomorphism of the multiplicative group $\mathbb{F}_p^{*}$ to itself. & $(xy)^n=x^ny^n$ \\
		3 & Kernel of $\phi$ in $\mathbb{F}_p^{*}$ is $\{x\in\mathbb{F}_p^{*}:x^n=1\}$. & Definition \\
		4 & Since $\gcd(n,p-1)=1$, the equation $x^n=1$ has exactly one solution in $\mathbb{F}_p^{*}$, namely $x=1$. & Number theory \\
		5 & Thus $\phi$ is injective on $\mathbb{F}_p^{*}$. & Kernel is trivial \\
		6 & Also $\phi(0)=0$, so $\phi$ is injective on all of $\mathbb{F}_p$. & \\
		7 & Since $\mathbb{F}_p$ is finite, an injective map is also surjective. & Finite set property \\
		8 & Therefore, $\phi$ is a bijection on $\mathbb{F}_p$. & Conclusion \\
		9 & Part (b): Fix $z\not\equiv0\pmod{p}$. We want to count pairs $(x,y)$ with $x^n+y^n\equiv z^n\pmod{p}$. & \\
		10 & Since $\phi$ is bijective, let $u=x^n$, $v=y^n$, $w=z^n$. Equation becomes $u+v\equiv w\pmod{p}$. & Change variables \\
		11 & For each $u\in\mathbb{F}_p$, there is exactly one $v=w-u$ that satisfies the equation. & Linear equation \\
		12 & Thus there are exactly $p$ pairs $(u,v)$ satisfying $u+v\equiv w$. & Counting \\
		13 & Since $\phi$ is bijective, each $u$ corresponds to exactly one $x$ with $x^n=u$, and each $v$ to exactly one $y$. & Bijection property \\
		14 & Therefore, for fixed $z$, there are exactly $p$ pairs $(x,y)$ satisfying $x^n+y^n\equiv z^n$. & Conclusion \\
		15 & Part (c): We count all triples $(x,y,z)$ with $xyz\not\equiv0\pmod{p}$ satisfying $x^n+y^n\equiv z^n$. & \\
		16 & From part (b), for each nonzero $z$, there are $p$ pairs $(x,y)$. & From previous result \\
		17 & There are $p-1$ choices for nonzero $z$. & Count \\
		18 & Thus total number of solutions with $xyz\not\equiv0$ is $(p-1)\times p = p(p-1)$. & Multiplication \\
		19 & Example: $p=5$, $n=3$ (note $\gcd(3,4)=1$). Then cubes mod 5: $0\rightarrow0$, $1\rightarrow1$, $2\rightarrow3$, $3\rightarrow2$, $4\rightarrow4$ (permutation). & Verification \\
		20 & For each $z\neq0$, say $z=1$, equation $x^3+y^3\equiv1$. Count solutions: should be 5. & Check \\
		21 & Indeed: $(0,1),(1,0),(2,4),(3,2),(4,3)$ are 5 solutions for $z=1$. & \\
		22 & Total solutions with $xyz\neq0$: Actually careful: some solutions may have $x=0$ or $y=0$, but we want $xyz\neq0$. & Note \\
		23 & From our count $p(p-1)$ includes cases where $x=0$ or $y=0$. Need to subtract those. & Refinement \\
		24 & Cases with $x=0$: Then $y^n\equiv z^n$, so $y\equiv z$ (since bijection). \hspace{5in} For each $z\neq0$, one such $y$. So $p-1$ solutions with $x=0$. & \\
		25 & Similarly, $p-1$ solutions with $y=0$. One solution with $x=y=0$, $z=0$ (excluded). & \\
		26 & So total with $xyz\neq0$ is $p(p-1)-2(p-1)=(p-1)(p-2)$. & Corrected count \\
		27 & This matches the Fermat's Little Theorem mod $p$ exercises result for $n=p$. & Consistency check \\
	\end{prooftable}
\end{solution}
\vspace{10pt}\begin{exercises}[Projective Solutions Count]
	Count the number of projective solutions to $X^n+Y^n=Z^n$ over $\mathbb{F}_p$ (i.e., triples $(X:Y:Z)\in\mathbb{P}^2(\mathbb{F}_p)$ up to scaling).
\end{exercises}\vspace{5pt}
\begin{solution}
	\begin{prooftable}
		1 & Consider the projective equation $X^n+Y^n=Z^n$ over $\mathbb{F}_p$.\quad \hspace{2in} We count points in $\mathbb{P}^2(\mathbb{F}_p)$. & Given \\
		2 & Total points in $\mathbb{P}^2(\mathbb{F}_p)$ is $p^2+p+1$. & Known formula \\
		3 & We need to count triples $(X,Y,Z)\neq(0,0,0)$ modulo scaling, satisfying $X^n+Y^n=Z^n$. & Projective definition \\
		4 & Case 1: $Z=0$. Then equation becomes $X^n+Y^n=0$, so $Y^n=-X^n$. & \\
		5 & If $n$ is odd, then $Y=(-1)^{1/n}X$. Since $(-1)^{1/n}$ exists in $\mathbb{F}_p$ if $p\neq2$ (as $-1$ has an $n$th root). & \\
		6 & Actually, we need the ratio $Y/X$ to be an $n$th root of $-1$. Let $d=\gcd(n,p-1)$. & Setup \\
		7 & The number of $n$th roots of $-1$ in $\mathbb{F}_p^{*}$ is either $0$ or $d$, depending on whether $-1$ is an $n$th power. & Group theory \\
		8 & If $-1$ is an $n$th power, then for each $X\neq0$, there are $d$ choices for $Y$ with $Y^n=-X^n$. & \\
		9 & But scaling: $(X,Y,0)\sim(\lambda X,\lambda Y,0)$. Each line through origin corresponds to one projective point. & \\
		10 & So points with $Z=0$ correspond to solutions of $U^n=-1$ in $\mathbb{F}_p^{*}$, where $U=Y/X$. & Homogeneous coordinates \\
		11 & Number of such $U$ is the number of $n$th roots of $-1$, which is $\gcd(n,p-1)$ if $-1$ is an $n$th power, else $0$. & \\
		12 & So points with $Z=0$: either $0$ or $d$ projective points. & \\
		13 & Case 2: $Z\neq0$. Scale so $Z=1$. Equation becomes $x^n+y^n=1$ where $x=X/Z$, $y=Y/Z$. & Affine chart \\
		14 & Number of affine solutions $(x,y)\in\mathbb{F}_p^2$ to $x^n+y^n=1$. & \\
		15 & For each $x$, we need $y^n=1-x^n$. Number of $y$ satisfying this is the number of $n$th roots of $1-x^n$. & \\
		16 & If $1-x^n=0$, there is $1$ solution $y=0$. If $1-x^n\neq0$ and is an $n$th power, there are $d$ solutions. & \\
		17 & Let $N$ be the number of $x$ such that $1-x^n$ is an $n$th power (including $0$). & Define \\
		18 & Then total affine solutions $= 1\cdot\#\{x:1-x^n=0\} + d\cdot\#\{x:1-x^n\neq0\text{ and is an }n\text{th power}\}$. & Counting \\
	19 & By symmetry and character sums, one can show $N\approx p/d$. & Heuristic \\
	20 & More precisely, using Weil bounds, number of affine solutions is $p + O(\sqrt{p})$. & Algebraic geometry \\
	21 & Adding the projective closure points ($Z=0$), total projective points is approximately $p + d + 1$. & Estimate \\
	22 & Exact formula: For the Fermat curve $X^n+Y^n=Z^n$ of genus $g=\frac{(n-1)(n-2)}{2}$, & Genus formula \\
		23 & by Hasse-Weil theorem, $|\#C(\mathbb{F}_p) - (p+1)| \leq 2g\sqrt{p}$. & Bound \\
		24 & So number of projective solutions is $p+1+O(\sqrt{p})$, with constant depending on $n$ and $p$. & Conclusion \\
	\end{prooftable}
\end{solution}
\vspace{10pt}\begin{exercises}[Special Case $n=3$, $p\equiv2\pmod{3}$]
	For $p\equiv2\pmod{3}$, count the exact number of solutions to $x^3+y^3\equiv z^3\pmod{p}$ with $xyz\not\equiv0\pmod{p}$.
\end{exercises}\vspace{5pt}
\begin{solution}
	\begin{prooftable}
		1 & Let $p\equiv2\pmod{3}$, so $\gcd(3,p-1)=1$. & Given \\
		2 & By previous exercises (3), the map $x\mapsto x^3$ is a bijection on $\mathbb{F}_p$. & Bijection property \\
		3 & Consider equation $x^3+y^3\equiv z^3\pmod{p}$ with $xyz\not\equiv0\pmod{p}$. & \\
		4 & For fixed $z\neq0$, let $w=z^3$. Equation becomes $x^3+y^3\equiv w\pmod{p}$. & \\
		5 & Since cubing is bijective, let $u=x^3$, $v=y^3$. Then $u+v\equiv w\pmod{p}$. & Change variables \\
		6 & For each $u\in\mathbb{F}_p$, there is exactly one $v=w-u$ satisfying the equation. & Linear equation \\
		7 & Thus there are $p$ pairs $(u,v)$ satisfying $u+v\equiv w$. & Counting \\
		8 & Each $u$ corresponds to exactly one $x$ (since cubing bijective), similarly for $v$ and $y$. & Bijection \\
		9 & So for fixed $z$, there are exactly $p$ pairs $(x,y)$ satisfying the equation. & \\
		10 & However, some of these have $x=0$ or $y=0$. We want $xyz\neq0$. & Exclude zeros \\
		11 & Case $x=0$: Then $y^3\equiv z^3$, so $y\equiv z$ (bijection). For each $z\neq0$, one such solution. & \\
		12 & So $p-1$ solutions with $x=0$, $y=z\neq0$, $z\neq0$. & Count \\
		13 & Case $y=0$: Similarly, $x^3\equiv z^3$, so $x\equiv z$, giving $p-1$ solutions. & \\
		14 & The solution $x=y=0$ gives $z=0$, excluded. & \\
		15 & Also note: The solution $x=0,y=z$ is the same as solution $y=0,x=z$ only when $z=0$, which is excluded. & No overlap \\
		16 & Thus total solutions with $xyz\neq0$ is: $p(p-1) - (p-1) - (p-1) = (p-1)(p-2)$. & Final count \\
		17 & Example: $p=5$ (5 $\equiv$ 2 mod 3). Compute: $(5-1)(5-2)=4\times3=12$ solutions. & Verification \\
		18 & List them: For $z=1$, equation $x^3+y^3\equiv1$. Cubes mod5: 0,1,3,2,4 (bijection). & \\
		19 & Solutions with $xyz\neq0$: Need $x\neq0$, $y\neq0$, $z\neq0$. & \\
		20 & For $z=1$: Possible $(x,y)$ with $x^3+y^3\equiv1$ and $x,y\neq0$: & \\
		21 & $(1,0),(0,1)$ excluded (zero), $(2,4),(3,2),(4,3)$ and symmetric. Wait, check: $2^3=3$, $4^3=4$, $3+4=7\equiv2\neq1$. & Actually compute \\
		22 & Let's systematically compute: For $z=1$, need $x^3+y^3\equiv1$. & \\
		23 & $x=1$: $1+y^3\equiv1$ $\Rightarrow$ $y^3\equiv0$ $\Rightarrow$ $y=0$ (excluded). & \\
		24 & $x=2$: $3+y^3\equiv1$ $\Rightarrow$ $y^3\equiv3$ $\Rightarrow$ $y=2$ (since $2^3=8\equiv3$). So $(2,2)$ works. & \\
		25 & $x=3$: $2+y^3\equiv1$ $\Rightarrow$ $y^3\equiv4$ $\Rightarrow$ $y=4$ (since $4^3=64\equiv4$). So $(3,4)$ works. & \\
		26 & $x=4$: $4+y^3\equiv1$ $\Rightarrow$ $y^3\equiv2$ $\Rightarrow$ $y=3$. So $(4,3)$ works. & \\
		27 & So for $z=1$, we have $(2,2),(3,4),(4,3)$ and symmetric. No, equation symmetric, so $(2,2)$ symmetric to itself, others are distinct. & \\
		28 & That's 3 solutions for $z=1$. Similarly for other $z$, total $4\times3=12$, matching formula. & \\
	\end{prooftable}
\end{solution}
\makeatletter
\let\orig@addcontentsline 
\addcontentsline
\renewcommand{\addcontentsline}[3]{}   \makeatother
\vspace{10pt}
\subsection*{Index of Definitions}
\vspace{10pt}
\begin{enumerate}
	\item \textbf{Modular Arithmetic}: Congruence, Residue Class, $\mathbb{Z}/n\mathbb{Z}$, \textbf{CRT}, Finite Field
	\item \textbf{Number Representation}: Dyadic Rational, Greedy Binary Expansion, Binary Field, n-bit Ring
	\item \textbf{Algebra}: Convolution Algebras, Formal Power Series, Delta Function
	\item \textbf{Logic}: Many-Valued Algebra, Modular Set Algebra, Valuation Functions
	\item \textbf{Elliptic Curves}: Weierstrass Equation, Group Law, Hasse Bound, Frobenius
	\item \textbf{Calculus}: Modular Derivatives, p-adic Derivative, Fundamental Theorem
	\item \textbf{Polynomials}: Hensel Lifting, Modular Resolution
	\item \textbf{Fermat Equations}: Fermat-Type, Cubic Residue, Local-Global Principle
\end{enumerate}
\vspace{10pt}
\makeatletter
\let\addcontentsline\orig@addcontentsline
\makeatother
% ========== CONCLUSION ==========
\section{Conclusion}
\setstretch{1.2}
This work has established a comprehensive framework unifying modular arithmetic with logical systems, algebraic structures, and geometric objects. The key syntheses achieved include:
\subsection{Principal Results}
\begin{enumerate}
	\item The \textbf{Prime-Modular Logic-Set Isomorphism} provides a rigorous connection between Many-Valued logics and modular set algebras, valid only for prime moduli where clean algebraic structures emerge.
	\item \textbf{Constructive algorithms} for parametric congruences demonstrate how Hensel lifting and Chinese Remainder Theorem decomposition enable systematic resolution of modular equations.
	\item The \textbf{algebraic characterization} of convolution algebras (total, cyclic, and truncated) complete with isomorphism theorems establishes these structures as fundamental objects in harmonic analysis and signal processing.
	\item \textbf{Geometric applications} show how elliptic curves over finite fields naturally embody modular principles, with the Hasse bound providing a precise link between point counts and field characteristics.
\end{enumerate}
\subsection{Methodological Innovations}
The structured \texttt{prooftable} environment introduced in this report represents a significant pedagogical and expository advance. By presenting complex mathematical proofs as sequences of justified steps, we enhance verifiability, support automated proof checking, and make advanced mathematics more accessible.
\subsection{Future Research Directions}
\begin{enumerate}
	\item \textbf{Computational Implementations}: Developing software libraries that implement the logic-set isomorphisms described herein.
	\item \textbf{Generalized Logic Systems}: Extending the prime-modular framework to other non-classical logics and exploring connections with topos theory.
	\item \textbf{Cryptographic Applications}: Applying the unified framework to construct new cryptographic protocols based on the interplay between logical operations and modular arithmetic.
	\item \textbf{Formal Verification}: Using the stepwise proof structure to support fully formal verification of the mathematical results in proof assistants like Coq or Lean.
\end{enumerate}
\subsection{ Final Remarks}
The unity of mathematics is beautifully revealed in the connections between modular arithmetic, Many-Valued logic, algebra, and geometry developed in this work. Far from being isolated domains, these fields interact through shared structural principles that become particularly evident when examined through the lens of finiteness and modularity. The framework presented here not only synthesizes existing knowledge but also opens new pathways for research at the intersections of these fundamental mathematical disciplines.
% ============================================
% APPENDIX
% ============================================
\makeatletter
\let\orig@addcontentsline 
\addcontentsline
\renewcommand{\addcontentsline}[3]{}   \makeatother
\appendix
\part*{Appendices} 
\section{Proof Table Environment Code} 
\label{sec:proof-table-code}
The following LaTeX code defines the \texttt{prooftable} environment 
used for structured proof presentation:
\begin{lstlisting}[
language={[LaTeX]TeX},
basicstyle=\ttfamily\fontsize{6}{8}\selectfont,
caption={LaTeX code for \texttt{prooftable} environment},
label=lst:prooftable,
frame=shadowbox,
rulesepcolor=\color{gray!40},
frameround=fttt,
framesep=6pt,
xleftmargin=10pt,
xrightmargin=10pt,
breaklines=true
]
% Define column types for prooftable
\newcolumntype{S}{>{\raggedright\arraybackslash}p{0.75\textwidth}}
\newcolumntype{J}{>{\raggedright\arraybackslash}p{0.25\textwidth}}

% prooftable environment - remove top/bottom spacing only
\newenvironment{prooftable}
{%
\setlength{\LTpre}{0pt}%
\setlength{\LTpost}{0pt}%
\noindent
\renewcommand{\arraystretch}{1.3}%
\begin{longtable}{r S J}%

\textbf{Step} & \textbf{Statement} & \textbf{Justification} \\

\endfirsthead

\textbf{Step} & \textbf{Statement} & \textbf{Justification} \\

\endhead

\multicolumn{3}{r}{\textit{Continued on next page}} \\
\endfoot

\endlastfoot
}
{%
\end{longtable}%
}
\end{lstlisting}
This environment creates a three-column table with:
\begin{enumerate}
	\item Step numbers in the first column
	\item Mathematical statements in the second column (75\% of text width)
	\item Justifications in the third column (25\% of text width)
	\item Automatic page continuation when proofs span multiple pages
	\item Proper header repetition on each page
	\item Professional formatting with horizontal lines
\end{enumerate}
The complete implementation details are available in the document source code.
The code was developed with assistance from the DeepSeek AI Assistant 
\cite{deepseek2024} for automated proof formatting and LaTeX environment 
design.
\section{Prooftables}
This appendix contains all major theorems of this report, each presented with complete step-by-step proofs using the structured \texttt{prooftable} environment.
\label{sec:prooftables-theorems}
\subsection{Modular Arithmetic Theorems}
\subsubsection{Chinese Remainder Theorem}
\begin{theorem}[Chinese Remainder Theorem (CRT)]
	Let $n_1, n_2, \dots, n_k \in \mathbb{N}$ be pairwise coprime integers, and let $n = \prod_{i=1}^k n_i$. Then the natural map
	\begin{equation}
	\phi: \mathbb{Z}/n\mathbb{Z} \to \mathbb{Z}/n_1\mathbb{Z} \times \cdots \times \mathbb{Z}/n_k\mathbb{Z}, \quad \phi([x]_n) = ([x]_{n_1}, \dots, [x]_{n_k})
	\end{equation}
	is a ring isomorphism.
\end{theorem}
\begin{proof}
	\begin{prooftable}
		1 & Define $\phi([x]_n) = ([x]_{n_1}, \dots, [x]_{n_k})$ for $x \in \mathbb{Z}$. & Definition of $\phi$ \\
		2 & If $x \equiv y \pmod{n}$, then $x \equiv y \pmod{n_i}$ for all $i$. & Since $n_i \mid n$ \\
		3 & Thus $[x]_{n_i} = [y]_{n_i}$ for all $i$, so $\phi$ is well-defined. & Well-definedness \\
		4 & Assume $\phi([x]_n) = \phi([y]_n)$. Then $x \equiv y \pmod{n_i}$ for all $i$. & Hypothesis \\
		5 & Since $n_i$ are pairwise coprime, $x \equiv y \pmod{\prod n_i = n}$. & CRT for congruences \\
		6 & Thus $[x]_n = [y]_n$, proving $\phi$ is injective. & Injectivity \\
		7 & For any $(a_1, \dots, a_k) \in \prod \mathbb{Z}/n_i\mathbb{Z}$, need $x$ with $x \equiv a_i \pmod{n_i}$. & Surjectivity requirement \\
		8 & Construct $x = \sum_{i=1}^k a_i M_i N_i$ where $M_i = n/n_i$, $N_i \equiv M_i^{-1} \pmod{n_i}$. & Explicit construction \\
		9 & Then $x \equiv a_i M_i N_i \equiv a_i \cdot 1 \equiv a_i \pmod{n_i}$ for each $i$. & Verification \\
		10 & Thus $\phi([x]_n) = (a_1, \dots, a_k)$, proving surjectivity. & Surjectivity \\
		11 & $\phi([x+y]_n) = ([x+y]_{n_i}) = ([x]_{n_i} + [y]_{n_i}) = \phi([x]_n) + \phi([y]_n)$. & Additive homomorphism \\
		12 & $\phi([xy]_n) = ([xy]_{n_i}) = ([x]_{n_i}[y]_{n_i}) = \phi([x]_n)\phi([y]_n)$. & Multiplicative homomorphism \\
		13 & $\phi([1]_n) = ([1]_{n_1}, \dots, [1]_{n_k})$, preserving unity. & Unity preservation \\
		14 & Therefore, $\phi$ is a ring isomorphism. & Conclusion \\
	\end{prooftable}
\end{proof}
\begin{corollary}[System of Linear Congruences]
	Given pairwise coprime moduli $n_1, \dots, n_k$ and integers $a_1, \dots, a_k$, the system
	\begin{equation}
	x \equiv a_i \pmod{n_i}, \quad i = 1, \dots, k
	\end{equation}
	has a unique solution modulo $n = n_1 \cdots n_k$.
\end{corollary}
\begin{proof}
	\begin{prooftable}
		1 & The element $(a_1, \dots, a_k) \in \prod \mathbb{Z}/n_i\mathbb{Z}$ exists. & By construction \\
		2 & By CRT isomorphism $\phi$, there exists unique $[x]_n \in \mathbb{Z}/n\mathbb{Z}$ with $\phi([x]_n) = (a_1, \dots, a_k)$. & Surjectivity of $\phi$ \\
		3 & This $x$ satisfies $x \equiv a_i \pmod{n_i}$ for all $i$. & Definition of $\phi$ \\
		4 & Uniqueness follows from injectivity of $\phi$. & Uniqueness \\
		5 & Explicit solution: $x = \sum_{i=1}^k a_i M_i N_i \mod n$ where $M_i = n/n_i$, $N_i \equiv M_i^{-1} \pmod{n_i}$. & Constructive formula \\
		6 & Verification: For each $j$, $x \equiv a_j M_j N_j \equiv a_j \cdot 1 \equiv a_j \pmod{n_j}$. & Check each congruence \\
		7 & For $i \neq j$, $M_i \equiv 0 \pmod{n_j}$, so those terms vanish modulo $n_j$. & Other terms vanish \\
		8 & Therefore, unique solution exists. & Conclusion \\
	\end{prooftable}
\end{proof}
\subsubsection{Fermat's and Euler's Theorems}
\begin{theorem}[Fermat's Little Theorem]
	If $p$ is prime and $a \in \mathbb{Z}$ with $p \nmid a$, then $a^{p-1} \equiv 1 \pmod{p}$. 
	Equivalently, $a^p \equiv a \pmod{p}$ for all $a \in \mathbb{Z}$.
\end{theorem}
\begin{proof}
	\begin{prooftable}
		1 & Consider the multiplicative group $\mathbb{F}_p^* = \{1, 2, \dots, p-1\}$. & Group definition \\
		2 & $|\mathbb{F}_p^*| = p-1$ since $p$ is prime and we exclude 0. & Group order \\
		3 & For $a \in \mathbb{F}_p^*$, consider the set $A = \{a, 2a, 3a, \dots, (p-1)a\}$. & Construction \\
		4 & If $ia \equiv ja \pmod{p}$, then $p \mid (i-j)a$. & Congruence implies divisibility \\
		5 & Since $\gcd(a,p)=1$, $p \mid (i-j)$, so $i \equiv j \pmod{p}$. & Number theory lemma \\
		6 & Thus elements of $A$ are distinct modulo $p$, so $A$ is a permutation of $\mathbb{F}_p^*$. & Conclusion from step 5 \\
		7 & Multiply all elements: $\prod_{k=1}^{p-1} (ka) \equiv \prod_{k=1}^{p-1} k \pmod{p}$. & From step 6 \\
		8 & This gives $a^{p-1} \cdot (p-1)! \equiv (p-1)! \pmod{p}$. & Factoring $a^{p-1}$ \\
		9 & Since $(p-1)! \not\equiv 0 \pmod{p}$ (Wilson's theorem), we can cancel. & Wilson's theorem \\
		10 & Thus $a^{p-1} \equiv 1 \pmod{p}$, proving the first statement. & Cancellation \\
		11 & For $a$ divisible by $p$: $a \equiv 0 \pmod{p}$, so $a^p \equiv 0 \equiv a \pmod{p}$. & Trivial case \\
		12 & For $a$ not divisible by $p$: Multiply $a^{p-1} \equiv 1$ by $a$ to get $a^p \equiv a \pmod{p}$. & Multiply by $a$ \\
		13 & Therefore, $a^p \equiv a \pmod{p}$ holds for all $a \in \mathbb{Z}$. & Conclusion \\
	\end{prooftable}
\end{proof}
\begin{theorem}[Euler's Theorem]
	For $n \in \mathbb{N}$ and $a \in \mathbb{Z}$ with $\gcd(a,n) = 1$, then
	\begin{equation}
	a^{\varphi(n)} \equiv 1 \pmod{n},
	\end{equation}
	where $\varphi$ is Euler's totient function.
\end{theorem}
\begin{proof}
	\begin{prooftable}
		1 & Consider the multiplicative group $(\mathbb{Z}/n\mathbb{Z})^\times$ of units modulo $n$. & Group definition \\
		2 & $|(\mathbb{Z}/n\mathbb{Z})^\times| = \varphi(n)$ by definition of $\varphi$. & Group order \\
		3 & For $a$ with $\gcd(a,n)=1$, $[a]_n \in (\mathbb{Z}/n\mathbb{Z})^\times$. & Unit condition \\
		4 & By Lagrange's theorem, the order of $[a]_n$ divides $\varphi(n)$. & Group theory \\
		5 & Thus $([a]_n)^{\varphi(n)} = [1]_n$ in the group. & Consequence of Lagrange \\
		6 & In congruence notation: $a^{\varphi(n)} \equiv 1 \pmod{n}$. & Translation \\
		7 & Alternative proof: Let $\{r_1, \dots, r_{\varphi(n)}\}$ be reduced residues modulo $n$. & Alternative approach \\
		8 & Since $\gcd(a,n)=1$, $\{ar_1, \dots, ar_{\varphi(n)}\}$ is also a reduced residue system. & Number theory lemma \\
		9 & Thus $\prod_{i=1}^{\varphi(n)} (ar_i) \equiv \prod_{i=1}^{\varphi(n)} r_i \pmod{n}$. & Products are congruent \\
		10 & This gives $a^{\varphi(n)} \prod r_i \equiv \prod r_i \pmod{n}$. & Factor out $a^{\varphi(n)}$ \\
		11 & Since each $r_i$ is invertible modulo $n$, $\prod r_i$ is invertible. & Units multiply to unit \\
		12 & Cancel $\prod r_i$ to get $a^{\varphi(n)} \equiv 1 \pmod{n}$. & Cancellation \\
		13 & This generalizes Fermat's theorem (special case $n=p$ prime). & Relation to Fermat \\
	\end{prooftable}
\end{proof}
\begin{theorem}[Wilson's Theorem]
	For prime $p$, $(p-1)! \equiv -1 \pmod{p}$.
\end{theorem}
\begin{proof}
	\begin{prooftable}
		1 & For $p=2$: $(2-1)! = 1 \equiv -1 \pmod{2}$, true. & Check small case \\
		2 & For odd prime $p$, consider $\mathbb{F}_p^* = \{1, 2, \dots, p-1\}$. & Setup \\
		3 & In $\mathbb{F}_p^*$, each element $a$ has a unique multiplicative inverse $a^{-1}$. & Group property \\
		4 & Note: $a = a^{-1}$ iff $a^2 \equiv 1 \pmod{p}$, i.e., $a \equiv \pm 1 \pmod{p}$. & Solving $a^2=1$ \\
		5 & Thus elements $2, 3, \dots, p-2$ pair up into $(p-3)/2$ pairs of inverses. & Pairing \\
		6 & Product of each pair: $a \cdot a^{-1} \equiv 1 \pmod{p}$. & Inverse property \\
		7 & So $\prod_{a=2}^{p-2} a \equiv 1^{(p-3)/2} = 1 \pmod{p}$. & Product of pairs \\
		8 & Then $(p-1)! = 1 \cdot (2 \cdots (p-2)) \cdot (p-1)$. & Factorial expansion \\
		9 & $\equiv 1 \cdot 1 \cdot (p-1) \pmod{p}$. & Substitution from step 7 \\
		10 & Since $p-1 \equiv -1 \pmod{p}$, we get $(p-1)! \equiv -1 \pmod{p}$. & Final congruence \\
		11 & Conversely, if $(n-1)! \equiv -1 \pmod{n}$, then $n$ must be prime. & Converse \\
		12 & Proof of converse: If $n$ composite with proper divisor $d$, then $d \mid (n-1)!$ but $d \nmid -1$. & Contradiction \\
		13 & Thus Wilson's theorem gives primality criterion. & Primality test \\
	\end{prooftable}
\end{proof}
\subsection{Number Representation Theorems}
\subsubsection{Binary Expansions}
\begin{theorem}[Binary Expansion Field Isomorphism]
	The binary expansion field $\mathbb{B}$ with digit-wise addition and carry propagation is isomorphic to $[0,1] \subset \mathbb{R}$.
\end{theorem}
\begin{proof}
	\begin{prooftable}
		1 & Define $\phi: \mathbb{B} \to [0,1]$ by $\phi(\{b_k\}) = \sum_{k\geq 1} b_k 2^{-k}$. & Mapping definition \\
		2 & $\phi$ is well-defined: Every binary sequence converges to a real in $[0,1]$. & Convergence \\
		3 & $\phi$ is injective: Different sequences give different real sums (except dyadic rationals). & Uniqueness of expansion \\
		4 & For dyadic rationals: Two representations exist but are identified in $\mathbb{B}$. & Technical detail \\
		5 & $\phi$ is surjective: Every $x \in [0,1]$ has a binary expansion via greedy algorithm. & Existence of expansion \\
		6 & Check addition: $\phi(\{b_k\} + \{c_k\}) = \phi(\{b_k\}) + \phi(\{c_k\})$. & Homomorphism check \\
		7 & This follows from standard binary addition algorithm with carries. & Binary arithmetic \\
		8 & Example: $0.0111\ldots + 0.0001\ldots = 0.1000\ldots$ maps to $1/2 + 1/16 = 9/16$. & Verification \\
		9 & Check multiplication: Define multiplication in $\mathbb{B}$ via convolution of expansions. & Operation definition \\
		10 & Then $\phi(\{b_k\} \cdot \{c_k\}) = \phi(\{b_k\}) \cdot \phi(\{c_k\})$. & Multiplicative homomorphism \\
		11 & The distributive laws hold by properties of binary arithmetic. & Algebraic properties \\
		12 & $\phi$ preserves order: $\{b_k\} \leq \{c_k\}$ iff $\phi(\{b_k\}) \leq \phi(\{c_k\})$. & Order isomorphism \\
		13 & Thus $\phi$ is an isomorphism of ordered fields. & Conclusion \\
		14 & The inverse $\phi^{-1}$ is the greedy binary expansion algorithm. & Inverse mapping \\
	\end{prooftable}
\end{proof}
\begin{proposition}[Error Control in Partial Sums]
	Let $S_M(x) = \sum_{m=1}^M \frac{b_m}{2^m}$ be the $M$-th partial sum of the greedy binary expansion of $x \in [0,1]$. Then
	\begin{equation}
	0 \leq x - S_M(x) < \frac{1}{2^M}.
	\end{equation}
\end{proposition}
\begin{proof}
	\begin{prooftable}
		1 & By construction of greedy algorithm: $S_M(x) \leq x$ for all $M$. & Greedy property \\
		2 & Also, $x < S_M(x) + \frac{1}{2^M}$ by digit choice criterion. & Algorithm specification \\
		3 & Combine: $S_M(x) \leq x < S_M(x) + \frac{1}{2^M}$. & Inequality chain \\
		4 & Subtract $S_M(x)$: $0 \leq x - S_M(x) < \frac{1}{2^M}$. & Simple algebra \\
		5 & Alternative inductive proof: Base case $M=0$: $S_0(x)=0$, $0 \leq x < 1$. & Inductive approach \\
		6 & Assume true for $M$: $0 \leq x - S_M(x) < 2^{-M}$. & Induction hypothesis \\
		7 & Let $x' = 2^M(x - S_M(x)) \in [0,1)$. & Rescaled remainder \\
		8 & Next digit $b_{M+1} = 1$ if $x' \geq 1/2$, else $0$. & Greedy choice \\
		9 & If $b_{M+1}=1$: $S_{M+1}(x) = S_M(x) + 2^{-(M+1)}$. & Update \\
		10 & Then $x - S_{M+1}(x) = x - S_M(x) - 2^{-(M+1)} < 2^{-M} - 2^{-(M+1)} = 2^{-(M+1)}$. & Bound \\
		11 & If $b_{M+1}=0$: $S_{M+1}(x) = S_M(x)$, and $x' < 1/2$. & Other case \\
		12 & Then $x - S_{M+1}(x) = x - S_M(x) < 2^{-(M+1)}$. & Bound \\
		13 & In both cases, bound holds for $M+1$, completing induction. & Conclusion \\
	\end{prooftable}
\end{proof}
\subsection{Convolution Algebra Theorems}
\subsubsection{Algebraic Structure}
\begin{theorem}[Algebraic Characterization of Total Convolution Algebra]
	For any abelian group $G$, the total convolution algebra $\mathcal{C}(G)$ is a commutative associative algebra over $\mathbb{C}$ with unit $\delta_0$, where $\delta_0(x) = 1$ if $x=0$, and $0$ otherwise.
\end{theorem}
\begin{proof}
	\begin{prooftable}
		1 & Define $\mathcal{C}(G) = \{f: G \to \mathbb{C}\}$ with pointwise addition. & Vector space structure \\
		2 & Convolution: $(f*g)(x) = \sum_{y \in G} f(y)g(x-y)$. & Operation definition \\
		3 & Commutativity: $(f*g)(x) = \sum_{y} f(y)g(x-y)$. & Expression \\
		4 & Change variable $z = x-y$: $= \sum_{z} g(z)f(x-z) = (g*f)(x)$. & Commutativity proof \\
		5 & Associativity: Check $((f*g)*h)(x) = (f*(g*h))(x)$. & Need to verify \\
		6 & LHS: $\sum_y (f*g)(y) h(x-y) = \sum_y \sum_z f(z)g(y-z) h(x-y)$. & Expand \\
		7 & Change summation order: $= \sum_z f(z) \sum_y g(y-z) h(x-y)$. & Rearrange \\
		8 & Let $w = y-z$: $= \sum_z f(z) \sum_w g(w) h(x-z-w)$. & Variable change \\
		9 & This equals $\sum_z f(z) (g*h)(x-z) = (f*(g*h))(x)$. & RHS \\
		10 & Thus convolution is associative. & Associativity proven \\
		11 & Distributive over addition: $(f+g)*h = f*h + g*h$. & Easy check \\
		12 & Identity: Check $(\delta_0*f)(x) = \sum_y \delta_0(y)f(x-y) = f(x)$. & Identity element \\
		13 & Similarly $(f*\delta_0)(x) = f(x)$. & Both sides \\
		14 & Therefore $\mathcal{C}(G)$ is unital commutative associative algebra. & Conclusion \\
	\end{prooftable}
\end{proof}
\begin{theorem}[Isomorphism Theorem for Cyclic Convolution Algebras]
	For cyclic group $C_n = \mathbb{Z}/n\mathbb{Z}$, there is an algebra isomorphism
	\begin{equation}
	\mathcal{C}_n \cong \mathbb{C}[x]/(x^n - 1)
	\end{equation}
	via the Discrete Fourier Transform (DFT).
\end{theorem}
\begin{proof}
	\begin{prooftable}
		1 & Define DFT matrix: $F_{jk} = \omega^{jk}$ with $\omega = e^{2\pi i/n}$. & DFT definition \\
		2 & DFT maps sequence $a = (a_0, \dots, a_{n-1})$ to $\hat{a} = F a$. & Transform \\
		3 & Convolution theorem: $\widehat{a*b} = \hat{a} \odot \hat{b}$ (pointwise product). & Key property \\
		4 & Define $\phi: \mathcal{C}_n \to \mathbb{C}[x]/(x^n-1)$ by $a \mapsto \sum_{k=0}^{n-1} a_k x^k$. & Polynomial map \\
		5 & $\phi$ is linear and bijective (dimension $n$ both sides). & Vector space isomorphism \\
		6 & Check multiplication: $\phi(a*b) = \phi(a) \cdot \phi(b) \mod (x^n-1)$. & Need to verify \\
		7 & In $\mathbb{C}[x]/(x^n-1)$: $x^n = 1$, so $x^{k} \cdot x^{\ell} = x^{k+\ell \mod n}$. & Ring structure \\
		8 & This matches cyclic convolution: $(a*b)_m = \sum_{k+\ell \equiv m \mod n} a_k b_\ell$. & Correspondence \\
		9 & Algebraically: Multiplication in quotient ring gives convolution. & Formal check \\
		10 & DFT diagonalizes convolution: $F(a*b) = (Fa) \odot (Fb)$. & Diagonalization \\
		11 & This means convolution corresponds to pointwise multiplication of DFT coefficients. & Interpretation \\
		12 & The isomorphism preserves all algebraic operations. & Complete isomorphism \\
		13 & Inverse map: $\phi^{-1}(p(x)) =$ coefficients of $p(x) \mod (x^n-1)$. & Inverse \\
		14 & Therefore, $\mathcal{C}_n \cong \mathbb{C}[x]/(x^n-1)$. & Conclusion \\
	\end{prooftable}
\end{proof}
\begin{theorem}[Unit Characterization in Formal Power Series]
	A formal power series $f = \sum_{n\geq 0} a_n x^n \in R[[x]]$ is a unit if and only if its constant term $a_0$ is a unit in the base ring $R$.
\end{theorem}
\begin{proof}
	\begin{prooftable}
		1 & ($\Rightarrow$) Suppose $f$ is unit with inverse $g = \sum b_n x^n$. & Assume unit \\
		2 & Then $f*g = 1 = 1 + 0x + 0x^2 + \cdots$. & Identity \\
		3 & Constant term gives $a_0 b_0 = 1$. & First equation \\
		4 & Thus $a_0$ has inverse $b_0$ in $R$, so $a_0$ is unit. & Conclusion \\
		5 & ($\Leftarrow$) Suppose $a_0$ is unit in $R$ with inverse $a_0^{-1}$. & Assume condition \\
		6 & We construct $g = \sum b_n x^n$ recursively. & Constructive proof \\
		7 & Set $b_0 = a_0^{-1}$. & Base case \\
		8 & For $n \geq 1$, require coefficient of $x^n$ in $f*g$ to be 0. & Condition \\
		9 & This gives: $a_0 b_n + a_1 b_{n-1} + \cdots + a_n b_0 = 0$. & Equation for $b_n$ \\
		10 & Solve for $b_n$: $b_n = -a_0^{-1} \sum_{k=1}^n a_k b_{n-k}$. & Recursive formula \\
		11 & This defines $g$ uniquely by induction. & Existence and uniqueness \\
		12 & Check by induction: $f*g = 1$ by construction. & Verification \\
		13 & Also $g*f = 1$ by similar reasoning (commutative ring). & Two-sided inverse \\
		14 & Therefore $f$ is a unit in $R[[x]]$. & Conclusion \\
	\end{prooftable}
\end{proof}
\subsection{Many-Valued Logic Theorems}
\subsubsection{Prime-Modular Logic-Set Isomorphism}
\begin{theorem}[Prime-Modular Logic-Set Isomorphism]
	For prime $p$, the Many-Valued Algebra $MV_p$ is isomorphic to the modular set algebra $S_p$. The isomorphism $\phi: MV_p \to S_p$ is given by
	\begin{equation}
	\phi\left(\frac{k}{p-1}\right) = [\{0, 1, \dots, k-1\}]_{\sim},
	\end{equation}
	where $[A]_{\sim}$ denotes equivalence class under cardinality modulo $p$.
\end{theorem}
\begin{proof}
	\begin{prooftable}
		1 & Define $\phi\left(\frac{k}{p-1}\right) = [\{0,\dots,k-1\}]_{\sim}$ for $k=0,\dots,p-1$. & Isomorphism definition \\
		2 & $\phi$ is well-defined: Each $k$ gives unique equivalence class. & Well-definedness \\
		3 & $\phi$ is injective: Different $k$ give sets of different cardinalities modulo $p$. & Since $p$ is prime \\
		4 & Cardinalities $0,1,\dots,p-1$ are all distinct modulo $p$. & Modular arithmetic \\
		5 & $\phi$ is surjective: Any $[A]_{\sim} \in S_p$ has cardinality $k \pmod{p}$. & By definition of $S_p$ \\
		6 & This $k$ corresponds to $\frac{k}{p-1} \in MV_p$ via $\phi^{-1}$. & Inverse construction \\
		7 & Check negation: $\phi(\neg x) = \phi(1 - \frac{k}{p-1}) = \phi(\frac{p-1-k}{p-1})$. & Negation in $MV_p$ \\
		8 & This equals $[\{0,\dots,p-2-k\}]_{\sim} = \mathbb{Z}_p \setminus \{0,\dots,k-1\} = \neg\phi(x)$. & Set complement \\
		9 & Check conjunction: $\phi(x \land y) = \phi(\max(0, \frac{k}{p-1} + \frac{\ell}{p-1} - 1))$. & Conjunction in $MV_p$ \\
		10 & After simplification: $\phi(\frac{\max(0, k+\ell-(p-1))}{p-1})$. & Algebra \\
		11 & This equals $[\{0,\dots,\max(0,k+\ell-(p-1))-1\}]_{\sim}$. & Application of $\phi$ \\
		12 & Meanwhile, $\phi(x) \sqcap \phi(y) = [\{0,\dots,k-1\}] \sqcap [\{0,\dots,\ell-1\}]$. & Conjunction in $S_p$ \\
		13 & By definition of $\sqcap$, this equals same set class. & Matching \\
		14 & Similar verification for disjunction $\phi(x \lor y) = \phi(x) \sqcup \phi(y)$. & Parallel argument \\
		15 & Therefore, $\phi$ preserves all operations, hence is isomorphism. & Conclusion \\
	\end{prooftable}
\end{proof}
\begin{corollary}[Sharp Bounds for $p > 2$]
	For prime $p > 2$ and any $x \in MV_p$,
	\begin{equation}
	x \land \neg x \leq \frac{p-1}{2}, \quad x \lor \neg x \geq \frac{p-1}{2},
	\end{equation}
	where bounds are in the normalized scale $[0,1]$.
\end{corollary}
\begin{proof}
	\begin{prooftable}
		1 & Let $x = \frac{k}{p-1}$ for some $k \in \{0,\dots,p-1\}$. & Representation \\
		2 & Then $\neg x = 1 - \frac{k}{p-1} = \frac{p-1-k}{p-1}$. & Negation \\
		3 & Compute $x \land \neg x = \max(0, \frac{k}{p-1} + \frac{p-1-k}{p-1} - 1)$. & Conjunction \\
		4 & Simplify: $\max(0, \frac{k+(p-1-k)-(p-1)}{p-1}) = \max(0,0) = 0$. & Algebra \\
		5 & In unnormalized form: $0 \leq \frac{p-1}{2}$ since $p>2$, $p-1 \geq 2$. & Inequality \\
		6 & Normalized: $0 \leq \frac{1}{2}$ in $[0,1]$ scale. & Normalized bound \\
		7 & For disjunction: $x \lor \neg x = \min(1, \frac{k}{p-1} + \frac{p-1-k}{p-1})$. & Disjunction \\
		8 & Simplify: $\min(1, \frac{p-1}{p-1}) = \min(1,1) = 1$. & Calculation \\
		9 & In unnormalized: $1 \geq \frac{p-1}{2}$ for $p>2$. & Inequality \\
		10 & Normalized: $1 \geq \frac{1}{2}$ always true. & Normalized bound \\
		11 & These bounds are sharp: attained for $x = \frac{1}{2}$ (when $p>2$). & Sharpness \\
		12 & For $p=2$, bounds become $x \land \neg x = 0$, $x \lor \neg x = 1$. & Special case \\
		13 & The theorem quantifies "law of excluded middle" in Many-Valued logic. & Interpretation \\
	\end{prooftable}
\end{proof}
\begin{theorem}[Polynomial Constraint Characterization]
	A polynomial identity $P(x_1, \dots, x_n) = Q(x_1, \dots, x_n)$ holds in all Many-Valued Algebras $MV_p$ if and only if it holds in $MV_2$ (Boolean algebra).
\end{theorem}
\begin{proof}
	\begin{prooftable}
		1 & ($\Rightarrow$) Trivial: if holds in all $MV_p$, holds in $MV_2$. & One direction \\
		2 & ($\Leftarrow$) Assume identity holds in Boolean algebra $MV_2$. & Hypothesis \\
		3 & Any Many-Valued Algebra $MV_p$ contains Boolean algebra as subalgebra. & Substructure \\
		4 & Specifically, elements $0$ and $1$ in $MV_p$ form Boolean algebra. & Two-element subset \\
		5 & More generally, any polynomial evaluated on $\{0,1\}$ values in $MV_p$... & On Boolean inputs \\
		6 & ...gives same result as in $MV_2$ by design of operations. & Operation compatibility \\
		7 & Need to check for intermediate values in $[0,1]$. & General case \\
		8 & Key fact: Many-Valued operations are piecewise linear. & Property \\
		9 & Polynomials in these operations are also piecewise linear functions. & Closure \\
		10 & If identity holds on all Boolean inputs, it holds on all inputs. & Linear extension \\
		11 & Formal proof uses McNaughton's theorem on piecewise linear functions. & Reference \\
		12 & Alternatively: Show by induction on structure of polynomials. & Inductive proof \\
		13 & Base case: variables and constants obviously behave correctly. & Base \\
		14 & Inductive step: Preserved under Many-Valued operations. & Induction \\
		15 & Therefore, Boolean validity implies validity in all $MV_p$. & Conclusion \\
	\end{prooftable}
\end{proof}
\subsection{Elliptic Curve Theorems}
\begin{theorem}[Hasse's Theorem for Elliptic Curves]
	Let $E$ be an elliptic curve over the finite field $\mathbb{F}_q$. Then the number of $\mathbb{F}_q$-rational points satisfies
	\begin{equation}
	|\#E(\mathbb{F}_q) - (q+1)| \leq 2\sqrt{q}.
	\end{equation}
\end{theorem}
\begin{proof}
	\begin{prooftable}
		1 & Let $\#E(\mathbb{F}_q) = q + 1 - t$ where $t$ is the trace of Frobenius. & Definition of $t$ \\
		2 & The Frobenius endomorphism $\pi_q$ satisfies characteristic polynomial: & Property \\
		3 & $\pi_q^2 - t\pi_q + q = 0$ as endomorphisms of $E$. & Characteristic polynomial \\
		4 & This polynomial has discriminant $\Delta = t^2 - 4q$. & Discriminant \\
		5 & For elliptic curves over $\mathbb{C}$, endomorphism ring is order in imaginary quadratic field. & Complex case \\
		6 & Over finite fields, $\pi_q$ satisfies similar properties. & Analogy \\
		7 & The inequality $|t| \leq 2\sqrt{q}$ follows from positivity of certain pairings. & Geometric argument \\
		8 & Alternative: Consider the zeta function of $E/\mathbb{F}_q$: & Algebraic approach \\
		9 & $Z(E/\mathbb{F}_q, T) = \frac{1 - tT + qT^2}{(1-T)(1-qT)}$. & Zeta function \\
		10 & Functional equation implies Riemann hypothesis for curves. & RH for curves \\
		11 & This gives $|\alpha| = \sqrt{q}$ for roots $\alpha$ of $1 - tT + qT^2$. & Consequence of RH \\
		12 & Thus $|t| \leq 2\sqrt{q}$. & From root bounds \\
		13 & Rewriting: $|(q+1-t) - (q+1)| = |t| \leq 2\sqrt{q}$. & Substitution \\
		14 & Therefore $|\#E(\mathbb{F}_q) - (q+1)| \leq 2\sqrt{q}$. & Conclusion \\
	\end{prooftable}
\end{proof}
\begin{corollary}[Possible Orders of Elliptic Curves]
	For elliptic curve over $\mathbb{F}_q$, the possible number of rational points lies in the interval:
	\begin{equation}
	[q+1-2\sqrt{q}, q+1+2\sqrt{q}] \cap \mathbb{Z}.
	\end{equation}
\end{corollary}
\begin{proof}
	\begin{prooftable}
		1 & From Hasse's theorem: $|\#E - (q+1)| \leq 2\sqrt{q}$. & Inequality \\
		2 & This is equivalent to: $q+1-2\sqrt{q} \leq \#E \leq q+1+2\sqrt{q}$. & Rewriting \\
		3 & Since $\#E$ is an integer (number of points), it must be in the intersection with $\mathbb{Z}$. & Integer condition \\
		4 & Example: For $q=7$, $2\sqrt{7} \approx 5.29$, so interval is $[2.71, 12.29]$. & Numerical example \\
		5 & Integer values: $3,4,5,6,7,8,9,10,11,12$. & Possible orders \\
		6 & Not all integers in this interval necessarily occur for given $q$. & Note \\
		7 & But all satisfy the Hasse bound. & Summary \\
	\end{prooftable}
\end{proof}
\subsubsection{Group Structure}
\begin{theorem}[Group Structure of Elliptic Curves over Finite Fields]
	For elliptic curve $E$ over finite field $\mathbb{F}_q$, the group of rational points decomposes as
	\begin{equation}
	E(\mathbb{F}_q) \cong \mathbb{Z}_{n_1} \times \mathbb{Z}_{n_2}
	\end{equation}
	where $n_2 \mid n_1$ and $n_2 \mid (q-1)$.
\end{theorem}

\begin{proof}
	\begin{prooftable}
		1 & $E(\mathbb{F}_q)$ is a finite abelian group. & Basic property \\
		2 & By structure theorem for finite abelian groups: & Group theory \\
		3 & $E(\mathbb{F}_q) \cong \mathbb{Z}_{d_1} \times \cdots \times \mathbb{Z}_{d_k}$ with $d_i \mid d_{i+1}$. & Structure theorem \\
		4 & For elliptic curves, $k \leq 2$ (at most two generators). & Special property \\
		5 & Thus $E(\mathbb{F}_q) \cong \mathbb{Z}_{n_1} \times \mathbb{Z}_{n_2}$ with $n_2 \mid n_1$. & Simplified form \\
		6 & The Weil pairing gives non-degenerate alternating map: & Advanced theory \\
		7 & $E[n] \times E[n] \to \mu_n$ (roots of unity). & Weil pairing \\
		8 & This implies $n_2 \mid (q-1)$ for the $n_2$-torsion. & Consequence \\
		9 & More concretely: The $n_2$-torsion points are defined over $\mathbb{F}_q(\mu_{n_2})$. & Field of definition \\
		10 & Since $E(\mathbb{F}_q)$ already contains these points, $\mu_{n_2} \subseteq \mathbb{F}_q^*$. & Condition \\
		11 & Thus $n_2 \mid (q-1)$. & Conclusion \\
		12 & Example: For $q=11$, possible structures include $\mathbb{Z}_{12}$, $\mathbb{Z}_6 \times \mathbb{Z}_2$, etc. & Example \\
		13 & The theorem constrains possible group structures. & Application \\
	\end{prooftable}
\end{proof}
\subsection{Modular Calculus Theorems}
\subsubsection{Fundamental Theorem of Modular Calculus}
\begin{theorem}[Fundamental Theorem of Modular Calculus]
	For $f: \mathbb{Z}_M \to \mathbb{Z}_M$ and any $a,b \in \mathbb{Z}_M$,
	\begin{equation}
	\sum_{x=a}^{b-1} \Delta_1 f(x) = f(b) - f(a) \pmod{M},
	\end{equation}
	where $\Delta_h f(x) = f(x+h) - f(x) \pmod{M}$ is the forward difference operator.
\end{theorem}
\begin{proof}
	\begin{prooftable}
		1 & Expand the sum: $\sum_{x=a}^{b-1} \Delta_1 f(x) = \sum_{x=a}^{b-1} [f(x+1) - f(x)]$. & Definition of $\Delta_1$ \\
		2 & This is a telescoping sum: & Observation \\
		3 & $= [f(a+1) - f(a)] + [f(a+2) - f(a+1)] + \cdots + [f(b) - f(b-1)]$. & Write terms \\
		4 & Cancel intermediate terms: $f(a+1), f(a+2), \dots, f(b-1)$ cancel. & Telescoping \\
		5 & Remaining terms: $-f(a) + f(b)$. & After cancellation \\
		6 & Thus $\sum_{x=a}^{b-1} \Delta_1 f(x) = f(b) - f(a)$. & Result \\
		7 & All operations are modulo $M$, so equality holds modulo $M$. & Modular arithmetic \\
		8 & If $F$ is an antiderivative ($\Delta_1 F = f$), then: & Corollary \\
		9 & $\sum_{x=a}^{b-1} f(x) = \sum_{x=a}^{b-1} \Delta_1 F(x) = F(b) - F(a)$. & Apply theorem \\
		10 & This is discrete analogue of $\int_a^b f(x)dx = F(b) - F(a)$. & Analogue \\
		11 & Example: For $f(x)=x$, $\Delta_1 f(x)=1$, $\sum_{x=a}^{b-1} 1 = b-a = f(b)-f(a)$. & Verification \\
		12 & The theorem works for any modulus $M$, not necessarily prime. & Generality \\
	\end{prooftable}
\end{proof}
\subsubsection{p-adic Derivative}
\begin{theorem}[p-adic Derivative Properties]
	For prime $p$ and $f: \mathbb{Z}_{p^e} \to \mathbb{Z}_{p^e}$, the p-adic derivative
	\begin{equation}
	D_p f(x) = \displaystyle\limsup_{n \in \mathbb{N}}  \frac{f(x+p^n) - f(x)}{p^n}
	\end{equation}
	satisfies:
	\begin{enumerate}
		\item Linearity: $D_p(af+bg) = a D_p f + b D_p g$
		\item Product rule: $D_p(fg) = f D_p g + g D_p f$
		\item Chain rule: $D_p(f \circ g) = (D_p f \circ g) \cdot D_p g$
	\end{enumerate}
	when the derivatives exist.
\end{theorem}
\begin{proof}
	\begin{prooftable}
		1 & Linearity: $D_p(af+bg)(x) = \displaystyle\limsup_{n \in \mathbb{N}}  \frac{af(x+p^n)+bg(x+p^n)-af(x)-bg(x)}{p^n}$. & Definition \\
		2 & $= a\displaystyle\limsup_{n \in \mathbb{N}}  \frac{f(x+p^n)-f(x)}{p^n} + b\displaystyle\limsup_{n \in \mathbb{N}}  \frac{g(x+p^n)-g(x)}{p^n}$. & Separate limits \\
		3 & $= a D_p f(x) + b D_p g(x)$. & Result \\
		4 & Product rule: Consider $\frac{f(x+p^n)g(x+p^n) - f(x)g(x)}{p^n}$. & Definition for product \\
		5 & Add and subtract $f(x+p^n)g(x)$: $= \frac{f(x+p^n)g(x+p^n) - f(x+p^n)g(x)}{p^n} + \frac{f(x+p^n)g(x) - f(x)g(x)}{p^n}$. & Algebraic trick \\
		6 & $= f(x+p^n) \frac{g(x+p^n)-g(x)}{p^n} + g(x) \frac{f(x+p^n)-f(x)}{p^n}$. & Factor \\
		7 & \text{In the limit of large } n, $f(x+p^n) \to f(x)$ p-adically. & Limit properties \\
		8 & Thus $D_p(fg)(x) = f(x) D_p g(x) + g(x) D_p f(x)$. & Result \\
		9 & Chain rule: For $h = f \circ g$, consider $\frac{h(x+p^n)-h(x)}{p^n}$. & Definition \\
		10 & Write as $\frac{f(g(x+p^n)) - f(g(x))}{p^n}$. & Composition \\
		11 & Multiply and divide by $g(x+p^n)-g(x)$: $= \frac{f(g(x+p^n))-f(g(x))}{g(x+p^n)-g(x)} \cdot \frac{g(x+p^n)-g(x)}{p^n}$. & Algebraic manipulation \\
		12 & \text{In the limit of large } n, first factor $\to D_p f(g(x))$, second $\to D_p g(x)$. & Limits \\
		13 & Thus $D_p(f \circ g)(x) = D_p f(g(x)) \cdot D_p g(x)$. & Result \\
		14 & These rules mirror classical calculus but in p-adic setting. & Analogue \\
	\end{prooftable}
\end{proof}
\subsection{Polynomial Congruence Theorems}
\subsubsection{Hensel's Lemma}
\begin{lemma}[Hensel's Lemma]
	Let $f(x) \in \mathbb{Z}[x]$, $p$ prime, and $x_0 \in \mathbb{Z}$ such that
	\begin{equation}
	f(x_0) \equiv 0 \pmod{p} \quad \text{and} \quad f'(x_0) \not\equiv 0 \pmod{p}.
	\end{equation}
	Then for each $k \geq 1$, there exists a unique $x_k \in \mathbb{Z}/p^{k+1}\mathbb{Z}$ such that
	\begin{equation}
	x_k \equiv x_0 \pmod{p} \quad \text{and} \quad f(x_k) \equiv 0 \pmod{p^{k+1}}.
	\end{equation}
	Moreover, $x_k$ can be computed recursively by
	\begin{equation}
	x_{k+1} = x_k - f(x_k) \cdot [f'(x_k)]^{-1} \pmod{p^{k+2}}.
	\end{equation}
\end{lemma}
\begin{proof}
	\begin{prooftable}
		1 & Base case $k=0$: $x_0$ exists by hypothesis. & Given \\
		2 & Inductive step: Assume $x_k$ exists with $f(x_k) \equiv 0 \pmod{p^{k+1}}$. & Induction hypothesis \\
		3 & Write $x_{k+1} = x_k + p^{k+1} t$ for some $t \in \mathbb{Z}$. & Ansatz \\
		4 & Taylor expand: $f(x_{k+1}) = f(x_k + p^{k+1}t) = f(x_k) + p^{k+1}t f'(x_k) + \cdots$. & Taylor series \\
		5 & Higher terms divisible by $p^{2(k+1)}$, so $\equiv 0 \pmod{p^{k+2}}$ if $k \geq 0$. & Higher order terms \\
		6 & We need $f(x_{k+1}) \equiv 0 \pmod{p^{k+2}}$. & Requirement \\
		7 & From expansion: $f(x_k) + p^{k+1}t f'(x_k) \equiv 0 \pmod{p^{k+2}}$. & Condition \\
		8 & Since $f(x_k) \equiv 0 \pmod{p^{k+1}}$, write $f(x_k) = p^{k+1} A$. & Representation \\
		9 & Then condition becomes: $p^{k+1}A + p^{k+1}t f'(x_k) \equiv 0 \pmod{p^{k+2}}$. & Substitute \\
		10 & Divide by $p^{k+1}$: $A + t f'(x_k) \equiv 0 \pmod{p}$. & Simplify \\
		11 & Since $f'(x_k) \equiv f'(x_0) \not\equiv 0 \pmod{p}$, it's invertible modulo $p$. & Hypothesis \\
		12 & Solve: $t \equiv -A \cdot [f'(x_k)]^{-1} \pmod{p}$. & Solution for $t$ \\
		13 & This gives $x_{k+1} = x_k - f(x_k) \cdot [f'(x_k)]^{-1} \pmod{p^{k+2}}$. & Formula \\
		14 & Uniqueness: Different $t$ would give different solutions modulo $p^{k+2}$. & Uniqueness proof \\
		15 & By induction, solution exists for all $k$. & Conclusion \\
	\end{prooftable}
\end{proof}
\subsection{Fermat-Type Equations}
\begin{theorem}[Cubic Residue Classification]
	For a prime \(p \equiv 1 \pmod{3}\), the set of cubic residues modulo \(p\) forms a subgroup of \(\mathbb{F}_p^*\) of index \(3\).
\end{theorem}
\begin{proof}
\begin{prooftable}
	1 & Let \(p\) be a prime with \(p \equiv 1 \pmod{3}\). & Given \\
	
	2 & Then \(\mathbb{F}_p^*\) is a cyclic group of order \(p-1\). & Finite field multiplicative group is cyclic \\
	
	3 & Since \(3 \mid (p-1)\), there exists a unique subgroup \(H \leq \mathbb{F}_p^*\) of order \(\frac{p-1}{3}\). & Cyclic group property \\
	
	4 & Define the cubic residues as \(C = \{x^3 : x \in \mathbb{F}_p^*\}\). & Definition \\
	
	5 & The map \(\phi: \mathbb{F}_p^* \to \mathbb{F}_p^*\) given by \(\phi(x) = x^3\) is a group homomorphism. & \((xy)^3 = x^3 y^3\) \\
	
	6 & The kernel of \(\phi\) is \(\{x : x^3 = 1\}\), which has size 3 since the equation \(x^3 = 1\) has exactly 3 solutions in \(\mathbb{F}_p^*\) when \(3 \mid (p-1)\). & Roots of unity in cyclic group \\
	
	7 & By the first isomorphism theorem, \(\mathbb{F}_p^*/\ker(\phi) \cong \text{Im}(\phi)\). & Group theory \\
	
	8 & Thus \(|C| = |\text{Im}(\phi)| = \frac{p-1}{3}\). & Counting: \(|\mathbb{F}_p^*|/|\ker(\phi)| = (p-1)/3\) \\
	
	9 & Therefore \(C\) is a subgroup of \(\mathbb{F}_p^*\) of index \([\mathbb{F}_p^* : C] = 3\). & Subgroup index = order of group / order of subgroup \\
	
	10 & Specifically, \(C = H\) where \(H\) is the subgroup from step 3. & Uniqueness of subgroup of given order \\
\end{prooftable}
\end{proof}
\begin{theorem}[Local-Global Solvability via CRT and Hensel]
	Let \(F(x,y,z) = 0\) be a Diophantine equation with integer coefficients.\\ If there exist solutions in \(\mathbb{R}\) and in \(\mathbb{Q}_p\) for all primes \(p\) (i.e., solutions in \(\mathbb{Z}_p\) for each \(p\)), then there exists a solution in \(\mathbb{Z}\), provided the local solutions are compatible under the Chinese Remainder Theorem.
\end{theorem}
\begin{proof}
\begin{prooftable}
	1 & Assume \(F\) has integer coefficients and has solutions in \(\mathbb{R}\) and in \(\mathbb{Z}_p\) for every prime \(p\). & Given (Hasse principle condition) \\
	
	2 & For each prime \(p\), choose a \(p\)-adic solution \((x_p, y_p, z_p) \in \mathbb{Z}_p^3\). & Existence by hypothesis \\
	
	3 & Since \(\mathbb{Z}_p\) is the inverse limit of \(\mathbb{Z}/p^k\mathbb{Z}\), we can approximate each \(p\)-adic solution by integers modulo \(p^{N}\) for large \(N\). & Structure of \(p\)-adic integers \\
	
	4 & Choose a common modulus \(M = \prod_{i=1}^r p_i^{N_i}\) for all relevant primes\hspace{5in} (finitely many primes where solutions are not trivial mod \(p^k\)). & Chinese Remainder Theorem setup \\
	
	5 & For each prime \(p_i\), reduce the \(p_i\)-adic solution modulo \(p_i^{N_i}\) to get integers \(a_i, b_i, c_i\) with: \hspace{2in} \(F(a_i, b_i, c_i) \equiv 0 \pmod{p_i^{N_i}}\). & Approximation \\
	
	6 & Use Chinese Remainder Theorem to find integers \(X, Y, Z\) such that:\hspace{2in}
	\(X \equiv a_i \pmod{p_i^{N_i}}\), \(Y \equiv b_i \pmod{p_i^{N_i}}\), \(Z \equiv c_i \pmod{p_i^{N_i}}\) for all \(i\). & CRT application \\
	
	7 & Then \(F(X, Y, Z) \equiv 0 \pmod{p_i^{N_i}}\) for each prime \(p_i\). & Since reduction preserves congruence \\
	
	8 & Thus \(F(X, Y, Z) \equiv 0 \pmod{M}\). & Chinese Remainder Theorem \\
	
	9 & For primes not among the \(p_i\), the equation \(F \equiv 0 \pmod{p^j}\) holds automatically for large \(j\) because we may have taken \(N_i\) sufficiently large. & Careful choice of exponents \\
	
	10 & Therefore \(F(X, Y, Z) = 0\) holds exactly as an integer equation if the local conditions force the integer solution. & In general, this gives a solution modulo \(M\); for exact solution, use approximation theorem and the fact that \(\mathbb{Z}\) is dense in the adeles \\
	
	11 & The real solution condition ensures the integer solution found is not trivial in real sense. & Real condition excludes degenerate cases \\
\end{prooftable}
\end{proof}
\begin{theorem}[Density Heuristic for Solvability over Finite Fields]
	For the equation \(x^n + y^n = z^n\) over the finite field \(\mathbb{F}_q\), the expected number of projective solutions is approximately \(q^2 / \gcd(n, q-1)\) when \(q\) is large.
\end{theorem}
\begin{proof}
\begin{prooftable}
	1 & Consider the equation \(x^n + y^n = z^n\) in \(\mathbb{F}_q\) with \(q = p^r\). & Setup \\
	
	2 & We count triples \((x, y, z) \in \mathbb{F}_q^3 \setminus \{(0,0,0)\}\) up to scaling. & Projective solutions \(\mathbb{P}^2(\mathbb{F}_q)\) \\
	
	3 & Let \(d = \gcd(n, q-1)\). Then the map \(x \mapsto x^n\) is \(d\)-to-1 onto the subgroup \(H = \{u^d : u \in \mathbb{F}_q^*\}\) of \(\mathbb{F}_q^*\). & Group theory: exponent \(n\) map \\
	
	4 & For fixed \(z \neq 0\), set \(w = z^n\). Equation becomes \(x^n + y^n = w\). & Normalization \\
	
	5 & For each possible \(u = x^n\) and \(v = y^n\) with \(u+v = w\), there are \(d\) choices for \(x\) given \(u\), and \(d\) choices for \(y\) given \(v\). & Counting preimages \\
	
	6 & Number of pairs \((u,v) \in H \times H\) with \(u+v = w\), if random, expected number is about: \(|H|^2 / q = ((q-1)/d)^2 / q \approx q/d^2\). & Heuristic: \(u,v\) uniform in \(H\) \\
	
	7 & Thus expected number of \((x,y)\) for fixed \(z\) is about \(d^2 \cdot (q/d^2) = q\). & Multiply by \(d^2\) choices for \(x,y\) \\
	
	8 & Multiply by number of nonzero \(z\) choices (\(q-1\) choices), but careful:\hspace{2in} solutions are projective, so divide by scaling factor. & Avoid overcount \\
	
	9 & Actually, for each nonzero \(w\), the expected number of solutions \((x,y,z)\) with \(z^n = w\) is about \(q\) (from step 7). & Per \(w\) \\
	
	10 & There are \((q-1)/d\) possible \(w\) values (size of \(H\)). So total affine nonzero solutions \(\approx q \cdot (q-1)/d \approx q^2/d\). & Multiply \\
	
	11 & Projective space \(\mathbb{P}^2\) has \((q^2 + q + 1)\) points.\hspace{5in} The heuristic density: proportion of points satisfying equation is about \(1/d\). & Since \(|\mathbb{P}^2| \approx q^2\) \\
	
	12 & Expected number of projective solutions \(\approx q^2 / d = q^2 / \gcd(n, q-1)\). & Final estimate \\
\end{prooftable}
\end{proof}
\begin{theorem}[Fermat's Little Theorem]
	For prime \(p\) and integer \(a\) with \(\gcd(a,p) = 1\):
	\[
	a^{p-1} \equiv 1 \pmod{p}.
	\]
\end{theorem}
\begin{proof}
\begin{prooftable}
	1 & Consider the set \(S = \{1, 2, \dots, p-1\}\), all nonzero modulo \(p\). & Setup \\
	
	2 & Since \(\gcd(a,p)=1\), multiplication by \(a\) permutes \(S\). That is, \(\{a \cdot 1, a \cdot 2, \dots, a \cdot (p-1)\} \equiv S \pmod{p}\). & \(a\) is invertible mod \(p\) \\
	
	3 & Take the product of all elements in \(S\): \((p-1)! \equiv 1 \cdot 2 \cdots (p-1) \pmod{p}\). & Definition of factorial \\
	
	4 & The product of elements in the permuted set is \(a^{p-1} \cdot (p-1)! \pmod{p}\). & Each factor multiplied by \(a\) \\
	
	5 & Since both products are the same set (up to ordering), we have:
	\(a^{p-1} \cdot (p-1)! \equiv (p-1)! \pmod{p}\). & Equality of products \\
	
	6 & Cancel \((p-1)!\) modulo \(p\) (possible since \(p\) does not divide \((p-1)!\)). & Wilson's theorem not needed, just invertibility of each factor \\
	
	7 & Thus \(a^{p-1} \equiv 1 \pmod{p}\). & Conclusion \\
	
	8 & Alternative group-theoretic proof: \(\mathbb{F}_p^*\) is a group of order \(p-1\), so for any \(a \in \mathbb{F}_p^*\), \(a^{p-1} = 1\). & Lagrange's theorem \\
\end{prooftable}
\end{proof}
\begin{theorem}[Galois Theory of Finite Fields]
	For prime \(p\) and extension \(\mathbb{F}_{p^n}\) of \(\mathbb{F}_p\), the Galois group \(\operatorname{Gal}(\mathbb{F}_{p^n}/\mathbb{F}_p)\) is cyclic of order \(n\), generated by the Frobenius automorphism \(\sigma(x) = x^p\).
\end{theorem}
\begin{proof}
\begin{prooftable}
	1 & Consider the field extension \(\mathbb{F}_{p^n} / \mathbb{F}_p\). It is Galois because finite fields are perfect and normal. & Finite fields properties \\
	
	2 & Define the Frobenius map \(\sigma: \mathbb{F}_{p^n} \to \mathbb{F}_{p^n}\) by \(\sigma(x) = x^p\). & Definition \\
	
	3 & \(\sigma\) is a field automorphism: \(\sigma(x+y) = (x+y)^p = x^p + y^p\) (freshman's dream in characteristic \(p\)), and \(\sigma(xy) = x^p y^p\). & Homomorphism in characteristic \(p\) \\
	
	4 & \(\sigma\) fixes \(\mathbb{F}_p\) because for \(a \in \mathbb{F}_p\), \(a^p = a\) by Fermat's Little Theorem. & Fixed field \\
	
	5 & Thus \(\sigma \in \operatorname{Gal}(\mathbb{F}_{p^n}/\mathbb{F}_p)\). & Definition of Galois group \\
	
	6 & The order of \(\sigma\) divides \(n\): \(\sigma^k(x) = x^{p^k}\) fixes all \(x\) iff \(x^{p^k} = x\) for all \(x \in \mathbb{F}_{p^n}\). & Condition for being identity \\
	
	7 & But \(x^{p^k} = x\) for all \(x\) means \(\mathbb{F}_{p^n} \subseteq \mathbb{F}_{p^k}\), so \(n \mid k\). & Subfield criterion \\
	
	8 & The smallest positive \(k\) with \(\sigma^k = \mathrm{id}\) is \(k = n\). & Since \(\mathbb{F}_{p^n}^\times\) is cyclic of order \(p^n - 1\) \\
	
	9 & Therefore \(\sigma\) has order \(n\) in the Galois group. & Step 8 \\
	
	10 & The degree of extension is \([\mathbb{F}_{p^n} : \mathbb{F}_p] = n\), so \(|\operatorname{Gal}| = n\). & Galois correspondence \\
	
	11 & Since \(\langle \sigma \rangle\) is a subgroup of order \(n\), it must equal the whole Galois group. & Group order argument \\
	
	12 & Hence \(\operatorname{Gal}(\mathbb{F}_{p^n}/\mathbb{F}_p) = \langle \sigma \rangle \cong \mathbb{Z}/n\mathbb{Z}\). & Cyclic group structure \\
\end{prooftable}
\end{proof}
\setstretch{1.7}
\printbibliography[title={References \smallskip }]
\end{document}